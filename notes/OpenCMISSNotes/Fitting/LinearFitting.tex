\section{Linear Fitting}

\subsection{Problem formulation}

The \emph{geometric fitting problem} is, given a set of data defining some
geometry, find the nodal positions and derivatives that, for a given mesh,
minimises the error in the mesh approximation to the data. Hence before a
fitting problem can be formulated the error in the mesh approximation must be
defined.

Consider a set of rectangular Cartesian data with geometric locations
$\vectr{z}^{d}$, $d=1,\ldots,D$. For each data point a point on the mesh
that has the smallest distance to that data point can be found. This point is
termed the \emph{orthogonal projection} of the data point onto the mesh. This
point is specified by a local element coordinate, $\vect{\xi}^{d}$, in
the element of projection, $\varepsilon^{d}$. The geometric location of
this point, $\fnof{\vectr{z}}{\vect{\xi}^{d}}$, is found from
interpolation \ie
\begin{equation}
  \fnof{\vectr{z}}{\vect{\xi}^{d}}=\gbfn{n}{\alpha}{\vectr{\xi}^{d}}
  \vectr{z}^{n}_{,\alpha}\gsf{n}{\alpha}
  \label{eqn:dataprojectionpos}
\end{equation}
A schematic of this projection is shown in \figref{fig:dataprojection}.  To
calculate $\vectr{\xi}^{d}$ a non-linear iterative calculation is
required, details of which are given in \Secref{sec:xipointcalculation}.

\epstexfigure{Fitting/svgs/dataprojection.eps_tex}{Schematic of an orthogonal
  projection of a data point into an element.}{Schematic of an orthogonal
  projection of a data point into an element. A data point at the geometric
  location $\vectr{z}^{d}$ is projected into an element,
  $\varepsilon^{d}$, at a parametric element coordinate, $\vectr{\xi}^{d}$,
  and geometric position $\fnof{\vectr{z}}{\vectr{\xi}^{d}}$ such that the
  distance between the projection and the data point is
  minimised.}{fig:dataprojection}{0.75}

The measure of the error for each data point, $f^{d}$, is defined as the
Euclidean distance between the data point and its closest projection onto the
current mesh \ie
\begin{equation}
  \fnof{f^{d}}{\vectr{\xi}^{d}}=\norm{\fnof{\vectr{z}}{\vect{\xi}^{d}}-\vect{z}^{d}}
\end{equation}

For a given projection of the data points onto the mesh (\ie if
$\vectr{\xi}^{d}$ are held constant), the \emph{data objective} function
to be minimised in the fit is formed from the weighted sum of squares of the
individual errors \ie
\begin{equation}
  \fnof{\mathcal{F}}{\vectr{u}}=\gsum{e=1}{E}{\gsum{d=1}{D_{e}}{\gamma^{d}
      \pbrac{\fnof{f^{d}}{\vectr{\xi}^{d}}}^{2}}}=\gsum{e=1}{E}{
    \gsum{d=1}{D_{e}}{\gamma^{d}\norm{\fnof{\vectr{z}}{\vectr{\xi}^{d}}-\vectr{z}^{d}}^{2}}}
  \label{eqn:ldataresidualfn}
\end{equation}
where $\gamma^{d}$ is a weight for each data point, $D_{e}$ the number of
data points within an element $e$ and $\vectr{u}$ is a vector of mesh
parameters (nodal positions, derivatives and scale factors).

The fitting problem is hence to find the set of mesh parameters that minimises
this objective function. Substituting \eqnref{eqn:dataprojectionpos} into
\eqnref{eqn:ldataresidualfn} and differentiating with respect to the mesh
parameters within an element $e$ yields
\begin{equation}
  \begin{split}
    \delby{\fnof{\mathcal{F}}{\vect{u}}}{z^{m}_{j,\alpha}}&=2
    \gsum{d=1}{D_{e}}{\gamma^{d}
      \pbrac{\idxgbfn{j}{n}{\beta}{\vectr{\xi}^{d}}z^{n}_{j,\beta}\idxgsf{j}{n}{\beta}-z^{d}_{j}}
      \idxgbfn{j}{m}{\alpha}{\vectr{\xi}^{d}}\idxgsf{j}{m}{\alpha}} \\
    \delby{\fnof{\mathcal{F}}{\vect{u}}}{\gsf{m}{\alpha}}&=
      2\gsum{d=1}{D_{e}}{\gamma^{d}\dotprod{\pbrac{\gbfn{n}{\beta}{\vectr{\xi}^{d}}
          \vectr{z}^{n}_{,\beta}\gsf{n}{\beta}-
          \vectr{z}^{d}}}{\gbfn{m}{\alpha}{\vectr{\xi}^{d}}
          \vectr{z}^{m}_{,\alpha}}}
  \end{split}
  \label{eqn:derivldataresidualfn}
\end{equation}

A minimum of the objective function is hence found by setting the derivatives
of the objective function in \eqnref{eqn:derivldataresidualfn} to zero.  This
will only result in a linear system if the scale factors are kept constant
during the fit. That is, the vector $\vectr{u}$ will only contain the nodal
positions and the nodal arc-length derivatives.  With this restriction the
derivative of \eqnref{eqn:ldataresidualfn} with respect to the scale factors
can be ignored and the minimum of the objective function is given by
\begin{equation}
  \gsum{d=1}{D_{e}}{\gamma^{d}\idxgbfn{j}{m}{\alpha}{\vectr{\xi}^{d}}
    \idxgsf{j}{m}{\alpha}\idxgbfn{j}{n}{\beta}{\vectr{\xi}^{d}}\idxgsf{j}{n}{\beta}}
  z^{n}_{j,\alpha}=\gsum{d=1}{D_{e}}{\gamma^{d}\idxgbfn{j}{m}{\alpha}
    {\vectr{\xi}^{d}}\idxgsf{j}{m}{\alpha}z^{d}_{j}}
\end{equation}
that is, a linear elemental finite element stiffness matrix and residual
vector system of the form
\begin{equation}
  K^{\alpha\beta}_{jmn}z^{n}_{j,\beta}=r^{\alpha}_{jm}
  \label{eqn:fittingelemstiffnessystem}
\end{equation}
where
\begin{equation}
  K^{\alpha\beta}_{jmn}=\idxgsf{j}{m}{\alpha}\idxgsf{j}{n}{\beta}\gsum{d=1}{D_{e}}{
      \gamma^{d}\idxgbfn{j}{m}{\alpha}{\vectr{\xi}^{d}}\idxgbfn{j}{n}{\beta}{\vectr{\xi}^{d}}}
\end{equation}
and
\begin{equation}
  r^{\alpha}_{jm}=\idxgsf{j}{m}{\alpha}\gsum{d=1}{D_{e}}{\gamma^{d}\idxgbfn{j}{m}{\alpha}
    {\vectr{\xi}^{d}}z^{d}_{j}}
\end{equation}

A linear system of equations governing the entire mesh is found by assembling
a global stiffness matrix from all of the individual element matrices. This is
then solved to yield the nodal positions and derivatives that minimise the
error in the mesh.

As the scale factors and data point projections are held constant during the
above fitting procedure the resultant mesh may not completely minimise the
error. It is hence necessary to iteratively apply the fitting procedure in
order to obtain a mesh that best fits the data. The fit can be considered
converged when the RMS error in the fit does not change significantly
between iterations. The RMS error in the fit is defined as
\begin{equation}
  \text{RMS error}=\sqrt{\dfrac{\gsum{d=1}{D}{\norm{\fnof{\vectr{z}}{
            \vectr{\xi}^{d}}-\vect{z}^{d}}^{2}}}{D}}
\end{equation}

The linear fitting algorithm used in this thesis is hence:
\begin{enumerate}
\item Define an initial mesh (and calculate the initial nodal scale factors).
  \item Calculate the initial data point projections and initial error in the
    mesh.
  \item Repeat until converged or the maximum number of iterations is exceeded.
  \begin{enumerate}
  \item Fit the mesh by applying the linear fitting procedure.
  \item Update the scale factors to be average \arclens based on the new
    mesh.
  \item Recalculate the data point projections on the new mesh.
  \end{enumerate}
\end{enumerate}

\subsection{$\xi$ point calculation}
\label{sec:xipointcalculation}

The fitting method described above requires the location of the orthogonal
projection of a data point (the point at which the distance between the data
point and the projection is a minimum) onto the mesh to be determined. To find
this point consider the distance between a data point and it's projection,
$\fnof{f^{d}}{\vectr{\xi}}$, as the Euclidean norm of the error vector from some point (at a
parametric elemental position of $\vectr{\xi}$), $\fnof{\vectr{z}}{\vectr{\xi}}$,
and the location of the data point, $\vectr{z}^{d}$ \ie
\begin{equation}
  \begin{split}
    \fnof{f^{d}}{\vectr{\xi}}&=\norm{\fnof{\vectr{e}^{d}}{\vect{\xi}}} \\ 
    &=\norm{\fnof{\vect{z}}{\vect{\xi}}-\vect{z}^{d}} \\ 
    &=\norm{\gbfn{n}{\beta}{\vectr{\xi}}\vect{z}^{n}_{,\beta}
      \gsf{n}{\beta}{u}-\vect{z}^{d}}
  \end{split}
\end{equation}

The parametric elemental location which minimises this distance (assuming
that it exists within the element), $\vect{\xi}^{d}$, is hence given by
\begin{equation}
  \delby{\fnof{f^{d}}{\vectr{\xi}^{d}}}{\xi_{l}}=
  \dfrac{\delby{\gbfn{n}{\beta}{\vectr{\xi}^{d}}}{\xi_{l}}
    \dotprod{\vectr{z}^{n}_{,\beta}\gsf{n}{\beta}}{\fnof{\vectr{e}^{d}}{\vectr{\xi}^{d}}}}{
    \norm{\fnof{\vectr{e}^{d}}{\vectr{\xi}^{d}}}}=0
\end{equation}
with $l$ varying from $1$ to the number of $\xi$ coordinates in the mesh under
consideration.

The parametric projection location, $\vectr{\xi}^{d}$, can hence be found
from the (vector) root of the (vector) function
\begin{equation}
  \fnof{g_{l}}{\vect{\xi}}=\dotprod{\delby{\fnof{\vect{z}}{\vectr{\xi}}}{
      \xi_{l}}}{\fnof{\vectr{e}^{d}}{\vectr{\xi}}}=0
\end{equation}

This root is easily found using an iterative Newton-Raphson method. If an
initial location, $\vectr{\xi}$, is assumed a correction to this position,
$\Delta\vectr{\xi}$, can be found from 
\begin{equation}
  \Delta\vectr{\xi}=-\inverse{\fnof{\matr{J}}{\fnof{\vectr{g}}{\vectr{\xi}}}}
  \fnof{\vectr{g}}{\vectr{\xi}}
\end{equation}
where $\fnof{\matr{J}}{\fnof{\vectr{g}}{\vectr{\xi}}}$ is the \emph{Jacobian
  matrix} given by
\begin{equation}
  \fnof{\matr{J}}{\fnof{\vectr{g}}{\vect{\xi}}}=\begin{bmatrix}
    \delby{g_{1}}{\xi_{1}} & \delby{g_{1}}{\xi_{2}} & \ldots & 
    \delby{g_{1}}{\xi_{n}} \\
    \delby{g_{2}}{\xi_{1}} & \delby{g_{2}}{\xi_{2}} & \ldots & 
    \delby{g_{2}}{\xi_{n}} \\
    \vdots & \vdots & \ddots & \vdots \\
    \delby{g_{m}}{\xi_{1}} & \delby{g_{m}}{\xi_{2}} & \ldots & 
    \delby{g_{m}}{\xi_{n}}
  \end{bmatrix}
\end{equation}

For the function considered here
\begin{equation}
  \fnof{J_{ij}}{\fnof{\vectr{g}}{\vectr{\xi}}}=\dotprod{\deltwoby{\fnof{
        \vectr{z}}{\vectr{\xi}}}{\xi_{i}}{\xi_{j}}}{\fnof{\vect{e}^{d}}{
      \vectr{\xi}}}+\dotprod{\delby{\fnof{\vectr{z}}{\vectr{\xi}}}{\xi_{i}}}{
    \delby{\fnof{\vectr{e}^{d}}{\vectr{\xi}}}{\xi_{j}}}
\end{equation}

The algorithm to find the orthogonal projections is hence:
\begin{enumerate}
\item For each data point, d.
  \begin{enumerate}
  \item For each element.
    \begin{enumerate}
    \item Guess an initial starting location for the projection (\eg the
      middle of the element).
    \item Iteratively apply the Newton-Raphson method until (a) The projection
      is converged; (b) The maximum number of Newton-Raphson iterations is
      exceeded; (c) The projection moves outside the element.
    \item If a projection has been found inside the element and the distance
      from this projection to the data point is smaller than other projections
      in other elements set $\varepsilon^{d}$ to this element and
      $\vect{\xi}^{d}$ to this projection.
    \end{enumerate}
  \end{enumerate}
\end{enumerate}

The initial calculation of the orthogonal projections is computationally
expensive although the method is parallelisable at the data point loop level
and steps can be taken to ensure that excessive
computation is avoided (\eg immediately disregard elements at early stages in
the Newton-Raphson iteration whose projections yield distances far greater
than the current smallest projection). Once the initial projections have been
calculated then, if the mesh is changed (as part of an iterative fitting
procedure), the projections can be quickly and easily updated by applying the
Newton-Raphson method from the current projections.

\subsection{Sobolev smoothing for linear fitting}
\label{sec:sobolevlinearfitting}

With an insufficient number of data points, fitting 'noisy' data or fitting
data that has an uneven spread, a smoothness constraint \cite{young:1989}
can be introduced by adding a second term to the objective function:
\begin{equation}
  \fnof{F}{\vectr{u}}=\gsum{e=1}{E}{\gsum{d=1}{D_{e}}{\gamma^{d}
    \norm{\fnof{\vectr{z}}{\vectr{\xi}^{d}}-\vectr{z}^{d}}^{2}}}+
  \gint{\Omega}{}{\fnof{g}{\fnof{\vectr{u}}{\vectr{\xi}}}}{\Omega}
  \label{eqn:soboleverrorfn}
\end{equation}

The first term in \eqnref{eqn:soboleverrorfn} measures the error in the
surface from the data and the second term measures deformation of the surface.
To measure the deformation of the surface a $\nth{p}$ order weighted
Sobolev norm \cite{terzopoulos:1986,zzz-tikhonov:1977} is used, defined by
\begin{equation}
  \fnof{g_{p,w}}{\fnof{\vectr{u}}{\xi}}=\gsum{q=0}{p}{\gsum{i=q}{}
  {w_{i}\norm{\delnby{q}{\vectr{u}}{\xi}{i}}^{2}}}
  \label{eqn:1DSobolevnorm}
\end{equation}
for one $\xi$ direction and
\begin{equation}
  \fnof{g_{p,w}}{\fnof{\vectr{u}}{\vectr{\xi}}}=\gsum{q=0}{p}{\gsum{i+j=q}{}
  {w_{ij}\norm{\delntwoby{q}{\vectr{u}}{\xione}{i}{\xitwo}{j}}^{2}}}
  \label{eqn:2DSobolevnorm}
\end{equation}
for two $\xi$ directions and
\begin{equation}
  \fnof{g_{p,w}}{\fnof{\vectr{u}}{\vectr{\xi}}}=\gsum{q=0}{p}{\gsum{i+j+k=q}{}
  {w_{ijk}\norm{\delnthreeby{q}{\vectr{u}}{\xione}{i}{\xitwo}{j}{\xithree}{k}}^{2}}}
  \label{eqn:3DSobolevnorm}
\end{equation}
for three $\xi$ directions and where $w_{i}$, $w_{ij}$, and $w_{ijk}$ are the weights for the norm.

The addition to the objective function, called the \emph{Sobolev value}, is
defined as
\begin{equation}
  \fnof{G}{\fnof{\vectr{u}}{\vectr{\xi}}}=\gint{\Omega}{}{\fnof{g}{
      \fnof{\vectr{u}}{\vectr{\xi}}}}{\Omega}
  \label{eqn:Sobolevvalue}
\end{equation}
where $\Omega$ is the mesh domain.

Consider the case for $p=2$, and for one $\xi$ dimension: $w_{0}=0,
w_{1}=\tau, w_{2}=\kappa$; and for two $\xi$ dimensions: $w_{00}=0,
w_{01}=w_{10}=\tau, w_{20}=w_{02}= \kappa, w_{11}=2\kappa$; and for three
$\xi$ dimensions: $w_{000}=0, w_{100}=w_{010}=w_{001}=\tau,
w_{200}=w_{020}=w_{002}= \kappa, w_{110}=w_{101}=w_{011}=2\kappa$. In one
$\xi$ dimension the Sobolev value now becomes
\begin{equation}
  \fnof{G}{\fnof{\vectr{u}}{\vectr{\xi}}}=\goneint{\pbrac{\tau
      \norm{\dby{\vectr{u}}{\xi}}^{2}+\kappa\norm{\dtwosqby{
          \vectr{u}}{\xi}}^{2}}}{\Omega}
  \label{eqn:1DptwoSobolevnorm}
\end{equation}
and in two $\xi$ dimensions it becomes
\begin{equation}
  \fnof{G}{\fnof{\vectr{u}}{\vectr{\xi}}}=\goneint{\pbrac{\tau\pbrac{
        \norm{\delby{\vectr{u}}{\xione}}^{2}+\norm{\delby{\vect{u}}
          {\xitwo}}^{2}}+\kappa\pbrac{\norm{\deltwosqby{\vectr{u}}{\xione}}^{2}+
        2\norm{\deltwoby{\vectr{u}}{\xione}{\xitwo}}^{2}+
        \norm{\deltwosqby{\vectr{u}}{\xitwo}}^{2}}}}{\Omega}
  \label{eqn:2DptwoSobolevnorm}
\end{equation}
and in three $\xi$ dimensions it becomes
\begin{multline}
  \fnof{G}{\fnof{\vectr{u}}{\vectr{\xi}}}=\dintl{\Omega}{}\left(\tau\pbrac{
        \norm{\delby{\vectr{u}}{\xione}}^{2}+\norm{\delby{\vect{u}}
          {\xitwo}}^{2}+\norm{\delby{\vect{u}}
          {\xithree}}^{2}}\right. \\
    +\left.\kappa\pbrac{\norm{\deltwosqby{\vectr{u}}{\xione}}^{2}+
        2\norm{\deltwoby{\vectr{u}}{\xione}{\xitwo}}^{2}+
        \norm{\deltwosqby{\vectr{u}}{\xitwo}}^{2}+
        2\norm{\deltwoby{\vectr{u}}{\xione}{\xithree}}^{2}+
        2\norm{\deltwoby{\vectr{u}}{\xitwo}{\xithree}}+
        \norm{\deltwosqby{\vectr{u}}{\xithree}}^{2}}\right)\,\exteriorderivop\Omega
  \label{eqn:3DptwoSobolevnorm}
\end{multline}

Thus the parameter $\tau$ controls the tension of the surface and the
parameter $\kappa$ controls the degree of surface curvature
\cite{terzopoulos:1986}.

Now if we smoothing based on the value of the fitted field \ie
$\vectr{u}\equiv\vectr{z}$, then the additional elemental stiffness matrix
terms for \eqnref{eqn:fittingelemstiffnessystem} are
\begin{equation}
  K^{\alpha\beta}_{jmn}=2\idxgsf{j}{m}{\alpha}\idxgsf{j}{n}{\beta}
  \gint{0}{1}{\pbrac{\tau\delby{\idxgbfn{j}{m}{\alpha}{\xi}}{\xi}\delby{\idxgbfn{j}{n}{\beta}{\xi}}{\xi}+
      \kappa\deltwosqby{\idxgbfn{j}{m}{\alpha}{\xi}}{\xi}
      \deltwosqby{\idxgbfn{j}{n}{\beta}{\xi}}{\xi}}\abs{\fnof{J}{\xi}}}{\xi}
\end{equation}
for one $\xi$ dimension and 
\begin{multline}
  K^{\alpha\beta}_{jmn}=2\idxgsf{j}{m}{\alpha}\idxgsf{j}{n}{\beta}
  \dintl{\vect{0}}{\vect{1}}\left(\tau\pbrac{\delby{\idxgbfn{j}{m}{\alpha}{\vectr{\xi}}}{\xione}
      \delby{\idxgbfn{j}{n}{\beta}{\vectr{\xi}}}{\xione}+
      \delby{\idxgbfn{j}{m}{\alpha}{\vectr{\xi}}}{\xitwo}
      \delby{\idxgbfn{j}{n}{\beta}{\vectr{\xi}}}{\xitwo}}+\right. \\
  \left.\kappa\pbrac{\deltwosqby{\idxgbfn{j}{m}{\alpha}{\vectr{\xi}}}{\xione}
      \deltwosqby{\idxgbfn{j}{n}{\beta}{\vectr{\xi}}}{\xione}
      +2\deltwoby{\idxgbfn{j}{m}{\alpha}{\vectr{\xi}}}{\xione}{\xitwo}
      \deltwoby{\idxgbfn{j}{n}{\beta}{\vectr{\xi}}}{\xione}{\xitwo}
      +\deltwosqby{\idxgbfn{j}{m}{\alpha}{\vectr{\xi}}}{\xitwo}
      \deltwosqby{\idxgbfn{j}{n}{\beta}{\vectr{\xi}}}{\xitwo}}\right)
  \abs{\fnof{\matr{J}}{\vectr{\xi}}}\,\exteriorderivop\vect{\xi}
\end{multline}
for two $\xi$ dimensions and
\begin{multline}
  K^{\alpha\beta}_{jmn}=2\idxgsf{j}{m}{\alpha}\idxgsf{j}{n}{\beta}
  \dintl{\vect{0}}{\vect{1}}\left(\tau\pbrac{\delby{\idxgbfn{j}{m}{\alpha}{\vectr{\xi}}}{\xione}
      \delby{\idxgbfn{j}{n}{\beta}{\vectr{\xi}}}{\xione}
      +\delby{\idxgbfn{j}{m}{\alpha}{\vectr{\xi}}}{\xitwo}
      \delby{\idxgbfn{j}{n}{\beta}{\vectr{\xi}}}{\xitwo}
      +\delby{\idxgbfn{j}{m}{\alpha}{\vectr{\xi}}}{\xithree}
      \delby{\idxgbfn{j}{n}{\beta}{\vectr{\xi}}}{\xithree}}\right. \\
  +\kappa\left(\deltwosqby{\idxgbfn{j}{m}{\alpha}{\vectr{\xi}}}{\xione}
      \deltwosqby{\idxgbfn{j}{n}{\beta}{\vectr{\xi}}}{\xione}+
      2\deltwoby{\idxgbfn{j}{m}{\alpha}{\vectr{\xi}}}{\xione}{\xitwo}
      \deltwoby{\idxgbfn{j}{n}{\beta}{\vectr{\xi}}}{\xione}{\xitwo}+
      \deltwosqby{\idxgbfn{j}{m}{\alpha}{\vectr{\xi}}}{\xitwo}
      \deltwosqby{\idxgbfn{j}{n}{\beta}{\vectr{\xi}}}{\xitwo}\right. \\
      \left.\left.+2\deltwoby{\idxgbfn{j}{m}{\alpha}{\vectr{\xi}}}{\xione}{\xithree}
      \deltwoby{\idxgbfn{j}{n}{\beta}{\vectr{\xi}}}{\xione}{\xithree}
       +2\deltwoby{\idxgbfn{j}{m}{\alpha}{\vectr{\xi}}}{\xitwo}{\xithree}
      \deltwoby{\idxgbfn{j}{n}{\beta}{\vectr{\xi}}}{\xitwo}{\xithree}
      +\deltwosqby{\idxgbfn{j}{m}{\alpha}{\vectr{\xi}}}{\xithree}
      \deltwosqby{\idxgbfn{j}{n}{\beta}{\vectr{\xi}}}{\xithree}\right)\right)
  \abs{\fnof{\matr{J}}{\vectr{\xi}}}\,\exteriorderivop\vect{\xi}
\end{multline}
for three $\xi$ dimensions.

It should be noted that basing the smooth on the value of the fitted field can
cause problems and high smoothing weights as the resulting fitted geometry can
shrink inside the data. To avoid this problem we can base the smoothing on the
difference between the fitted and unfitted fields \ie
$\vectr{u}\equiv\vectr{z}-\vectr{x}$ where $\vectr{x}$ is the unfitted
geometric field. For this case then in addition to the
stiffness matrix terms above, the addition terms for the right-hand side
vector in \eqnref{eqn:fittingelemstiffnessystem} are
\begin{equation}
  r^{\alpha}_{jm}=2\idxgsf{j}{m}{\alpha}\gint{0}{1}{\pbrac{
      \tau\delby{\idxgbfn{j}{m}{\alpha}{\xi}}{\xi}\delby{\fnof{x_{i}}{\xi}}{\xi}e^{i}+
  \kappa\deltwosqby{\idxgbfn{j}{m}{\alpha}{\xi}}{\xi}\deltwosqby{\fnof{x_{i}}{\xi}}{\xi}e^{i}}}{\xi}
\end{equation}
for one $\xi$ dimension and where $\vectr{e}$ is the vector of all ones. We
also have
\begin{multline}
  r^{\alpha}_{jm}=2\idxgsf{j}{m}{\alpha}\dintl{\vect{0}}{\vect{1}}\left(
  \tau\pbrac{\delby{\idxgbfn{j}{m}{\alpha}{\vectr{\xi}}}{\xione}\delby{\fnof{x_{i}}{\vectr{\xi}}}{\xione}e^{i}+
    \delby{\idxgbfn{j}{m}{\alpha}{\vectr{\xi}}}{\xitwo}\delby{\fnof{x_{i}}{\vectr{\xi}}}{\xitwo}e^{i}}\right. \\
  \left.+\kappa\pbrac{\deltwosqby{\idxgbfn{j}{m}{\alpha}{\vectr{\xi}}}{\xione}\deltwosqby{\fnof{x_{i}}{\vectr{\xi}}}{\xione}e^{i}
    +2\deltwoby{\idxgbfn{j}{m}{\alpha}{\vectr{\xi}}}{\xione}{\xitwo}\deltwoby{\fnof{x_{i}}{\vectr{\xi}}}{\xione}{\xitwo}e^{i}
    +\deltwosqby{\idxgbfn{j}{m}{\alpha}{\vectr{\xi}}}{\xitwo}\deltwosqby{\fnof{x_{i}}{\vectr{\xi}}}{\xitwo}e^{i}
  }\right)\abs{\fnof{\matr{J}}{\vectr{\xi}}}\,\exteriorderivop\vect{\xi}
\end{multline}
for two $\xi$ dimensions and
\begin{multline}
  r^{\alpha}_{jm}=2\idxgsf{j}{m}{\alpha}\dintl{\vect{0}}{\vect{1}}\left(
  \tau\pbrac{\delby{\idxgbfn{j}{m}{\alpha}{\vectr{\xi}}}{\xione}\delby{\fnof{x_{i}}{\vectr{\xi}}}{\xione}e^{i}+
    \delby{\idxgbfn{j}{m}{\alpha}{\vectr{\xi}}}{\xitwo}\delby{\fnof{x_{i}}{\vectr{\xi}}}{\xitwo}e^{i}+
    \delby{\idxgbfn{j}{m}{\alpha}{\vectr{\xi}}}{\xithree}\delby{\fnof{x_{i}}{\vectr{\xi}}}{\xithree}e^{i}}\right. \\
  +\kappa\left(\deltwosqby{\idxgbfn{j}{m}{\alpha}{\vectr{\xi}}}{\xione}\deltwosqby{\fnof{x_{i}}{\vectr{\xi}}}{\xione}e^{i}+
  2\deltwoby{\idxgbfn{j}{m}{\alpha}{\vectr{\xi}}}{\xione}{\xitwo}\deltwoby{\fnof{x_{i}}{\vectr{\xi}}}{\xione}{\xitwo}e^{i}+
  \deltwosqby{\idxgbfn{j}{m}{\alpha}{\vectr{\xi}}}{\xitwo}\deltwosqby{\fnof{x_{i}}{\vectr{\xi}}}{\xitwo}e^{i}\right. \\
  \left.\left.+2\deltwoby{\idxgbfn{j}{m}{\alpha}{\vectr{\xi}}}{\xione}{\xithree}\deltwoby{\fnof{x_{i}}{\vectr{\xi}}}{\xione}{\xithree}e^{i}+
  2\deltwoby{\idxgbfn{j}{m}{\alpha}{\vectr{\xi}}}{\xitwo}{\xithree}\deltwoby{\fnof{x_{i}}{\vectr{\xi}}}{\xitwo}{\xithree}e^{i}+
  \deltwosqby{\idxgbfn{j}{m}{\alpha}{\vectr{\xi}}}{\xithree}\deltwosqby{\fnof{x_{i}}{\vectr{\xi}}}{\xithree}e^{i}
  \right)\right)\abs{\fnof{\matr{J}}{\vectr{\xi}}}\,\exteriorderivop\vect{\xi}
\end{multline}
for three $\xi$ dimensions.
