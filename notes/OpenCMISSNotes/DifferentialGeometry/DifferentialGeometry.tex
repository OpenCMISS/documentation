\chapter{Differential Geometry}
\label{cha:differentialgeometry}

\section{Introduction}

\section{Topology}

The topology of a space specifies the connectedness, continuity, compactness,
and other properties that can be defined on the space of specified
dimension. It brings in the concept of open neighbourhoods of points. Types of
topology include

\begin{enumerate}
\item Discrete (OD) topology, $\dntopology{}$.
\item Real topology, $\rntopology{n}$.
\item Complex topology, $\cntopology{n}$.
\item Spherical topology, $\sntopology{n}$.
\end{enumerate}

\section{Manifold}

A manifold $\manifold{M}$ is a topological space (\ie a set of points
with a specified topology) of a fixed dimension $n$ in which every
point is homeomorphic to $\rntopology{n}$ \ie they admit a coordinate
system that is locally Euclidean. For our purposes we need to relax
this last restriction because we wish to allow branching structures,
but we will still refer to this as a manifold – the neighbourhoods of
points at the branch do not admit a coordinate system that is locally
Euclidean.

\section{Vectors and Covectors}

To motivate this section consider standing on a hill as show in
\figref{fig:hillvectors}. Now, consider two vectors that can be
defined. If we move around the hill at constant altitude we can talk
about our tangential velocity vector around the hill. We can also talk
about moving down the slope of the hill. The gradient vector would
give the steepest direction down the hill.

\epstexfigure{DifferentialGeometry/svgs/hillvectors.eps_tex}{Differential
  hill vectors.}{Imagine a person standing on a hill. Two vectors
  could be considered in this situation. The first vector would be
  found by moving around the hill at a constant altitude - this would
  be the tangential velocity vector. The second vector would be
  gradient vector which corresponds to moving down the hill in the
  steepest direction.}{fig:hillvectors}{0.5}

If both these vectors had a unit magnitude in what ever units of
measurement we are using we might consider these vectors to be similar
(apart from the obvious difference in direction). Now consider a
change of units \eg from $\m$ to $\cm$. If our tangential velocity was
prevoiusly $\nunit{1}{\mps}$ then it would be now
$\nunit{100}{\cmps}$. However, our gradient vector that was previously
$\nunit{1}{\pmetre}$ would become $\nunit{0.01}{\pcm}$. If these two
``vectors'' are really so similar why do the transform differently
under a change of units/coordinates?  The reason is that,
geometrically, these two ``vectors'' are not actually that similar at
all. Indeed, one of them is not a vector at all but rather it is a
\emph{covector}.

\section{Differential geometry of manifolds}

\subsection{De Rham Complex}


The \emph{de Rham complex}\footnote{named after
\link{https://en.wikipedia.org/wiki/Georges_de_Rham}{Georges de Rham}
(1903-1990), a Swiss mathematician.}\index{de Rham complex} can be
view graphically in \figref{fig:DeRhamComplex}.

\epstexfigure{DifferentialGeometry/svgs/diffformcomplex.eps_tex}{De
  Rham Complex.}{The de Rham complex gives the relationship between
  differential forms. $\hodgestarop$ is the Hodge star operator,
  $\inverse{\hodgestarop}$ is the inverse Hodge star operator,
  $\exteriorderivop$ is the external differential operator, and
  $\exteriorcoderivop$ is the exterior co-differential
  operator.}{fig:DeRhamComplex}{1.0}

\subsection{Connections}


\subsection{Covariant Differentiation}

Consider transforming a constant vector $\vectr{a}$ by moving it
through $\Delta\theta$ in polar coordinates to the vector
$\hat{\vectr{a}}$ as shown in
\figref{fig:CovariantDerivativeConstantVector}.

\epstexfigure{DifferentialGeometry/svgs/covariantderivvector.eps_tex}{Changes
  of a vector due to changes in the coordinate system.}{Transformation
  of a constant vector $\vectr{a}$ to $\hat{\vectr{a}}$ by moving it
  through $\Delta\theta$ in a polar coordinate system. The
  transformation shows that $\delby{a^{r}}{\theta}$ is non-zero even
  though the physical vector $\vectr{a}$ is
  constant.}{fig:CovariantDerivativeConstantVector}{0.5}

A covariant derivative is a derivative along a tangent vector of a
manifold. It is a generalisation of a directional derivative to manifolds. It
is a derivative that under a coordinate transforms covariantly \ie linearly
with the Jacobian. A covariant derivative is equivalent to the idea of a
connection. For a connection $\connection{}{}$ and a vector field $X$ then
$\covarderivop{X}{}$ is the covariant derivative. 

The covariant derivative of a $\pbrac{r,s}$ tensor is an $\pbrac{r,s+1}$
tensor defined by
\begin{equation}
  \covarderivop{X}{Y}=\covarderivop{X}{\pbrac{X^{i}e_{i}}}=Y\pbrac{X^{i}}e_{i}+X^{i}\covarderivop{Y}{e_{i}}
\end{equation}

The covariant derivative has the following properties
%\begin{list}
%\item Sumation Rule: $abc$ \\
%\item kaldf
%\end{list}

\subsubsection{Double covariant differentiation}

For an $\pbrac{r,s}$ tensor field, $T$, the double covariant derivative is
\begin{equation}
  \doublecovarderivop{}{T}=\covarderivop{}{\covarderivop{}{T}}
\end{equation}
which is an $\pbrac{r,s+2}$ tensor field. It is defined by
\begin{equation}
  \doublecovarderivop{X,Y}{T}=\covarderivop{X}{\covarderivop{Y}{T}}-\covarderivop{\connection{X}{Y}}{T}
\end{equation}

\section{Tensor Analysis}
\subsection{Base vectors}

Now, if we have a vector, $\vectr{v}$ we can write
\begin{equation}
  \vectr{v}=v^{i}\generalbasevector_{i}
\end{equation}
where $v^{i}$ are the components of the contravariant vector, and
$\generalbasevector_{i}$ are the covariant base vectors.

Similarly, the vector $\vectr{v}$ can also be written as 
\begin{equation}
  \vectr{v}=v_{i}\generalbasevector^{i}
\end{equation}
where $v_{i}$ are the components of the covariant vector, and
$\generalbasevector^{i}$ are the contravariant base vectors. 

We now note that
\begin{equation}
  \vectr{v}=v^{i}\generalbasevector_{i}=v^{i}\sqrt{g_{ii}}\hat{\generalbasevector_{i}}
\end{equation}
where $v^{i}\sqrt{g_{ii}}$ are the physical components of the vector and
$\hat{\generalbasevector_{i}}$ are the unit vectors given by
\begin{equation}
  \hat{\generalbasevector_{i}}=\dfrac{\generalbasevector_{i}}{\sqrt{g_{ii}}}
\end{equation}

\subsection{Metric Tensors}
\label{sec:metric tensors}

Metric tensors are the inner product of base vectors. If $\generalbasevector_{i}$ are the
covariant base vectors then the covariant metric tensor is given by
\begin{equation}
  g_{ij}=\dotprod{\generalbasevector_{i}}{\generalbasevector_{j}}
\end{equation}

Similarily if $\generalbasevector^{i}$ are the contravariant base vectors then the
contravariant metric tensor is given by 
\begin{equation}
  g^{ij}=\dotprod{\generalbasevector^{i}}{\generalbasevector^{j}}
\end{equation}

We can also form a mixed metric tensor from the dot product of a contravariant
and a covariant base vector \ie
\begin{equation}
  g^{i}_{.j}=\dotprod{\generalbasevector^{i}}{\generalbasevector_{j}}
\end{equation}
and 
\begin{equation}
  g_{i}^{.j}=\dotprod{\generalbasevector_{i}}{\generalbasevector^{j}}
\end{equation}

Note that for mixed tensors the ``.'' indicates the order of the index \ie
$g^{i}_{.j}$ indicates that the first index is contravariant and the second
index is covariant whereas $g_{i}^{.j}$ indicates that the first index is
covariant and the second index is contravariant.

If the base vectors are all mutually orthogonal and constant then
$\generalbasevector_{i}=\generalbasevector^{i}$ and $g_{ij}=g^{ij}$.

The metric tensors generalise (Euclidean) distance \ie
\begin{equation}
  ds^{2}=g_{ij}dx^{i}dx^{j}
\end{equation}

\subsubsection{Raising and lowering indices}

Note that multiplying by the covariant metric tensor lowers indices \ie
\begin{equation}
  \begin{split}
    \vectr{A}_{i} &= g_{ij}\vectr{A}^{j} \\
    A_{ij} &= g_{ik}g_{jl}A^{kl} = g_{jk}A_{i}^{.k} = g_{ik}A^{k}_{.j} 
  \end{split}
\end{equation}
and that multiplying by the contravariant metric tensor raises indices \ie
\begin{equation}
  \begin{split}
  \vectr{A}^{i} &=  g^{ij}\vectr{A}_{j} \\
   A^{ij} &= g^{ik}g^{jl}A_{kl} = g^{ik}A_{k}^{.j} = g^{jk}A^{i}_{.k}
  \end{split}
\end{equation}
and for the mixed tensors
\begin{equation}
  \begin{split}
  A_{i}^{.j} &= g^{jk}A_{ik} = g_{ik}A^{kj} \\
  A^{i}_{.j} &= g^{ik}A_{kj} = g_{jk}A^{ik} \\
  \end{split}
\end{equation}

We can denote a tensor in which all indicies have been raised as a
\emph{sharp} tensor, $\sharptensor{\tensor{A}}$, and one in which all indicies have been lowered as a
\emph{flat} tensor, $\flattensor{\tensor{A}}$. This is known as \emph{musical isomorphism} \ie
isomorphism between the tangent and cotangent bundles of a manifold, $\manifold{M}$.
\begin{equation}
  \mapping{\sharptensor{}}{\cotangentbundle{M}}{\tangentbundle{M}}
\end{equation}
and
\begin{equation}
  \mapping{\flattensor{}}{\tangentbundle{M}}{\cotangentbundle{M}}
\end{equation}

\subsubsection{Induced metric}

An induced metric is the metric tensor induced on a submanifold that has been
embedded into a larger manifold with a metric. If $\vectr{\xi}$ are the
coordinates in the submanifold and $\fnof{\vectr{x}}{\vectr{\xi}}$ are the
functions which embedded the submanifold into a larger manifold then the
induced metric is given by
\begin{equation}
  g_{ab}=\delby{x^{\mu}}{\xi^{a}}\delby{x^{\nu}}{\xi^{b}}g_{\mu\nu}
\end{equation}
where $a, b$ are the coordinate indices in the submanifold, $\mu, \nu$ are the
coordinate indices in the larger manifold, $g_{\mu\nu}$ are the components of
the metric tensor in the larger manifold and $g_{ab}$ are the components of
the induced metric in the submanifold.

\subsubsection{Arc length}

\subsubsection{Angle between vectors}

The angle between two vectors $\vectr{u}$ and $\vectr{v}$ is given by
\begin{equation}
  \cos\theta=\dfrac{\dotprod{\vectr{u}}{\vectr{v}}}{\norm{\vectr{u}}\norm{\vectr{v}}}=
  \dfrac{g_{ij}u^{i}v^{j}}{\sqrt{g_{pq}u^{p}v^{q}}\sqrt{g_{rs}u^{r}v^{s}}}
  \label{eqn:AngleBetweenVectors}
\end{equation}

\subsubsection{Properties of the metric tensor}

A metric tensor $\generalmetrictensor$ has the following properties
\begin{align}
  \divergence{}{\sharptensor{\generalmetrictensor}}=0
  \label{eqn:DivergenceSharpMetricTensor}
\end{align}

\subsection{Transformations}

The transformation rules for tensors in going from a $\vectr{\nu}$ coordinate
system to a $\vectr{\xi}$ coordinate system are as follows: 


For a covariant vector (a rank (0,1) tensor)
\begin{equation}
  {\tilde{a}}_{i}=\delby{\nu^{a}}{\xi^{i}}a_{a}
\end{equation}

For a contravariant vector (a rank (1,0) tensor)
\begin{equation}
  {\tilde{a}}^{i}=\delby{\xi^{i}}{\nu^{a}}a^{a}
\end{equation}

For a covariant tensor (a rank (0,2) tensor)
\begin{equation}
  {\tilde{A}}_{ij}=\delby{\nu^{a}}{\xi^{i}}\delby{\nu^{b}}{\xi^{j}}A_{ab} 
\end{equation}

For a contravariant tensor (a rank (2,0) tensor)
\begin{equation}
  {\tilde{A}}^{ij}=\delby{\xi^{i}}{\nu^{a}}\delby{\xi^{j}}{\nu^{b}}A^{ab}
\end{equation}

and for Mixed tensors (rank (1,1) tensors)
\begin{equation}
  {\tilde{A}}^{i}_{.j}=\delby{\xi^{i}}{\nu^{a}}\delby{\nu^{b}}{\xi^{j}}A^{a}_{.b}
\end{equation}
and
\begin{equation}
  {\tilde{A}}_{i}^{.j}=\delby{\nu^{a}}{\xi^{i}}\delby{\xi^{j}}{\nu^{b}}A_{a}^{.b}
\end{equation}

\subsection{Derivatives}
\label{subsec:function derivatives}

\subsubsection{Scalars}

We note that a scalar quantity $\fnof{u}{\vectr{\xi}}$ has derivatives
\begin{equation}
  \delby{u}{\xi^{i}}=\partialderiv{u}{i}
\end{equation}

Or more formally, the covariant derivative ($\covarderiv{\cdot}{\cdot}$) of a
rank 0 tensor $u$ is
\begin{equation}
  \covarderiv{u}{i}=\delby{u}{\xi^{i}}=\partialderiv{u}{i}
\end{equation}

In more formal mathematical notation consider $f$ as a map between the
manifolds $\manifold{M}$ and $\manifold{N}$ \ie
\begin{equation}
  \mapping{f}{\manifold{M}}{\manifold{N}}
\end{equation}
then, at the point $u$, the derivative can be thought of as 
\begin{equation}
  \mapping{\derivativeop{u}{f}}{\tangentspace{M}{u}}{\tangentspace{N}{\fnof{f}{u}}}
\end{equation}
\ie the derivative operator takes a vector $v\in\tangentspace{M}{u}$ and maps
it to another vector $w\in\tangentspace{N}{\fnof{f}{u}}$.

Second derivatives. The second derivative or Hessian operator is defined as
\begin{equation}
  \mapping{\hessianop{u,v}{f}}{\tangentspace{\tangentspace{M}{u}}{v}}{\tangentspace{\tangentspace{N}{\fnof{f}{u}}}{\fnof{f^{'}}{v}}}
\end{equation}
\ie the hessian operator takes the first input as a vector
$v\in\tangentspace{M}{u}$ as the base direction you are moving in and a
second input as a vector $w\in\tangentspace{\tangentspace{M}}{}{v}$ as the direction
you are varying in.

The definition of the covariant Hessian is
\begin{equation}
  \hessianop{X,Y}{f} = \funccomposition{X}{\funccomposition{Y}{f}}-\funccomposition{\connection{X}{Y}}{f}
\end{equation}
or in component notation
\begin{equation}
  \hessianop{}{f}=\pbrac{\deltwoby{f}{x^{i}}{x^{j}}-\christoffel{k}{i}{j}\delby{f}{x^{k}}}\tensorprod{dx^{i}}{dx^{j}}
\end{equation}



\subsubsection{Vectors}

The derivatives of a vector $\vectr{v}$ are given by
\begin{equation}
  \begin{split}
    \delby{\vectr{v}}{\xi^{i}} &=
    \delby{}{\xi^{i}}\pbrac{v^{k}\generalbasevector_{k}} \\
    &= \delby{v^{k}}{\xi^{i}}\generalbasevector_{k}+v^{k}\delby{\generalbasevector_{k}}{\xi^{i}} \\
    &= \partialderiv{v^{k}}{i}\generalbasevector_{k}+v^{k}\partialderiv{\generalbasevector_{k}}{i}
  \end{split}
\end{equation}

Now introducing the notation
\begin{equation}
  \christoffelsecond{i}{j}{k} = \dotprod{\generalbasevector^{i}}{\delby{\generalbasevector_{j}}{x^{k}}}
\end{equation}
where $\christoffelsecond{i}{j}{k}$ are the Christoffel symbols of the second
kind. 

Note that the Christoffel symbols of the first kind are given by
\begin{equation}
  \christoffelfirst{i}{j}{k} = \dotprod{\generalbasevector_{i}}{\delby{\generalbasevector_{j}}{x^{k}}}
\end{equation}

Note that
\begin{equation}
  \begin{split}
    \christoffel{i}{j}{k} &= \dotprod{\generalbasevector^{i}}{\partialderiv{\generalbasevector_{j}}{k}} \\
    &=\dotprod{\generalbasevector^{i}}{\christoffelsecond{l}{j}{k}\generalbasevector_{l}} \\
    &= \christoffel{i}{j}{l}g^{i}_{.l} 
  \end{split}
\end{equation}

The Christoffel symbols of the first kind are also given by
\begin{equation}
  \christoffelfirst{i}{j}{k}=\frac{1}{2}\pbrac{\delby{g_{ij}}{\xi^{k}}+\delby{g_{ik}}{\xi^{j}}-\delby{g_{jk}}{\xi^{i}}}
\end{equation}
and that Christoffel symbols of the second kind are given by
\begin{equation}
  \begin{split}
    \christoffelsecond{i}{j}{k} &= g^{il}\christoffelfirst{l}{j}{k} \\
    &= \frac{1}{2}g^{il}\pbrac{\delby{g_{lj}}{\xi^{k}}+\delby{g_{lk}}{\xi^{j}}-\delby{g_{jk}}{\xi^{l}}} 
  \end{split}
\end{equation}

Note that Christoffel symbols are not tensors and the have the following
transformation laws from $\vectr{\nu}$ to $\vectr{\xi}$ coordinates
\begin{align}
  \christoffelfirst{i}{j}{k} &=
  \christoffelfirst{a}{b}{c}\delby{\nu^{b}}{\xi^{j}}\delby{\nu^{c}}{\xi^{k}}\delby{\nu^{a}}{\xi^{i}}+
  g_{ab}\delby{\nu^{c}}{\xi^{i}}\deltwoby{\nu^{c}}{\xi^{j}}{\xi^{k}} \\
  \christoffelsecond{i}{j}{k} &= \christoffelsecond{a}{b}{c}\delby{\xi^{i}}{\nu^{a}}\delby{\nu^{b}}{\xi^{k}}\delby{\nu^{c}}{\xi^{j}}+
  \delby{\xi^{i}}{\nu^{a}}\deltwoby{\nu^{a}}{\xi^{j}}{\xi^{k}} \\
\end{align}

We can now write (BELOW SEEMS WRONG - CHECK)
\begin{equation}
  \begin{split}
    \partialderiv{\vectr{v}}{i}&=\partialderiv{v^{k}}{i}\generalbasevector_{k}+\christoffel{k}{i}{j}v^{j}\generalbasevector_{j}\\
    &=\partialderiv{v^{k}}{i}\generalbasevector_{k}+\christoffel{j}{i}{k}v^{k}\generalbasevector_{k}\\
    &=\pbrac{\partialderiv{v^{k}}{i}+\christoffel{j}{i}{k}v^{k}}\generalbasevector_{k}\\
    &=\covarderiv{v^{k}}{i}\generalbasevector_{k}
  \end{split}
\end{equation}
where $\covarderiv{v^{k}}{i}$ is the covariant derivative of $v^{k}$ . 

The covariant derivative of a contravariant (rank (0,1)) tensor $v^{k}$ is
\begin{equation}
  \covarderiv{v^{k}}{i} =\partialderiv{v^{k}}{i}+\christoffel{k}{i}{j}v^{j}
\end{equation}
and the covariant derivative of a covariant tensor  (rank (1,0)) $v_{k}$ is
\begin{equation}
  \covarderiv{v_{k}}{i} =\partialderiv{v_{k}}{i}-\christoffel{j}{k}{i}v_{j}
\end{equation}

\subsubsection{Tensors}

The covariant derivative of a contravariant (rank (0,2)) tensor $W^{mn}$ is
\begin{equation}
  \covarderiv{W^{mn}}{i}=\partialderiv{W^{mn}}{i}+\christoffel{m}{j}{i}W^{jn}+\christoffel{n}{j}{i}W^{mj}
\end{equation}
and the covariant derivative of a covariant (rank (2,0)) tensor $W_{mn}$ is
\begin{equation}
  \covarderiv{W_{mn}}{i}=\partialderiv{W_{mn}}{i}-\christoffel{j}{m}{i}W_{jn}-\christoffel{j}{n}{i}W_{mj}
\end{equation}
and the covariant derivative of a mixed (rank (1,1)) tensor $W^{m}_{.n}$ is
\begin{equation}
  \covarderiv{W^{m}_{.n}}{i}=\partialderiv{W^{m}_{.n}}{i}+\christoffel{m}{j}{i}W^{j}_{.n}-\christoffel{j}{n}{i}W^{m}_{.j}
\end{equation}

\subsection{Common Operators}

For tensor equations to hold in any coordinate system the equations must
involve tensor quantities \ie covariant derivatives rather than partial derivatives.

\subsubsection{Gradient}

As the covariant derivative of a scalar is just the partial derivative the
gradient of a scalar function $\phi$ using covariant derivatives is
\begin{equation}
  \gradop \phi = \gradient{}{\phi}=\covarderiv{\phi}{i}\generalbasevector^{i}=\partialderiv{\phi}{i}\generalbasevector^{i}
\end{equation}
and
\begin{equation}
  \gradient{}{\phi}=\partialderiv{\phi}{i}\generalbasevector^{i}=\partialderiv{\phi}{i}g^{ij}\generalbasevector_{j}
\end{equation}

\subsubsection{Divergence}

The divergence of a vector using covariant derivatives is
\begin{equation}
  \divop \vectr{\phi} = \divergence{}{\vectr{\phi}}=\covarderiv{\phi^{i}}{i}=\frac{1}{\sqrt{\abs{g}}}\partialderiv{\pbrac{\sqrt{\abs{g}}\phi^{i}}}{i}
\end{equation}
where $g$ is the determinant of the covariant metric tensor $g_{ij}$.

\subsubsection{Curl}

The curl of a vector using covariant derivatives is
\begin{equation}
  \curlop \vectr{\phi} = \curl{}{\vectr{\phi}}=\frac{1}{\sqrt{g}}\pbrac{\covarderiv{\phi_{j}}{i}-\covarderiv{\phi_{i}}{j}}\generalbasevector_{k}
\end{equation}
where $g$ is the determinant of the covariant metric tensor $g_{ij}$.

\subsubsection{Laplacian}

The Laplacian of a scalar using covariant derivatives is
\begin{equation}
  \begin{aligned}
    \laplacian{}{\phi}&=\divop\pbrac{\gradop\phi}=\divergence{}{\gradient{}{\phi}}=\mixedderiv{\phi}{i}{i}=\frac{1}{\sqrt{g}}\partialderiv{\pbrac{\sqrt{g}g^{ij}\partialderiv{\phi}{j}}}{i}\\
    &=g^{ij}\pbrac{\deltwoby{\phi}{x^{i}}{x^{j}}-\christoffel{k}{i}{j}\delby{\phi}{x^{k}}}
  \end{aligned}
\end{equation}
where $g$ is the determinant of the covariant metric tensor $g_{ij}$ \ie $g=\determinant{\flattensor{\tensortwo{g}}}$.

The Laplacian of a vector using covariant derivatives is
\begin{equation}
  \laplacian{}{\vectr{\phi}}=\gradop\pbrac{\divop\vectr{\phi}}-\curlop \pbrac{\curlop\vectr{\phi}}=\mixedderiv{\vectr{\phi}}{i}{i}
\end{equation}

The Laplacian of a contravariant (rank (0,1)) tensor $\phi^{k}$ is
\begin{equation}
  \laplacian{}{\vectr{\phi}}=\pbrac{\laplacian{}{\phi_{k}}-2g^{ij}\christoffel{K}{j}{H}\delby{\phi^{h}}{x^{i}}+\phi^{h}\delby{g^{ij}\christoffel{K}{i}{j}}{x^{h}}}\vectr{e}^{k}
\end{equation}
and the covariant derivative of a covariant tensor  (rank (1,0)) $\phi_{k}$ is
\begin{equation}
  \laplacian{}{\vectr{\phi}}=\pbrac{\laplacian{}{\phi_{k}}-2g^{ij}\christoffel{h}{j}{k}\delby{\phi_{h}}{x^{i}}+\phi_{h}g^{ij}\delby{\christoffel{h}{i}{j}}{x^{i}}}\vectr{e}_{k}
\end{equation}

\section{Lie Derivatives}

\epstexfigure{DifferentialGeometry/svgs/lie_derivative.eps_tex}{Lie
  Derivative.}{Lie derivative.}{fig:LieDerivative}{0.5}

\section{Curves}

\epstexfigure{DifferentialGeometry/svgs/curve.eps_tex}{A curve in space.}{A
  parametised curve in space. The position, $\fnof{\vectr{r}}{\xi}$ is given in
  terms of the curve parameter $\xi$. $\vectr{t}$ is the tangent vector to the
  curve and $\vectr{n}$ is the normal vector to the curve.}{fig:curve}{1.0}

\section{Tensor Algebra}
\label{sec:TensorAlgebra}

\subsection{Tensor Symbols}
\label{subsec:TensorAlgebraSymbols}

\subsubsection{Kronecker Delta}
\label{subsubsec:TensorSymbolsKroneckerDelta}

The \emph{Kronecker delta}\footnote{Named
after\link{https://en.wikipedia.org/wiki/Leopold_Kronecker}{Leopold
  Kronecker} (1823-1891), a German Mathematician.}\index{Kronecker
  delta}\index{Special symbols!Kronecker delta} is a function of two
variables or indices. The Kronecker delta as the value $1$ if the two
indices are equal and $0$ if not \ie
\begin{equation}
  \flatkroneckertwo{i}{j}=\begin{cases}
  +1&\text{if $i=j$}, \\
  0 &\text{if $i\neq j$.}
  \end{cases}
  \label{eqn:DefinitionKroneckerDelta}
\end{equation}

In order to generalise, the Kronecker delta can be considered as representing a type $\pbrac{1,1}$ tensor with a covariant index $j$ and a contravariant index $i$ \ie
\begin{equation}
  \mixedkroneckertwo{i}{j}=\begin{cases}
  +1&\text{if $i=j$}, \\
  0 &\text{if $i\neq j$.}
  \end{cases}
  \label{eqn:DefinitionGeneralisedKroneckerDeltaTwo}
\end{equation}

The \emph{generalised Kronecker delta}\index{Kronecker delta!Generalised}\index{Special symbols!Generalised Kronecker delta} of order $2n$ is a type $\pbrac{n,n}$ tensor that is antisymmetric in both is upper and lower indices. It is defined by
\begin{equation}
  \mixedkronecker{i_{1},i_{2},\ldots,i_{n}}{j_{1},j_{2},\ldots,j_{n}}=\begin{cases}
  +1&\text{if $\pbrac{j_{1},j_{2},\ldots,j_{n}}$ are distinct integers and are an even permutation of $\pbrac{i_{1},i_{2},\ldots,i_{n}}$}, \\
  -1&\text{if $\pbrac{j_{1},j_{2},\ldots,j_{n}}$ are distinct integers and are an odd permutation of $\pbrac{i_{1},i_{2},\ldots,i_{n}}$}, \\
  0 &\text{otherwise.}
  \end{cases}
  \label{eqn:DefinitionGeneralisedKroneckerDelta}
\end{equation}

The generalised Kronecker delta of order $2n$ can also be defined in terms of the generalised Kronecker delta of order $1$ \ie
\begin{equation}
  \mixedkronecker{i_{1},i_{2},\ldots,i_{n}}{j_{1},j_{2},\ldots,j_{n}}=\determinant{\begin{bmatrix}
      \mixedkronecker{i_{1}}{j_{1}} & \mixedkronecker{i_{1}}{j_{2}} & \hdots & \mixedkronecker{i_{1}}{j_{n}} \\
      \mixedkronecker{i_{2}}{j_{1}} & \mixedkronecker{i_{2}}{j_{2}} & \hdots & \mixedkronecker{i_{2}}{j_{n}} \\
      \vdots & \vdots & \ddots & \vdots \\
      \mixedkronecker{i_{n}}{j_{1}} & \mixedkronecker{i_{n}}{j_{2}} & \hdots & \mixedkronecker{i_{n}}{j_{n}} \\      
  \end{bmatrix}}
  \label{eqn:DefinitionGeneralisedKroneckerDeltaDeterminant}
\end{equation}

\subsubsection{Levi-Civita Symbol}
\label{subsubsec:TensorSymbolsLeviCivita}

The \emph{Levi-Civita symbol}\footnote{Named after\link{https://en.wikipedia.org/wiki/Tullio_Levi-Civita}{Tullio
  Levi-Civita} (1873-1941), an Italian
Mathematician.}\index{Levi-Civita symbol}\index{Special
  symbols!Levi-Civita symbol} or \emph{permutation
symbol}\index{Permutation symbol}\index{Special symbols!Permutation
  symbol} is a symbol defined from the permutations of natural numbers.

Where the components of the Levi-Civita symbol are defined by
\begin{equation}
  \flatlevicivitathree{i}{j}{k}=\begin{cases}
  +1&\text{if $ijk$ is an even permutation \ie $\pbrac{1,2,3}$, $\pbrac{2,3,1}$, or $\pbrac{3,1,2}$}, \\
  -1&\text{if $ijk$ is an odd permutation \ie $\pbrac{1,3,2}$, $\pbrac{2,1,3}$, or $\pbrac{3,2,1}$}, \\
  0 &\text{otherwise \ie $i=j$, or $i=k$, or $j=k$.}
  \end{cases}
  \label{eqn:DefinitionLeviCivitaSymbolThirdOrder}
\end{equation}

The Levi-Civita symbol can also be defined for other orders. A second
order Levi-Civita symbol is defined by
\begin{equation}
  \flatlevicivitatwo{i}{j}=\begin{cases}
  +1&\text{if $ij$ is an even permutation \ie $\pbrac{1,2}$}, \\
  -1&\text{if $ij$ is an odd permutation \ie $\pbrac{2,1}$}, \\
  0 &\text{otherwise \ie $i=j$.}
  \end{cases}
  \label{eqn:DefinitionLeviCivitaSymbolSecondOrder}
\end{equation}
and a fourth order Levi-Civita symbol is defined by
\begin{equation}
  \flatlevicivitafour{i}{j}{k}{l}=\begin{cases}
  +1&\text{if $ijkl$ is an even permutation \ie $\pbrac{1,2,3,4}$, $\pbrac{2,3,4,1}$, $\pbrac{3,4,1,2}$, or $\pbrac{4,1,2,3}$}, \\
  -1&\text{if $ijkl$ is an odd permutation \ie $\pbrac{1,4,3,2}$, $\pbrac{2,1,4,3}$, $\pbrac{3,2,1,4}$, or $\pbrac{4,3,2,1}$}, \\
  0 &\text{otherwise \ie $i=j$, or $i=k$, or $i=l$, or $j=k$, or $j=l$, or $k=j$.}
  \end{cases}
  \label{eqn:DefinitionLeviCivitaSymbolFourthOrder}
\end{equation}

The definition can be extended to an arbitrary order. For example the
$\nth{n}$ order definition for $n$ dimensions with indices
$\pbrac{i_{1},i_{2},i_{3},\ldots,i_{n}}$ is given by
\begin{equation}
  \levicivitasymbol_{i_{1},i_{2},i_{3},\ldots,i_{n}}=\begin{cases}
  +1&\text{if $\pbrac{i_{1},i_{2},i_{3},\ldots,i_{n}}$ is an even permutation of $\pbrac{1,2,3,\ldots,n}$}, \\
  -1&\text{if $\pbrac{i_{1},i_{2},i_{3},\ldots,i_{n}}$ is an odd permutation of $\pbrac{1,2,3,\ldots,n}$}, \\
  0 &\text{otherwise.}
  \end{cases}
  \label{eqn:DefinitionLeviCivitaSymbolNOrder}
\end{equation}

Some properties of the Levi-Civita symbol include
\begin{align}
  \levicivitasymbol^{i_{1},i_{2},i_{3},\ldots,i_{n}}\levicivitasymbol_{i_{1},i_{2},i_{3},\ldots,i_{n}}&=\factorial{n} \label{eqn:LeviCivitaSymbolProperty1} \\
  \levicivitasymbol^{i_{1},i_{2},i_{3},\ldots,i_{n}}\levicivitasymbol_{j_{1},j_{2},j_{3},\ldots,j_{n}}&=\mixedkronecker{i_{1},i_{2},i_{3},\ldots,i_{n}}{j_{1},j_{2},j_{3},\ldots,j_{n}} \label{eqn:LeviCivitaSymbolProperty2} \\
  \levicivitasymbol^{i_{1},\ldots,i_{k},i_{k+1},\ldots,i_{n}}\levicivitasymbol_{i_{1},\ldots,i_{k},j_{k+1},\ldots,j_{n}}&=\mixedkronecker{i_{1},\ldots,i_{k},i_{k+1},\ldots,i_{n}}{i_{1},\ldots,i_{k},j_{k+1},\ldots,j_{n}}=\factorial{k}\mixedkronecker{j_{k+1},\ldots,j_{n}}{i_{k+1},\ldots,i_{n}}\label{eqn:LeviCivitaSymbolProperty3} \\
  \levicivitasymbol^{i_{1},i_{2},i_{3},\ldots,i_{n}}\levicivitasymbol_{j_{1},j_{2},j_{3},\ldots,j_{n}}&=\determinant{\begin{bmatrix}
      \mixedkronecker{i_{1}}{j_{1}} & \mixedkronecker{i_{1}}{j_{2}} & \hdots & \mixedkronecker{i_{1}}{j_{n}} \\
      \mixedkronecker{i_{2}}{j_{1}} & \mixedkronecker{i_{2}}{j_{2}} & \hdots & \mixedkronecker{i_{2}}{j_{n}} \\
      \vdots & \vdots & \ddots & \vdots \\
     \mixedkronecker{i_{n}}{j_{1}} & \mixedkronecker{i_{n}}{j_{2}} & \hdots & \mixedkronecker{i_{n}}{j_{n}}       
  \end{bmatrix}} \label{eqn:LeviCivitaSymbolProperty4}
\end{align}
where
$\mixedkronecker{i_{1},i_{2},i_{3},\ldots,i_{n}}{j_{1},j_{2},j_{3},\ldots,j_{n}}$
is the generalised Kronecker delta defined in
\eqnref{eqn:DefinitionGeneralisedKroneckerDelta}.

\subsection{First Order Tensors}
\label{subsec:TensorAlgebraFirstOrder}

\subsubsection{Identity Vector}
\label{subsubsec:IdentityTensorFirstOrder}

The first order \emph{identity vector}\index{Special tensors!First-order tensors!Identity vector}, $\identitytensorone$, is defined by
\begin{equation}
  \dotprod{\identitytensorone}{\vectr{a}}=\vectr{a}
  \label{eqn:DefinitionIdentityFirstOrder}
\end{equation}
for all vectors $\vectr{a}$.

\subsubsection{Null Vector}
\label{subsubsec:NullTensorFirstOrder}

The first order \emph{null vector}\index{Special tensors!First-order tensors!Null vector}, $\nulltensorone$, is defined by
\begin{equation}
  \dotprod{\nulltensorone}{\vectr{a}}=\nulltensorone
  \label{eqn:DefinitionNullFirstOrder}
\end{equation}
for all vectors $\vectr{a}$.

\subsubsection{Norm of a Vector}
\label{subsubsec:NormTensorFirstOrder}

The \emph{norm} of a first order tensor, or vectror, $\vectr{a}$, is denoted as
$\norm{\vectr{a}}$ defined as
\begin{equation}
  \norm{\vectr{a}}=\sqrt{\dotprod{\vectr{a}}{\vectr{a}}}
\end{equation}

\subsection{Second Order Tensors}
\label{sec:TensorAlgebraSecondOrder}

\subsubsection{Tensor or Outer Product}
\label{subsubsec:TensorProductSecondOrder}

Given two vectors, $\vectr{a}$ and $\vectr{b}$ we can form the \emph{tensor or outer
product} as
\begin{equation}
  \tensortwo{A}=\tensorprod{\vectr{a}}{\vectr{b}}=\vectr{a}\transpose{\vectr{b}}
\end{equation}
where $\tensortwo{A}$ is a second order tensor.

In component form we have
\begin{equation}
  \begin{aligned}
    \tensortwo{A}&=\tensorprod{a^{i}\generalbasevector_{i}}{b^{j}\generalbasevector_{j}}\\
    &=a^{i}b^{j}\tensorprod{\generalbasevector_{i}}{\generalbasevector_{j}}\\
    &=A^{ij}\tensorprod{\generalbasevector_{i}}{\generalbasevector_{j}}
  \end{aligned}
\end{equation}

\subsubsection{Identity Tensor}
\label{subsubsec:IdentityTensorSecondOrder}

The second order \emph{identity tensor}\index{Special tensors!Second-order tensors!Identity tensor}, $\identitytensortwo$, is defined by
\begin{equation}
  \dotprod{\identitytensortwo}{\vectr{a}}=\vectr{a}
  \label{eqn:DefinitionIdentitySecondOrder}
\end{equation}
for all vectors $\vectr{a}$. Now if a general second order tensor is defined
as
\begin{equation}
  \tensortwo{A}=A^{ij}\tensorprod{\generalbasevector_{i}}{\generalbasevector_{j}}=A_{ij}\tensorprod{\generalbasevector^{i}}{\generalbasevector^{j}}=A^{i}_{.j}\tensorprod{\generalbasevector_{i}}{\generalbasevector^{j}}=A^{.i}_{j}\tensorprod{\generalbasevector^{j}}{\generalbasevector_{i}}
\end{equation}
then the components of this tensor are given by
\begin{equation}
  \begin{aligned}
    A^{ij}&=\generalbasevector^{i}\tensortwo{A}\generalbasevector^{j} \\
    A_{ij}&=\generalbasevector_{i}\tensortwo{A}\generalbasevector_{j} \\
    A^{i}_{.j}&=\generalbasevector^{i}\tensortwo{A}\generalbasevector_{j} \\
    A^{.i}_{j}&=\generalbasevector_{j}\tensortwo{A}\generalbasevector^{i}    
  \end{aligned}
\end{equation}

From \eqnref{eqn:DefinitionIdentitySecondOrder}, the definition of the components of the identity tensor are thus given by
\begin{equation}
  \begin{aligned}
    I^{ij}&=\generalbasevector^{i}\tensortwo{I}\generalbasevector^{j}=\dotprod{\generalbasevector^{i}}{\generalbasevector^{j}}=g^{ij} \\
    I_{ij}&=\generalbasevector_{i}\tensortwo{I}\generalbasevector_{j}=\dotprod{\generalbasevector_{i}}{\generalbasevector_{j}}=g_{ij}  \\
    I^{i}_{.j}&=\generalbasevector^{i}\tensortwo{I}\generalbasevector_{j}=\dotprod{\generalbasevector^{i}}{\generalbasevector_{j}}=\mixedkronecker{i}{.j} \\
    I^{.i}_{j}&=\generalbasevector_{j}\tensortwo{I}\generalbasevector^{i}=\dotprod{\generalbasevector_{j}}{\generalbasevector^{i}}=\mixedkronecker{.i}{j}    
  \end{aligned}
\end{equation}

The second order \emph{identity tensor} is thus given by
\begin{equation}
  \begin{aligned}
    \identitytensortwo&=\tensorprod{\generalbasevector_{i}}{\generalbasevector^{i}}=\tensorprod{\generalbasevector^{i}}{\generalbasevector_{i}}\\
    &=\mixedkronecker{i}{.j}\tensorprod{\generalbasevector_{i}}{\generalbasevector^{j}}=\mixedkronecker{.i}{j}\tensorprod{\generalbasevector^{j}}{\generalbasevector_{i}}\\
    &=I^{i}_{.j}\tensorprod{\generalbasevector_{i}}{\generalbasevector^{j}}=I^{.i}_{j}\tensorprod{\generalbasevector^{j}}{\generalbasevector_{i}}
  \end{aligned}
\end{equation}

We also have
\begin{equation}
  \begin{aligned}
    \flattensor{\identitytensortwo}&=g_{ij}\tensorprod{\generalbasevector^{i}}{\generalbasevector^{j}} \\
    &=I_{ij}\tensorprod{\generalbasevector^{i}}{\generalbasevector^{j}} \\
    &=\flattensor{\tensortwo{g}}
  \end{aligned}
\end{equation}
and
\begin{equation}
  \begin{aligned}
    \sharptensor{\identitytensortwo}&=g^{ij}\tensorprod{\generalbasevector_{i}}{\generalbasevector_{j}}  \\
    &=I^{ij}\tensorprod{\generalbasevector_{i}}{\generalbasevector_{j}} \\
    &=\sharptensor{\tensortwo{g}}
  \end{aligned}
\end{equation}

\subsubsection{Null Tensor}
\label{subsubsec:NullTensorSecondOrder}

The second order \emph{null tensor}\index{Special tensors!Second-order tensors!Null tensor}, $\nulltensortwo$, is defined by
\begin{equation}
  \dotprod{\nulltensortwo}{\vectr{a}}=\nulltensorone
  \label{eqn:DefinitionNullSecondOrder}
\end{equation}
for all vectors $\vectr{a}$.

\subsubsection{Inverse Tensor}
\label{subsubsec:InverseTensorSecondOrder}

The second order \emph{inverse tensor} for a second order tensor
$\tensortwo{A}$ is defined such that
\begin{equation}
  \dotprod{\inverse{\tensortwo{A}}}{\tensortwo{A}}=\dotprod{\tensortwo{A}}{\inverse{\tensortwo{A}}}=\identitytensortwo
\end{equation}

The inverse of a tensor on exists if the tensor is not \emph{singular} that is
$\determinant{\tensortwo{A}}\neq 0$.

If $\phi$ is a scalar and $\tensortwo{A}$ and $\tensortwo{B}$ are second order tensors then some
properties of the inverse operator include
\begin{align}
  \inverse{\pbrac{\phi\tensortwo{A}}}&=\dfrac{\inverse{\tensortwo{A}}}{\phi}
  \label{eqn:ScaleInverseTensorTwo} \\
  \inverse{\pbrac{\inverse{\tensortwo{A}}}}&=\tensortwo{A}
  \label{eqn:InverseInverseTensorTwo} \\
  \inverse{\pbrac{\tensortwo{A}\tensortwo{B}}}&=\inverse{\tensortwo{B}}\inverse{\tensortwo{A}}
  \label{eqn:ProductInverseTensorTwo} \\
  \invtranspose{\tensortwo{A}}&=\transpose{\pbrac{\inverse{\tensortwo{A}}}}=
  \inverse{\pbrac{\transpose{\tensortwo{A}}}}
  \label{eqn:InverseTransposeInverseTensorTwo}
\end{align}

\subsubsection{Tensor Transpose}
\label{subsubsec:TensorTransposeSecondOrder}

A coordinate free definition of a tensor tranpose can be given as
\begin{equation}
  \dotprod{\vectr{a}}{\tensortwo{A}\vectr{b}}=\dotprod{\vectr{b}}{\transpose{\tensortwo{A}}\vectr{a}}
\end{equation}

For the second order tensor
$\tensortwo{A}=A^{ij}\tensorprodtwo{\generalbasevector_{i}}{\generalbasevector_{j}}$ then we
have
\begin{equation}
  \transpose{\tensortwo{A}}=A^{ji}\tensorprodtwo{\generalbasevector_{i}}{\generalbasevector_{j}}
\end{equation}

In component form the defintion of the transpose is
\begin{equation}
  \dotprod{\vectr{a}}{\tensortwo{A}\vectr{b}}=\dotprod{a^{i}\generalbasevector_{i}}{A_{ij}\tensorprod{\vectr{g^{i}}}{\vectr{g^{j}}}}b^{j}\generalbasevector_{j}=a^{i}A_{ij}b^{j}=b^{j}\pbrac{\transpose{A}}_{ji}a^{i}=\dotprod{b^{j}\generalbasevector_{j}}{\pbrac{\transpose{A}}_{ji}\tensorprod{\vectr{g^{j}}}{\vectr{g^{i}}}}a^{i}\generalbasevector_{i}=\dotprod{\vectr{b}}{\transpose{\tensortwo{A}}\vectr{a}}
\end{equation}
for $\sharptensor{\tensortwo{A}}$ and
\begin{equation}
  \dotprod{\vectr{a}}{\tensortwo{A}\vectr{b}}=\dotprod{a_{i}\generalbasevector^{i}}{A^{ij}\tensorprod{\vectr{g_{i}}}{\vectr{g_{j}}}}b_{j}\generalbasevector^{j}=a_{i}A^{ij}b_{j}=b_{j}\pbrac{\transpose{A}}^{ji}a_{i}=\dotprod{b_{j}\generalbasevector^{j}}{\pbrac{\transpose{A}}^{ji}\tensorprod{\vectr{g_{j}}}{\vectr{g_{i}}}}a_{i}\generalbasevector^{i}=\dotprod{\vectr{b}}{\transpose{\tensortwo{A}}\vectr{a}}
\end{equation}
for $\flattensor{\tensortwo{A}}$.

If $\tensortwo{A}=\transpose{\tensortwo{A}}$ then the tensor is said to be
symmetric. 

\subsubsection{Tensor contraction}
\label{subsubsec:TensorContractionSecondOrder}

\subsubsection{Trace of a Tensor}
\label{subsubsec:TraceTensorSecondOrder}

A scalar called the trace of a second order tensor, $\tensortwo{A}$, denoted as
$\trace{}{\tensor{A}}$, is given by 
\begin{equation}
  \trace{}{\tensortwo{A}}=\doubledotprod{\tensortwo{I}}{\tensortwo{A}}
\end{equation}

Or, in component form
\begin{equation}
  \begin{aligned}
    \trace{}{\tensortwo{A}}&=\doubledotprod{\flattensor{\tensortwo{I}}}{\sharptensor{\tensortwo{A}}}=\doubledotprod{\flattensor{\tensortwo{g}}}{\sharptensor{\tensortwo{A}}}=\doubledotprod{g_{ij}\tensorprod{\generalbasevector^{i}}{\generalbasevector^{j}}}{A^{kl}\tensorprod{\generalbasevector_{k}}{\generalbasevector_{l}}}=g_{ij}A^{ij}\\
    &=\doubledotprod{\sharptensor{\tensortwo{I}}}{\flattensor{\tensortwo{A}}}=\doubledotprod{\sharptensor{\tensortwo{g}}}{\flattensor{\tensortwo{A}}}=\doubledotprod{g^{ij}\tensorprod{\generalbasevector_{i}}{\generalbasevector_{j}}}{A_{kl}\tensorprod{\generalbasevector^{k}}{\generalbasevector^{l}}}=g^{ij}A_{ij}\\
    &=\doubledotprod{\tensortwo{I}}{\tensortwo{A}}=\doubledotprod{\mixedkronecker{i}{.j}\tensorprod{\generalbasevector_{i}}{\generalbasevector^{j}}}{A^{k}_{.l}\tensorprod{\generalbasevector_{k}}{\generalbasevector^{l}}}=\mixedkronecker{i}{.j}A^{j}_{.i}=A^{i}_{.i} \\
    &=\doubledotprod{\tensortwo{I}}{\tensortwo{A}}=\doubledotprod{\mixedkronecker{.i}{j}\tensorprod{\generalbasevector^{j}}{\generalbasevector_{i}}}{A^{.k}_{l}\tensorprod{\generalbasevector^{l}}{\generalbasevector_{k}}}=\mixedkronecker{.i}{j}A^{.j}_{i}=A^{.i}_{i}
  \end{aligned}
\end{equation}

To indicate what metric tensor the trace is repsect to the following notation
is used
\begin{equation}
  \begin{aligned}
    \trace{\tensortwo{g}}{\tensortwo{A}}&=\doubledotprod{\flattensor{\tensortwo{g}}}{\sharptensor{\tensortwo{A}}}=\doubledotprod{g_{ij}\tensorprod{\generalbasevector^{i}}{\generalbasevector^{j}}}{A^{kl}\tensorprod{\generalbasevector_{k}}{\generalbasevector_{l}}}=g_{ij}A^{ij}\\
    &=\doubledotprod{\sharptensor{\tensortwo{g}}}{\flattensor{\tensortwo{A}}}=\doubledotprod{g^{ij}\tensorprod{\generalbasevector_{i}}{\generalbasevector_{j}}}{A_{kl}\tensorprod{\generalbasevector^{k}}{\generalbasevector^{l}}}=g^{ij}A_{ij}\\
    &=\doubledotprod{\tensortwo{I}}{\tensortwo{A}}=\doubledotprod{\mixedkronecker{i}{.j}\tensorprod{\generalbasevector_{i}}{\generalbasevector^{j}}}{A^{k}_{.l}\tensorprod{\generalbasevector_{k}}{\generalbasevector^{l}}}=\mixedkronecker{i}{.j}A^{j}_{.i}=A^{i}_{.i} \\
    &=\doubledotprod{\tensortwo{I}}{\tensortwo{A}}=\doubledotprod{\mixedkronecker{.i}{j}\tensorprod{\generalbasevector^{j}}{\generalbasevector_{i}}}{A^{.k}_{l}\tensorprod{\generalbasevector^{l}}{\generalbasevector_{k}}}=\mixedkronecker{.i}{j}A^{.j}_{i}=A^{.i}_{i}
  \end{aligned}
\end{equation}

For tensors $\tensor{A}$ and $\tensor{B}$ and a scalar $\phi$, some properties of the trace operator include:
\begin{align}
  \trace{}{\pbrac{\tensor{A}+\tensor{B}}}&=\trace{}{\tensor{A}}+\trace{}{\tensor{B}}\\
  \trace{}{\pbrac{\phi\tensor{A}}}&=\phi\trace{}{\tensor{A}} \\
  \trace{}{\tensor{A}}&=\trace{}{\transpose{\tensor{A}}} \\
  \trace{}{\pbrac{\tensor{A}\tensor{B}}}&=\trace{}{\pbrac{\tensor{B}\tensor{A}}}\\
  \trace{}{\pbrac{\tensorprod{\tensor{A}}{\tensor{B}}}}&=\trace{}{\tensor{A}}\trace{}{\tensor{B}}\\
  \doubledotprod{\tensor{A}}{\tensor{B}}&=\trace{}{\pbrac{\transpose{\tensor{A}}\tensor{B}}}=\trace{}{\pbrac{\tensor{A}\transpose{\tensor{B}}}}=\trace{}{\pbrac{\transpose{\tensor{B}}\tensor{A}}}=\trace{}{\pbrac{\tensor{B}\transpose{\tensor{A}}}}
\end{align}


\subsubsection{Norm of a Tensor}
\label{subsubsec:NormTensorSecondOrder}

The \emph{norm} of a second order tensor, $\tensortwo{A}$, is denoted as
$\norm{\tensortwo{A}}$ defined as
\begin{equation}
  \norm{\tensortwo{A}}=\sqrt{\doubledotprod{\tensortwo{A}}{\tensortwo{A}}}
\label{eqn:DefinitionNormTensorSecondOrder}
\end{equation}

\subsubsection{Determinant of a Tensor}
\label{subsubsec:DeterminantTensorSecondOrder}

A scalar called the determinant of a second order tensor, $\tensortwo{A}$, denoted as
$\determinant{\tensor{A}}$, is defined as the determinant of the matrix of
tensor components \ie
\begin{equation}
  \begin{aligned}
    \determinant{\tensortwo{A}}&=\determinant{\begin{bmatrix}
        A_{11} & A_{12} & A_{13} \\
        A_{21} & A_{22} & A_{23} \\
        A_{31} & A_{32} & A_{33}
    \end{bmatrix}}\\
    &=A_{11}A_{22}A_{33}+A_{12}A_{23}A_{31}+A_{13}A_{21}A_{32}-A_{31}A_{22}A_{13}-A_{32}A_{23}A_{11}-A_{33}A_{21}A_{12}\\
    &=\sharplevicivitathree{i}{j}{k}A_{i1}A_{j2}A_{k3}\\
    &=\sharplevicivitathree{i}{j}{k}A_{1i}A_{2j}A_{3k}
  \end{aligned}
  \label{eqn:DefinitionDeterminantTensorSecondOrder}
\end{equation}

For a scalar $\phi$, vectors $\vectr{a}$ and $\vect{b}$ and second order
tensors $\tensortwo{A}$ and $\tensortwo{B}$, some properties of the determinant operator include:
\begin{align}
  \determinant{\tensorprod{\vectr{a}}{\vectr{b}}}&=0 \\
  \determinant{\phi\tensortwo{A}}&=\phi^{3}\determinant{\tensortwo{A}} \\
  \determinant{\tensortwo{A}\tensortwo{B}}&=\determinant{\tensortwo{A}}\determinant{\tensortwo{B}}\\
  \determinant{\transpose{\tensortwo{A}}}&=\determinant{\tensortwo{A}} \\
  \flatlevicivitathree{l}{m}{n}\determinant{\tensortwo{A}}&=\sharplevicivitathree{i}{j}{k}A_{il}A_{jm}A_{kn}
\end{align}

\subsubsection{Symmetric and Skew-symmetric}
\label{subsubsec:SymmetricSkew}

Consider a second order tensor, $\tensor{A}$. Any second order tensor can be
split into two parts \ie
\begin{align}
  \tensor{A}&=\frac{1}{2}\pbrac{\tensor{A}+\transpose{\tensor{A}}}+\frac{1}{2}\pbrac{\tensor{A}-\transpose{\tensor{A}}}
  \notag \\
  &=\symmetric{\tensor{A}}+\skewsym{\tensor{A}}
\end{align}
as first part of $\dfrac{1}{2}\pbrac{\tensor{A}+\transpose{\tensor{A}}}$ is a
strictly symmetric tensor and the second part of
$\dfrac{1}{2}\pbrac{\tensor{A}-\transpose{\tensor{A}}}$ is a strictly
skew-symmetric tensor. Here $\symmetric{}$ is the symmetric operation and $\skewsym{}$ the
skew-symmetric operator.

For second order tensors $\tensor{A}$ and $\tensor{B}$, some properties of a symmetric and skew-symmetric decomposition include:
\begin{align}
  \transpose{\pbrac{\symmetric{\tensor{A}}}}&=\symmetric{\tensor{A}} \\
  \transpose{\pbrac{\skewsym{\tensor{A}}}}&=-\skewsym{\tensor{A}} \\
  \determinant{\pbrac{\skewsym{\tensor{A}}}}&=0 \\
  \trace{}{\pbrac{\symmetric{\tensor{A}}\skewsym{\tensor{A}}}}&=0\\
  \doubledotprod{\symmetric{\tensor{A}}}{\skewsym{\tensor{A}}}&=0\\
  \doubledotprod{\tensor{A}}{\tensor{B}}&=\doubledotprod{\pbrac{\symmetric{\tensor{A}}+
      \skewsym{\tensor{A}}}}{\pbrac{\symmetric{\tensor{B}}+\skewsym{\tensor{B}}}} \\
  &=\doubledotprod{\symmetric{\tensor{A}}}{\symmetric{\tensor{B}}}+\doubledotprod{\skewsym{\tensor{A}}}{\skewsym{\tensor{B}}}
\end{align}
as the double contraction of a symmetric and skew-symmetric tensor is zero.

\subsubsection{Deviatoric and Spherical}
\label{subsubsec:DeviatoricSpherical}

Consider a second order tensor, $\tensor{A}$, on a manifold with a metric,
$\generalmetrictensor$. Any second order tensor can be split into two parts \ie
\begin{align}
  \tensor{A}&=\pbrac{\frac{1}{3}\trace{\generalmetrictensor}{\tensor{A}}}\tensor{I}+\pbrac{\tensor{A}-
    \pbrac{\frac{1}{3}\trace{\generalmetrictensor}{\tensor{A}}}\tensor{I}}
  \notag \\
  &=\spherical{\generalmetrictensor}{\tensor{A}}+\deviatoric{\generalmetrictensor}{\tensor{A}}
\end{align}

The spherical and deviatoric operators may also be written without the
optional metric \ie $\spherical{}{\tensor{A}}$ and $\deviatoric{}{\tensor{A}}$.

For tensors $\tensor{A}$ and $\tensor{B}$ and a scalar $c$, some properties of
the spherical and deviatoric operators include:
\begin{align}
  \determinant{\pbrac{\deviatoric{}{\tensor{A}}}}&=1\\
  \trace{}{\pbrac{\deviatoric{}{\tensor{A}}}}&=0\\
  \spherical{}{\pbrac{\deviatoric{}{\tensor{A}}}}&=0\\
  \doubledotprod{\deviatoric{}{\tensor{A}}}{\spherical{}{\tensor{B}}}&=0 \\
  \doubledotprod{\tensor{A}}{\tensor{B}}&=\doubledotprod{\pbrac{\spherical{}{\tensor{A}}+
      \deviatoric{}{\tensor{A}}}}{\pbrac{\spherical{}{\tensor{B}}+\deviatoric{}{\tensor{B}}}}
  \notag \\
  &=\doubledotprod{\spherical{}{\tensor{A}}}{\spherical{}{\tensor{B}}}+
  \doubledotprod{\deviatoric{}{\tensor{A}}}{\deviatoric{}{\tensor{B}}}
\end{align}
as the double contraction of a spherical and deviatoric tensor is zero.

\subsubsection{Special Orthogonal Tensors}
\label{subsubsec:SpecialOrthogonalTensorTwo}

A second order tensor, $\tensortwo{A}$ is a special orthogonal tensor\index{Tensors!Second-order tensors!Orthogonal tensor} if
\begin{equation}
  \transpose{\tensortwo{A}}\tensortwo{A}=\identitytensorfour
\end{equation}
or
\begin{equation}
  \inverse{\tensortwo{A}}=\transpose{\tensortwo{A}}
\end{equation}
which implies that $\determinant{\tensortwo{A}}=\pm 1$.

\subsubsection{Cayley-Hamilton Theorem}
\label{subsubsec:CayleyHamiltonTheorem}

Any second order tensor, $\tensor{A}$ will satisfy its own characteristic equation \ie
\begin{equation}
  \tensor{A}^{3}-\fnof{I_{1}}{\tensor{A}}\tensor{A}^{2}+\fnof{I_{2}}{\tensor{A}}-\fnof{I_{3}}{\tensor{A}}=0
  \label{eqn:CharacteristicEquationSecondOrder}
\end{equation}

\subsubsection{Invariants}
\label{subsubsec:InvariantsSecondOrder}

The relationships between invariants of a second order tensor, $\tensortwo{A}$,
and its inverse, $\inverse{\tensortwo{A}}$, are given by
\begin{equation}
  \fnof{I_1}{\inverse{\tensortwo{A}}}=\dfrac{\fnof{I_{2}}{\tensortwo{A}}}{\fnof{I_{3}}{\tensortwo{A}}}\quad\quad
  \fnof{I_2}{\inverse{\tensortwo{A}}}=\dfrac{\fnof{I_{1}}{\tensortwo{A}}}{\fnof{I_{3}}{\tensortwo{A}}}\quad\quad
  \fnof{I_3}{\inverse{\tensortwo{A}}}=\dfrac{1}{\fnof{I_{3}}{\tensortwo{A}}}
\end{equation}

\subsection{Third Order Tensors}
\label{sec:TensorAlgebraThirdOrder}

\subsubsection{Levi-Civita Tensor}
\label{subsubsec:Levi-CivitaTensorThirdOrder}

There is a special third-order pseudo-tensor called the
\emph{Levi-Civita tensor}\footnote{Named after
\link{https://en.wikipedia.org/wiki/Tullio_Levi-Civita}{Tullio
  Levi-Civita} (1873-1941), an Italian Mathematician.}\index{Special
  tensors!Third-order tensors!Levi-Civita tensor} or \emph{permutation
tensor}\index{Special tensors!Third-order tensors!Permutation tensor}
which is antisymmetric in all its indices.

As mentioned above the Levi-Civita tensor is an
\emph{psuedo-tensor}\index{Pseudo-tensor} rather than a normal
tensor. A pseduo-tensor is an object that transforms like a tensor
when undergoing a orientation preserving transformation but reverses
sign when underoing a transformation that does not preserve
orientation.

The Levi-Civita tensor, $\levicivitatensorthree$, is defined by
\begin{equation}
  \begin{aligned}
    \levicivitatensorthree&=\sharplevicivitathree{i}{j}{k}\tensorprodthree{\generalbasevector_{i}}{\generalbasevector_{j}}{\generalbasevector_{k}} \\
    &=\flatlevicivitathree{i}{j}{k}\tensorprodthree{\generalbasevector^{i}}{\generalbasevector^{j}}{\generalbasevector^{k}} \\
    &=\mixedlevicivitathree{i..}{.jk}\tensorprodthree{\generalbasevector_{i}}{\generalbasevector^{j}}{\generalbasevector^{k}} \\
    &=\mixedlevicivitathree{ij.}{..k}\tensorprodthree{\generalbasevector_{i}}{\generalbasevector_{j}}{\generalbasevector^{k}} \\
    &=\mixedlevicivitathree{i.k}{.j.}\tensorprodthree{\generalbasevector_{i}}{\generalbasevector^{j}}{\generalbasevector_{k}}    
  \end{aligned}
  \label{eqn:DefinitionLeviCivitaTensorThirdOrder}
\end{equation}
\etc

Where the components of the Levi-Civita tensor are defined by
\begin{equation}
  \flatlevicivitathree{i}{j}{k}=\begin{cases}
  +1&\text{if $ijk$ is an even permutation \ie $\pbrac{1,2,3}$, $\pbrac{2,3,1}$, or $\pbrac{3,1,2}$}, \\
  -1&\text{if $ijk$ is an odd permutation \ie $\pbrac{3,2,1}$, $\pbrac{1,3,2}$, or $\pbrac{2,3,1}$}, \\
  0 &\text{otherwise \ie $i=j$, or $j=k$, or $k=i$.}
  \end{cases}
  \label{eqn:DefinitionLeviCivitaTensorComponentsThirdOrder}
\end{equation}

Note that the components of the Levi-Civita tensor are constant with respect to covariant differentiation MOVE? DON'T KNOW ABOUT COVARIANT DERIVATIVES HERE? \ie
\begin{equation}
  \covarderiv{\flatlevicivitathree{i}{j}{k}}{m}=\covarderiv{\sharplevicivitathree{i}{j}{k}}{m}=0
\end{equation}
and similarily for mixed Levi-Civita components.

The Levi-Civita tensor is also defined for two indices \ie $\levicivitatensortwo$, is defined by
\begin{equation}
  \begin{aligned}
    \levicivitatensortwo&=\sharplevicivitatwo{i}{j}\tensorprodtwo{\generalbasevector_{i}}{\generalbasevector_{j}} \\
    &=\flatlevicivitatwo{i}{j}\tensorprodtwo{\generalbasevector^{i}}{\generalbasevector^{j}} \\
    &=\mixedlevicivitatwo{i.}{.j}\tensorprodtwo{\generalbasevector_{i}}{\generalbasevector^{j}} \\
    &=\mixedlevicivitatwo{.j}{i.}\tensorprodtwo{\generalbasevector_{i}}{\generalbasevector_{j}}    
  \end{aligned}
  \label{eqn:DefinitionLeviCivitaTensorTwoOrder}
\end{equation}
\etc


\subsection{Fourth Order Tensors}
\label{subsec:TensorAlgebraFourthOrder}

\subsubsection{Tensor Product}
\label{subsubsec:TensorProductFourthOrder}

The tensor product of two second order tensors, $\tensortwo{A}=A^{ij}\tensorprod{\generalbasevector_{i}}{\generalbasevector_{j}}$, and,
$\tensortwo{B}=B^{kl}\tensorprod{\generalbasevector_{k}}{\generalbasevector_{l}}$, will result in
\begin{equation}
  \begin{aligned}
    \tensorfour{C}&=\tensorprod{\tensortwo{A}}{\tensortwo{B}}=A^{ij}B^{kl}\tensorprodfour{\generalbasevector_{i}}{\generalbasevector_{j}}{\generalbasevector_{k}}{\generalbasevector_{l}}\\
    &=C^{ijkl}\tensorprodfour{\generalbasevector_{i}}{\generalbasevector_{j}}{\generalbasevector_{k}}{\generalbasevector_{l}}
  \end{aligned}
\end{equation}
where $\tensorfour{C}$ is a fourth order tensor.

Other tensor products can be defined
\citep{del_piero_properties_1979,curnier_conewise_1994,itskov_theory_2000,kintzel_fourth-order_2006}
such as the \emph{upper tensor product} \ie
\begin{equation}
  \begin{aligned}
    \tensorfour{C}&=\uppertensorprod{\tensortwo{A}}{\tensortwo{B}}=A^{ij}B^{kl}\tensorprodfour{\generalbasevector_{i}}{\generalbasevector_{k}}{\generalbasevector_{l}}{\generalbasevector_{j}}\\
    &=C^{iklj}\tensorprodfour{\generalbasevector_{i}}{\generalbasevector_{j}}{\generalbasevector_{k}}{\generalbasevector_{l}}
  \end{aligned}
\end{equation}
and the \emph{lower tensor product} \ie
\begin{equation}
  \begin{aligned}
    \tensorfour{C}&=\lowertensorprod{\tensortwo{A}}{\tensortwo{B}}=A^{ij}B^{kl}\tensorprodfour{\generalbasevector_{i}}{\generalbasevector_{k}}{\generalbasevector_{j}}{\generalbasevector_{l}}\\
    &=C^{ikjl}\tensorprodfour{\generalbasevector_{i}}{\generalbasevector_{j}}{\generalbasevector_{k}}{\generalbasevector_{l}}
  \end{aligned}
\end{equation}

A \emph{symmetric tensor product} can also be defined by combining the upper
and lower tensor products \ie
\begin{equation}
   \tensorfour{C}=\symtensorprod{\tensortwo{A}}{\tensortwo{B}}=\dfrac{1}{2}\pbrac{\lowertensorprod{\tensortwo{A}}{\tensortwo{B}}
      +\uppertensorprod{\tensortwo{B}}{\tensortwo{A}}}
\end{equation}

Similarily, an \emph{anti-symmetric tensor product} can also be defined by combining the upper
and lower tensor products \ie
\begin{equation}
   \tensorfour{C}=\antisymtensorprod{\tensortwo{A}}{\tensortwo{B}}=\dfrac{1}{2}\pbrac{\lowertensorprod{\tensortwo{A}}{\tensortwo{B}}
      -\uppertensorprod{\tensortwo{B}}{\tensortwo{A}}}
\end{equation}

Tensor products (where here $\circ = \tensorprodop,
\uppertensorprodop, \lowertensorprodop$) obey the \emph{associative law}
\begin{equation}
  \alpha\pbrac{\tensortwo{A}\circ\tensortwo{B}}=\pbrac{\alpha\tensortwo{A}}\circ\pbrac{\alpha\tensortwo{B}}
\end{equation}
and obey the \emph{distributive law}
\begin{equation}
  \tensortwo{A}\circ\pbrac{\tensortwo{B}+\tensortwo{C}}=\tensortwo{A}\circ\tensortwo{B}+\tensortwo{A}\circ\tensortwo{C}
\end{equation}
but, in general, they do not obey the \emph{commutative rule}
\begin{equation}
  \tensortwo{A}\circ\tensortwo{B}\neq\tensortwo{B}\circ\tensortwo{A}
\end{equation}

The lower tensor product $\lowertensorprodop$ is sometimes written $\boxtimes$
\citep{del_piero_properties_1979,kintzel_fourth-order_2006,kintzel_fourth-order2_2006,steinmann:2015}
or $\times$ \citep{itskov_theory_2000}, the upper tensor produce
$\uppertensorprodop$ is sometimes written $\times$
\citep{kintzel_fourth-order_2006,kintzel_fourth-order2_2006} or $\boxdot$
\citep{steinmann:2015}, and the symmetric tensor product $\symtensorprodop$ is
sometimes written $\, \overline{\underline{\otimes}} \,$ \citep{federico:2012}.

\subsubsection{Tensor Contraction}
\label{subsubsec:TensorContractionFourthOrder}

A number of additional double contractions can be defined. For example if
$\tensorfour{A}$ and $\tensorfour{B}$ are fourth order tensors then we can
define the standard \emph{double contraction} \ie
\begin{equation}
  \begin{aligned}
    \tensorfour{C}
    &=\doubledotprod{\tensorfour{A}}{\tensorfour{B}}\\
    &=\doubledotprod{\pbrac{A^{ijkl}\tensorprodfour{\generalbasevector_{i}}{\generalbasevector_{j}}{\generalbasevector_{k}}{\generalbasevector_{l}}}}{\pbrac{B_{mnrs}\tensorprodfour{\generalbasevector^{m}}{\generalbasevector^{n}}{\generalbasevector^{r}}{\generalbasevector^{s}}}}\\
    &=A^{ijkl}B_{klrs}\tensorprodfour{\generalbasevector_{i}}{\generalbasevector_{j}}{\generalbasevector^{r}}{\generalbasevector^{s}}\\
    &=C^{ij}_{rs}\tensorprodfour{\generalbasevector_{i}}{\generalbasevector_{j}}{\generalbasevector^{r}}{\generalbasevector^{s}}
  \end{aligned}
\end{equation}
  
We can also define an \emph{over double contraction} \ie
\begin{equation}
  \begin{aligned}
    \tensorfour{C}
    &=\upperdoubledotprod{\tensorfour{A}}{\tensorfour{B}}\\
    &=\upperdoubledotprod{\pbrac{A^{ijkl}\tensorprodfour{\generalbasevector_{i}}{\generalbasevector_{j}}{\generalbasevector_{k}}{\generalbasevector_{l}}}}{\pbrac{B_{mnrs}\tensorprodfour{\generalbasevector^{m}}{\generalbasevector^{n}}{\generalbasevector^{r}}{\generalbasevector^{s}}}}\\
    &=A^{ijkl}B_{mils}\tensorprodfour{\generalbasevector^{m}}{\generalbasevector_{j}}{\generalbasevector_{k}}{\generalbasevector^{s}}\\
    &=C^{jk}_{ms}\tensorprodfour{\generalbasevector^{m}}{\generalbasevector_{j}}{\generalbasevector_{k}}{\generalbasevector^{s}}
  \end{aligned}
\end{equation}
and an \emph{under double contraction} \ie
\begin{equation}
  \begin{aligned}
    \tensorfour{C}
    &=\lowerdoubledotprod{\tensorfour{A}}{\tensorfour{B}}\\
    &=\lowerdoubledotprod{\pbrac{A^{ijkl}\tensorprodfour{\generalbasevector_{i}}{\generalbasevector_{j}}{\generalbasevector_{k}}{\generalbasevector_{l}}}}{\pbrac{B_{mnrs}\tensorprodfour{\generalbasevector^{m}}{\generalbasevector^{n}}{\generalbasevector^{r}}{\generalbasevector^{s}}}}\\
    &=A^{ijkl}B_{jnrk}\tensorprodfour{\generalbasevector^{m}}{\generalbasevector_{j}}{\generalbasevector_{k}}{\generalbasevector^{s}}\\
    &=C^{il}_{nr}\tensorprodfour{\generalbasevector_{i}}{\generalbasevector^{n}}{\generalbasevector^{r}}{\generalbasevector_{k}}
  \end{aligned}
\end{equation}

The double contractions of a fourth order tensor $\tensorfour{A}$ and a second
order tensor $\tensortwo{B}$ results in a second order tensor
$\tensortwo{C}$. The different double contractions are given by
\begin{equation}
  \begin{aligned}
    \tensortwo{C}&=\doubledotprod{\tensorfour{A}}{\tensortwo{B}}\\
    &=\doubledotprod{\pbrac{A^{ijkl}\tensorprodfour{\generalbasevector_{i}}{\generalbasevector_{j}}{\generalbasevector_{k}}{\generalbasevector_{l}}}}{\pbrac{B_{mn}\tensorprod{\generalbasevector^{m}}{\generalbasevector^{n}}}}\\
    &=A^{ijkl}B_{kl}\tensorprod{\generalbasevector_{i}}{\generalbasevector_{j}}\\
    &=C^{ij}\tensorprod{\generalbasevector_{i}}{\generalbasevector_{j}}              
  \end{aligned}
\end{equation}
\begin{equation}
  \begin{aligned}
    \tensortwo{C}&=\upperdoubledotprod{\tensorfour{A}}{\tensortwo{B}}\\
    &=\upperdoubledotprod{\pbrac{A^{ijkl}\tensorprodfour{\generalbasevector_{i}}{\generalbasevector_{j}}{\generalbasevector_{k}}{\generalbasevector_{l}}}}{\pbrac{B_{mn}\tensorprod{\generalbasevector^{m}}{\generalbasevector^{n}}}}\\
    &=A^{ijkl}B_{il}\tensorprod{\generalbasevector_{j}}{\generalbasevector_{k}}\\
    &=C^{jk}\tensorprod{\generalbasevector_{j}}{\generalbasevector_{k}}              
  \end{aligned}
\end{equation}
\begin{equation}
  \begin{aligned}
    \tensortwo{C}&=\lowerdoubledotprod{\tensorfour{A}}{\tensortwo{B}}\\
    &=\lowerdoubledotprod{\pbrac{A^{ijkl}\tensorprodfour{\generalbasevector_{i}}{\generalbasevector_{j}}{\generalbasevector_{k}}{\generalbasevector_{l}}}}{\pbrac{B_{mn}\tensorprod{\generalbasevector^{m}}{\generalbasevector^{n}}}}\\
    &=A^{ijkl}B_{jk}\tensorprod{\generalbasevector_{i}}{\generalbasevector_{l}}\\
    &=C^{il}\tensorprod{\generalbasevector_{i}}{\generalbasevector_{l}}              
  \end{aligned}
\end{equation}
\begin{equation}
  \begin{aligned}
    \tensortwo{C}&=\doubledotprod{\tensortwo{B}}{\tensorfour{A}}\\
    &=\doubledotprod{\pbrac{B_{mn}\tensorprod{\generalbasevector^{m}}{g^{n}}}}{\pbrac{A^{ijkl}\tensorprodfour{\generalbasevector_{i}}{\generalbasevector_{j}}{\generalbasevector_{k}}{\generalbasevector_{l}}}}\\
    &=B_{ij}A^{ijkl}\tensorprod{\generalbasevector_{k}}{\generalbasevector_{l}}\\
    &=C^{kl}\tensorprod{\generalbasevector_{k}}{\generalbasevector_{l}}              
  \end{aligned}
\end{equation}
\begin{equation}
  \begin{aligned}
    \tensortwo{C}&=\upperdoubledotprod{\tensortwo{B}}{\tensorfour{A}}\\
    &=\upperdoubledotprod{\pbrac{B_{mn}\tensorprod{\generalbasevector^{m}}{\generalbasevector^{n}}}}{\pbrac{A^{ijkl}\tensorprodfour{\generalbasevector_{i}}{\generalbasevector_{j}}{\generalbasevector_{k}}{\generalbasevector_{l}}}}\\
    &=B_{jk}A^{ijkl}\tensorprod{\generalbasevector_{i}}{\generalbasevector_{l}}\\
    &=C^{il}\tensorprod{\generalbasevector_{i}}{\generalbasevector_{l}}\\
    &=\lowerdoubledotprod{\tensorfour{A}}{\tensortwo{B}}
  \end{aligned}
\end{equation}
\begin{equation}
  \begin{aligned}
    \tensortwo{C}&=\lowerdoubledotprod{\tensortwo{B}}{\tensorfour{A}}\\
    &=\lowerdoubledotprod{\pbrac{B_{mn}\tensorprod{\generalbasevector^{m}}{\generalbasevector^{n}}}}{\pbrac{A^{ijkl}\tensorprodfour{\generalbasevector_{i}}{\generalbasevector_{j}}{\generalbasevector_{k}}{\generalbasevector_{l}}}}\\
    &=B_{il}A^{ijkl}\tensorprod{\generalbasevector_{j}}{\generalbasevector_{k}}\\
    &=C^{jk}\tensorprod{\generalbasevector_{j}}{\generalbasevector_{k}} \\        
    &=\upperdoubledotprod{\tensorfour{A}}{\tensortwo{B}}
  \end{aligned}
\end{equation}

The double contraction of two second order tensors produce the same result \ie
\begin{equation}
  \doubledotprod{\tensortwo{A}}{\tensortwo{B}}=\upperdoubledotprod{\tensortwo{A}}{\tensortwo{B}}=\lowerdoubledotprod{\tensortwo{A}}{\tensortwo{B}}
\end{equation}

Tensor double contractions (where here $\circ = \doubledotprodop,
\upperdoubledotprodop, \lowerdoubledotprodop$) obey the \emph{associative law}
\begin{equation}
  \pbrac{\tensorfour{A}\circ\tensorfour{B}}\circ\tensorfour{C}=\tensorfour{A}\circ\pbrac{\tensorfour{B}\circ\tensorfour{C}}
\end{equation}
and obey the \emph{distributive law}
\begin{equation}
  \pbrac{\tensorfour{A}+\tensorfour{B}}\circ\tensorfour{C}=\tensorfour{A}\circ\tensorfour{C}+\tensorfour{B}\circ\tensorfour{C}
\end{equation}

For two fourth order tensors a double double or \emph{quadruple dot product}
contraction can be defined i.e., 
\begin{equation}
  \begin{aligned}
    c&=\quaddotprod{\tensorfour{A}}{\tensorfour{B}}\\
    &=\quaddotprod{\pbrac{A^{ijkl}\tensorprodfour{\generalbasevector_{i}}{\generalbasevector_{j}}{\generalbasevector_{k}}{\generalbasevector_{l}}}}{\pbrac{B_{mnrs}\tensorprodfour{\generalbasevector^{m}}{\generalbasevector^{n}}{\generalbasevector^{r}}{\generalbasevector^{s}}}}\\
    &=A^{ijkl}B_{ijkl}
  \end{aligned}
\end{equation}

Note that the \emph{trace} of a fourth order tensor can be defined as
\begin{equation}
  \trace{}{\tensorfour{A}}=\quaddotprod{\identitytensorfour}{\tensorfour{A}}=\quaddotprod{\tensorfour{A}}{\identitytensorfour}
\end{equation}
where $\identitytensorfour$ is the fourth order identity tensor defined below.

In \citep{kintzel_fourth-order_2006,kintzel_fourth-order2_2006}
$\upperdoubledotprodop$ is denoted $\circ\bullet$ and $\lowerdoubledotprodop$ is denoted $\bullet\circ$.

\subsubsection{Identity Tensor}
\label{subsubsec:IdentityTensorFourthOrder}

A number of \emph{fourth-order identity tensors}\index{Special tensors!Fourth-order tensors!Identity tensor} can be defined using a tensor
product of two second order identity tensors. We have
\begin{equation}
  \begin{aligned}
    \identitytensorfour&=\tensorprod{\identitytensortwo}{\identitytensortwo}\\
    &=\tensorprod{\pbrac{\mixedkronecker{i}{.j}\tensorprod{\generalbasevector_{i}}{\generalbasevector^{j}}}}{\pbrac{\mixedkronecker{k}{.l}\tensorprod{\generalbasevector_{k}}{\generalbasevector^{l}}}}\\
    &=\mixedkronecker{i}{.j}\mixedkronecker{k}{.l}\tensorprodfour{\generalbasevector_{i}}{\generalbasevector^{j}}{\generalbasevector_{k}}{\generalbasevector^{l}}\\
    &=I^{i.k.}_{.j.l}\tensorprodfour{\generalbasevector_{i}}{\generalbasevector^{j}}{\generalbasevector_{k}}{\generalbasevector^{l}}
  \end{aligned}
\end{equation}
\begin{equation}
  \begin{aligned}
    \upperidentitytensorfour&=\uppertensorprod{\identitytensortwo}{\identitytensortwo}\\
    &=\uppertensorprod{\pbrac{\mixedkronecker{i}{j}\tensorprod{\generalbasevector_{i}}{\generalbasevector^{j}}}}{\pbrac{\mixedkronecker{k}{l}\tensorprod{\generalbasevector_{k}}{\generalbasevector^{l}}}}\\
    &=\mixedkronecker{i}{.j}\mixedkronecker{k}{.l}\tensorprodfour{\generalbasevector_{i}}{\generalbasevector_{k}}{\generalbasevector^{l}}{\generalbasevector^{j}}\\
    &=\mixedkronecker{i}{.l}\mixedkronecker{.k}{j}\tensorprodfour{\generalbasevector_{i}}{\generalbasevector^{j}}{\generalbasevector_{k}}{\generalbasevector^{l}}\\
    &=I^{i.k.}_{.j.l}\tensorprodfour{\generalbasevector_{i}}{\generalbasevector^{j}}{\generalbasevector_{k}}{\generalbasevector^{l}}
  \end{aligned}
\end{equation}
\begin{equation}
  \begin{aligned}
    \loweridentitytensorfour&=\lowertensorprod{\identitytensortwo}{\identitytensortwo}\\
    &=\lowertensorprod{\pbrac{\mixedkronecker{i}{.j}\tensorprod{\generalbasevector_{i}}{\generalbasevector^{j}}}}{\pbrac{\mixedkronecker{k}{.l}\tensorprod{\generalbasevector_{k}}{\generalbasevector^{l}}}}\\
    &=\mixedkronecker{i}{.j}\mixedkronecker{k}{.l}\tensorprodfour{\generalbasevector_{i}}{\generalbasevector_{k}}{\generalbasevector^{j}}{\generalbasevector^{l}}\\
    &=\sharpkronecker{i}{k}\flatkronecker{j}{l}\tensorprodfour{\generalbasevector_{i}}{\generalbasevector^{j}}{\generalbasevector_{k}}{\generalbasevector^{l}}\\
    &=I^{i.k.}_{.j.l}\tensorprodfour{\generalbasevector_{i}}{\generalbasevector^{j}}{\generalbasevector_{k}}{\generalbasevector^{l}}
  \end{aligned}
\end{equation}
[FOR $\loweridentitytensorfour$ WHY IS THE MIXED KRONECKER SHARP (FLAT) WHEN WE HAVE
  NOT USED THE METRIC TO RAISE (LOWER) THE INDICES?]

We can also form the fully \emph{symmetric identity tensor},
$\symidentitytensorfour$, and the fully \emph{anti-symmetric identity tensor},
$\antisymidentitytensorfour$, can be found from the symmetric and
anti-symmetric tensor products \ie
\begin{equation}
  \begin{aligned}
    \symidentitytensorfour&=\symtensorprod{\tensortwo{I}}{\tensortwo{I}}=\dfrac{1}{2}\pbrac{\lowertensorprod{\tensortwo{I}}{\tensortwo{I}}+\uppertensorprod{\tensortwo{I}}{\tensortwo{I}}}=\dfrac{1}{2}\pbrac{\loweridentitytensorfour+\upperidentitytensorfour}\\
    &=\dfrac{1}{2}\pbrac{\lowertensorprod{\pbrac{\mixedkronecker{i}{.j}\tensorprod{\generalbasevector_{i}}{\generalbasevector^{j}}}}{\pbrac{\mixedkronecker{k}{.l}\tensorprod{\generalbasevector_{k}}{\generalbasevector^{l}}}}+\uppertensorprod{\pbrac{\mixedkronecker{i}{.j}\tensorprod{\generalbasevector_{i}}{\generalbasevector^{j}}}}{\pbrac{\mixedkronecker{k}{.l}\tensorprod{\generalbasevector_{k}}{\generalbasevector^{l}}}}}\\
    &=\dfrac{1}{2}\pbrac{\sharpkronecker{i}{k}\flatkronecker{j}{l}\tensorprodfour{\generalbasevector_{i}}{\generalbasevector^{j}}{\generalbasevector_{k}}{\generalbasevector^{l}}+\mixedkronecker{i}{.l}\mixedkronecker{.k}{j}\tensorprodfour{\generalbasevector_{i}}{\generalbasevector^{j}}{\generalbasevector_{k}}{\generalbasevector^{j}}}\\
    &=\dfrac{1}{2}\pbrac{\sharpkronecker{i}{k}\flatkronecker{j}{l}+\mixedkronecker{i}{.l}\mixedkronecker{.k}{j}}\tensorprodfour{\generalbasevector_{i}}{\generalbasevector^{j}}{\generalbasevector_{k}}{\generalbasevector^{l}}\\
    &=I^{i.k.}_{.j.l}\tensorprodfour{\generalbasevector_{i}}{\generalbasevector^{j}}{\generalbasevector_{k}}{\generalbasevector^{l}}
  \end{aligned}
\end{equation}
\begin{equation}
  \begin{aligned}
    \antisymidentitytensorfour&=\antisymtensorprod{\tensortwo{I}}{\tensortwo{I}}=\dfrac{1}{2}\pbrac{\lowertensorprod{\tensortwo{I}}{\tensortwo{I}}-\uppertensorprod{\tensortwo{I}}{\tensortwo{I}}}=\dfrac{1}{2}\pbrac{\loweridentitytensorfour-\upperidentitytensorfour}\\
    &=\dfrac{1}{2}\pbrac{\lowertensorprod{\pbrac{\mixedkronecker{i}{.j}\tensorprod{\generalbasevector_{i}}{\generalbasevector^{j}}}}{\pbrac{\mixedkronecker{k}{.l}\tensorprod{\generalbasevector_{k}}{\generalbasevector^{l}}}}-\uppertensorprod{\pbrac{\mixedkronecker{i}{.j}\tensorprod{\generalbasevector_{i}}{\generalbasevector^{i}}}}{\pbrac{\mixedkronecker{k}{.l}\tensorprod{\generalbasevector_{k}}{\generalbasevector^{l}}}}}\\
    &=\dfrac{1}{2}\pbrac{\sharpkronecker{i}{k}\flatkronecker{j}{l}\tensorprodfour{\generalbasevector_{i}}{\generalbasevector^{j}}{\generalbasevector_{k}}{\generalbasevector^{l}}-\mixedkronecker{i}{.l}\mixedkronecker{.k}{j}\tensorprodfour{\generalbasevector_{i}}{\generalbasevector^{j}}{\generalbasevector_{k}}{\generalbasevector^{l}}}\\
    &=\dfrac{1}{2}\pbrac{\sharpkronecker{i}{k}\flatkronecker{j}{l}-\mixedkronecker{i}{.l}\mixedkronecker{.k}{j}}\tensorprodfour{\generalbasevector_{i}}{\generalbasevector^{j}}{\generalbasevector_{k}}{\generalbasevector^{l}}\\
    &=I^{i.k.}_{.j.l}\tensorprodfour{\generalbasevector_{i}}{\generalbasevector^{j}}{\generalbasevector_{k}}{\generalbasevector^{l}}
  \end{aligned}
\end{equation}

We also the sharp fourth order identity tensors
\begin{equation}
  \begin{aligned}
    \sharptensor{\identitytensorfour}&=g^{ij}g^{kl}\tensorprodfour{\generalbasevector_{i}}{\generalbasevector_{j}}{\generalbasevector_{k}}{\generalbasevector_{l}}\\
    \sharptensor{\upperidentitytensorfour}&=g^{il}g^{jk}\tensorprodfour{\generalbasevector_{i}}{\generalbasevector_{j}}{\generalbasevector_{k}}{\generalbasevector_{l}}\\
    \sharptensor{\loweridentitytensorfour}&=g^{ik}g^{jl}\tensorprodfour{\generalbasevector_{i}}{\generalbasevector_{j}}{\generalbasevector_{k}}{\generalbasevector_{l}}\\
    \sharptensor{\symidentitytensorfour}&=\dfrac{1}{2}\pbrac{g^{ik}g^{jl}+g^{il}g^{jk}}\tensorprodfour{\generalbasevector_{i}}{\generalbasevector_{j}}{\generalbasevector_{k}}{\generalbasevector_{l}}\\
    \sharptensor{\antisymidentitytensorfour}&=\dfrac{1}{2}\pbrac{g^{ik}g^{jl}-g^{il}g^{jk}}\tensorprodfour{\generalbasevector_{i}}{\generalbasevector_{j}}{\generalbasevector_{k}}{\generalbasevector_{l}}
  \end{aligned}
\end{equation}
and the flat fourth order identity tensors
\begin{equation}
  \begin{aligned}
    \flattensor{\identitytensorfour}&=g_{ij}g_{kl}\tensorprodfour{\generalbasevector^{i}}{\generalbasevector^{j}}{\generalbasevector^{k}}{\generalbasevector^{l}}\\
    \flattensor{\upperidentitytensorfour}&=g_{il}g_{jk}\tensorprodfour{\generalbasevector^{i}}{\generalbasevector^{j}}{\generalbasevector^{k}}{\generalbasevector^{l}}\\
    \flattensor{\upperidentitytensorfour}&=g_{ik}g_{jl}\tensorprodfour{\generalbasevector^{i}}{\generalbasevector^{j}}{\generalbasevector^{k}}{\generalbasevector^{l}}\\
    \flattensor{\symidentitytensorfour}&=\dfrac{1}{2}\pbrac{g_{ik}g_{jl}+g_{il}g_{jk}}\tensorprodfour{\generalbasevector^{i}}{\generalbasevector^{j}}{\generalbasevector^{k}}{\generalbasevector^{l}}\\
    \flattensor{\antisymidentitytensorfour}&=\dfrac{1}{2}\pbrac{g_{ik}g_{jl}-g_{il}g_{jk}}\tensorprodfour{\generalbasevector^{i}}{\generalbasevector^{j}}{\generalbasevector^{k}}{\generalbasevector^{l}}
  \end{aligned}
\end{equation}

The components of the fully symmetric and fully anti-symmetric identity tensor are thus: TODO??? REFERENCE APPENDIX


The differences between the identity tensors can
be observed when they are contracted with a second order tensor,
$\tensortwo{A}$, \ie
\begin{align}
  \doubledotprod{\identitytensorfour}{\tensortwo{A}}&=\doubledotprod{\tensorprod{\identitytensortwo}{\identitytensortwo}}{\tensortwo{A}}=\pbrac{\trace{}{\tensortwo{A}}}\identitytensortwo \\
  \doubledotprod{\upperidentitytensorfour}{\tensortwo{A}}&=\doubledotprod{\uppertensorprod{\identitytensortwo}{\identitytensortwo}}{\tensortwo{A}}=\transpose{\tensortwo{A}} \\
  \doubledotprod{\loweridentitytensorfour}{\tensortwo{A}}&=\doubledotprod{\lowertensorprod{\identitytensortwo}{\identitytensortwo}}{\tensortwo{A}}=\tensortwo{A}\\
  \doubledotprod{\symidentitytensorfour}{\tensortwo{A}}&=\doubledotprod{\symtensorprod{\identitytensortwo}{\identitytensortwo}}{\tensortwo{A}}=\symmetric{\tensortwo{A}}\\
  \doubledotprod{\antisymidentitytensorfour}{\tensortwo{A}}&=\doubledotprod{\antisymtensorprod{\identitytensortwo}{\identitytensortwo}}{\tensortwo{A}}=\skewsym{\tensortwo{A}}
\end{align}
and
\begin{align}
  \upperdoubledotprod{\identitytensorfour}{\tensortwo{A}}&=\upperdoubledotprod{\tensorprod{\identitytensortwo}{\identitytensortwo}}{\tensortwo{A}}=\tensortwo{A} \\
  \upperdoubledotprod{\upperidentitytensorfour}{\tensortwo{A}}&=\upperdoubledotprod{\uppertensorprod{\identitytensortwo}{\identitytensortwo}}{\tensortwo{A}}=\pbrac{\trace{}{\tensortwo{A}}}\identitytensortwo \\
  \upperdoubledotprod{\loweridentitytensorfour}{\tensortwo{A}}&=\upperdoubledotprod{\lowertensorprod{\identitytensortwo}{\identitytensortwo}}{\tensortwo{A}}=\transpose{\tensortwo{A}}\\
  \lowerdoubledotprod{\identitytensorfour}{\tensortwo{A}}&=\lowerdoubledotprod{\tensorprod{\identitytensortwo}{\identitytensortwo}}{\tensortwo{A}}=\tensortwo{A} \\
  \lowerdoubledotprod{\upperidentitytensorfour}{\tensortwo{A}}&=\lowerdoubledotprod{\uppertensorprod{\identitytensortwo}{\identitytensortwo}}{\tensortwo{A}}=\pbrac{\trace{}{\tensortwo{A}}}\identitytensortwo \\
  \lowerdoubledotprod{\loweridentitytensorfour}{\tensortwo{A}}&=\lowerdoubledotprod{\lowertensorprod{\identitytensortwo}{\identitytensortwo}}{\tensortwo{A}}=\transpose{\tensortwo{A}}\\
\end{align}

In \citep{kintzel_fourth-order_2006,kintzel_fourth-order2_2006}
$\identitytensorfour$ is denoted $\tensorfour{I}$, $\upperidentitytensorfour$
is denoted $\tensorfour{I}^{L}$, and $\loweridentitytensorfour$ is denoted
$\tensorfour{I}^{R}$.

\subsubsection{Tensor Inverse}
\label{subsubsec:TensorInverseFourthOrder}

The inverse, $\inverse{\tensorfour{A}}$ of a fourth order tensor
$\tensorfour{A}$ is defined with respect to the double dot product such that
\begin{equation}
  \doubledotprod{\inverse{\tensorfour{A}}}{\tensorfour{A}}=\doubledotprod{\tensorfour{A}}{\inverse{\tensorfour{A}}}=\loweridentitytensorfour
\end{equation}

Note that for the upper and lower double dot products we have
\begin{equation}
  \upperdoubledotprod{\inverse{\tensorfour{A}}}{\tensorfour{A}}=\upperdoubledotprod{\tensorfour{A}}{\inverse{\tensorfour{A}}}=\lowerdoubledotprod{\inverse{\tensorfour{A}}}{\tensorfour{A}}=\lowerdoubledotprod{\tensorfour{A}}{\inverse{\tensorfour{A}}}=\identitytensorfour
\end{equation}

\subsubsection{Tensor Transpose}
\label{subsubsec:TensorTransposeFourthOrder}

The definition of a tensor tranpose for a fourth order tensor, $\tensorfour{C}$ is
\begin{equation}
  \doubledotprodthree{\tensortwo{A}}{\tensorfour{C}}{\tensortwo{B}}=\doubledotprodthree{\tensortwo{B}}{\transpose{\tensorfour{C}}}{\tensortwo{A}}
\end{equation}
where $\tensortwo{A}$ and $\tensortwo{B}$ are two second order tensors.

Depending on what indicies are transposed/swapped a number of different
transpose operations can occur with fourth order tensors \ie consider a fourth
order tensor
$\tensorfour{A}=A^{ijkl}\tensorprodfour{\generalbasevector_{i}}{\generalbasevector_{j}}{\generalbasevector_{k}}{\generalbasevector_{l}}$
then we have
\begin{align}
  \fulltranspose{\tensorfour{A}}&=A^{lkji}\tensorprodfour{\generalbasevector_{i}}{\generalbasevector_{j}}{\generalbasevector_{k}}{\generalbasevector_{l}}\\
  \leftrighttranspose{\tensorfour{A}}&=A^{jilk}\tensorprodfour{\generalbasevector_{i}}{\generalbasevector_{j}}{\generalbasevector_{k}}{\generalbasevector_{l}}\\
  \swaptranspose{\tensorfour{A}}&=A^{klij}\tensorprodfour{\generalbasevector_{i}}{\generalbasevector_{j}}{\generalbasevector_{k}}{\generalbasevector_{l}}\\
  \lefttranspose{\tensorfour{A}}&=A^{jikl}\tensorprodfour{\generalbasevector_{i}}{\generalbasevector_{j}}{\generalbasevector_{k}}{\generalbasevector_{l}}\\
  \righttranspose{\tensorfour{A}}&=A^{ijlk}\tensorprodfour{\generalbasevector_{i}}{\generalbasevector_{j}}{\generalbasevector_{k}}{\generalbasevector_{l}}\\
  \innertranspose{\tensorfour{A}}&=A^{ikjl}\tensorprodfour{\generalbasevector_{i}}{\generalbasevector_{j}}{\generalbasevector_{k}}{\generalbasevector_{l}}\\
  \outertranspose{\tensorfour{A}}&=A^{ljki}\tensorprodfour{\generalbasevector_{i}}{\generalbasevector_{j}}{\generalbasevector_{k}}{\generalbasevector_{l}}\\
  \xtranspose{\tensorfour{A}}&=A^{iljk}\tensorprodfour{\generalbasevector_{i}}{\generalbasevector_{j}}{\generalbasevector_{k}}{\generalbasevector_{l}}
\end{align}

Note that if $\tensorfour{A}=\swaptranspose{\tensorfour{A}}$ then
$\tensorfour{A}$ is said to have \emph{diagonal symmetry} or \emph{major
  symmetry}. If
$\tensorfour{A}=\lefttranspose{\tensorfour{A}}=\righttranspose{\tensorfour{A}}$
then $\tensorfour{A}$ is said to have \emph{pair symmetry} or \emph{minor
  symmetry}. If a tensor has both major and minor symmetry then the tensor is
said to have \emph{full symmetry} \ie
$\tensorfour{A}=\fulltranspose{\tensorfour{A}}$.

In \citet{kintzel_fourth-order_2006,kintzel_fourth-order2_2006}
$\fulltranspose{\tensorfour{A}}$ is denoted $\tensorfour{A}^{t}$,
$\leftrighttranspose{\tensorfour{A}}$ is denoted
$\tensorfour{A}^{T}=\tensorfour{A}^{d}$, $\swaptranspose{\tensorfour{A}}$ is
denoted $\tensorfour{A}^{D}$, $\lefttranspose{\tensorfour{A}}$ is denoted
$\tensorfour{A}^{dl}$, $\righttranspose{\tensorfour{A}}$ is denoted
$\tensorfour{A}^{dr}$, $\xtranspose{\tensorfour{A}}$ is denoted
$\tensorfour{A}^{L}$, $\innertranspose{\tensorfour{A}}$ is denoted
$\tensorfour{A}^{ti}$, and $\outertranspose{\tensorfour{A}}$ is denoted
$\tensorfour{A}^{to}$. In \citet{itskov_theory_2000}
$\leftrighttranspose{\tensorfour{A}}$ is denoted $\tensorfour{A}^{T}$,
$\innertranspose{\tensorfour{A}}$ is denoted $\tensorfour{A}^{t}$, and
$\xtranspose{\tensorfour{A}}$ is denoted $\tensorfour{A}^{R}$.

\subsubsection{Tensor Properties}
\label{subsubsec:TensorPropertiesFourthOrder}

From \citet{kintzel_fourth-order_2006} we can obtain a number of tensor
property relationships.

Single contraction of a fourth order tensor with a second order tensor
\begin{align}
  \dotprod{\pbrac{\tensorprod{\tensortwo{A}}{\tensortwo{B}}}}{\tensortwo{C}}&=\tensorprod{\tensortwo{A}}{\pbrac{\dotprod{\tensortwo{B}}{\tensortwo{C}}}}\\
  \dotprod{\pbrac{\uppertensorprod{\tensortwo{A}}{\tensortwo{B}}}}{\tensortwo{C}}&=\uppertensorprod{\pbrac{\dotprod{\tensortwo{A}}{\tensortwo{C}}}}{\tensortwo{B}}\\
  \dotprod{\pbrac{\lowertensorprod{\tensortwo{A}}{\tensortwo{B}}}}{\tensortwo{C}}&=\lowertensorprod{\tensortwo{A}}{\pbrac{\dotprod{\tensortwo{B}}{\tensortwo{C}}}}\\
  \dotprod{\tensortwo{C}}{\pbrac{\tensorprod{\tensortwo{A}}{\tensortwo{B}}}}&=\tensorprod{\pbrac{\dotprod{\tensortwo{C}}{\tensortwo{A}}}}{\tensortwo{B}}\\
  \dotprod{\tensortwo{C}}{\pbrac{\uppertensorprod{\tensortwo{A}}{\tensortwo{B}}}}&=\uppertensorprod{\pbrac{\dotprod{\tensortwo{C}}{\tensortwo{A}}}}{\tensortwo{B}}\\
  \dotprod{\tensortwo{C}}{\pbrac{\lowertensorprod{\tensortwo{A}}{\tensortwo{B}}}}&=\lowertensorprod{\pbrac{\dotprod{\tensortwo{C}}{\tensortwo{A}}}}{\tensortwo{B}}
\end{align}

Double contraction of a fourth order tensor with a second order tensor for a
double dot product
\begin{align}
  \doubledotprod{\pbrac{\tensorprod{\tensortwo{A}}{\tensortwo{B}}}}{\tensortwo{C}}&=\tensortwo{A}\pbrac{\doubledotprod{\tensortwo{B}}{\tensortwo{C}}}\\
  \doubledotprod{\pbrac{\uppertensorprod{\tensortwo{A}}{\tensortwo{B}}}}{\tensortwo{C}}&=\tensortwo{A}\transpose{\tensortwo{C}}\transpose{\tensortwo{B}}\\
  \doubledotprod{\pbrac{\lowertensorprod{\tensortwo{A}}{\tensortwo{B}}}}{\tensortwo{C}}&=\tensortwo{A}\tensortwo{C}\transpose{\tensortwo{B}}\\
  \doubledotprod{\tensortwo{C}}{\pbrac{\tensorprod{\tensortwo{A}}{\tensortwo{B}}}}&=\pbrac{\doubledotprod{\tensortwo{C}}{\tensortwo{A}}}\tensortwo{B}\\
  \doubledotprod{\tensortwo{C}}{\pbrac{\uppertensorprod{\tensortwo{A}}{\tensortwo{B}}}}&=\transpose{\tensortwo{B}}\transpose{\tensortwo{C}}\tensortwo{A}\\
  \doubledotprod{\tensortwo{C}}{\pbrac{\lowertensorprod{\tensortwo{A}}{\tensortwo{B}}}}&=\transpose{\tensortwo{A}}\tensortwo{C}\tensortwo{B}
\end{align}
and for an upper double dot product
\begin{align}
  \upperdoubledotprod{\pbrac{\tensorprod{\tensortwo{A}}{\tensortwo{B}}}}{\tensortwo{C}}&=\transpose{\tensortwo{A}}\tensortwo{C}\transpose{\tensortwo{B}}\\
  \upperdoubledotprod{\pbrac{\uppertensorprod{\tensortwo{A}}{\tensortwo{B}}}}{\tensortwo{C}}&=\pbrac{\doubledotprod{\tensortwo{A}}{\tensortwo{C}}}\tensortwo{B}\\
  \upperdoubledotprod{\pbrac{\lowertensorprod{\tensortwo{A}}{\tensortwo{B}}}}{\tensortwo{C}}&=\tensortwo{B}\transpose{\tensortwo{C}}\tensortwo{A}\\
  \upperdoubledotprod{\tensortwo{C}}{\pbrac{\tensorprod{\tensortwo{A}}{\tensortwo{B}}}}&=\tensortwo{A}\tensortwo{C}\tensortwo{B}\\
  \upperdoubledotprod{\tensortwo{C}}{\pbrac{\uppertensorprod{\tensortwo{A}}{\tensortwo{B}}}}&=\pbrac{\doubledotprod{\tensortwo{C}}{\tensortwo{B}}}\tensortwo{A}\\
  \upperdoubledotprod{\tensortwo{C}}{\pbrac{\lowertensorprod{\tensortwo{A}}{\tensortwo{B}}}}&=\tensortwo{A}\transpose{\tensortwo{C}}\tensortwo{B}
\end{align}
and for an lower double dot product
\begin{align}
  \lowerdoubledotprod{\pbrac{\tensorprod{\tensortwo{A}}{\tensortwo{B}}}}{\tensortwo{C}}&=\tensortwo{A}\tensortwo{C}\tensortwo{B}\\
  \lowerdoubledotprod{\pbrac{\uppertensorprod{\tensortwo{A}}{\tensortwo{B}}}}{\tensortwo{C}}&=\pbrac{\doubledotprod{\tensortwo{B}}{\tensortwo{C}}}\tensortwo{A}\\
  \lowerdoubledotprod{\pbrac{\lowertensorprod{\tensortwo{A}}{\tensortwo{B}}}}{\tensortwo{C}}&=\tensortwo{A}\transpose{\tensortwo{C}}\tensortwo{B}\\
  \lowerdoubledotprod{\tensortwo{C}}{\pbrac{\tensorprod{\tensortwo{A}}{\tensortwo{B}}}}&=\transpose{\tensortwo{A}}\tensortwo{C}\transpose{\tensortwo{B}}\\
  \lowerdoubledotprod{\tensortwo{C}}{\pbrac{\uppertensorprod{\tensortwo{A}}{\tensortwo{B}}}}&=\pbrac{\doubledotprod{\tensortwo{A}}{\tensortwo{C}}}\tensortwo{B}\\
  \lowerdoubledotprod{\tensortwo{C}}{\pbrac{\lowertensorprod{\tensortwo{A}}{\tensortwo{B}}}}&=\tensortwo{B}\transpose{\tensortwo{C}}\tensortwo{A}
\end{align}

Double contraction of two fourth order tensors for a double dot product
\begin{align}
  \doubledotprod{\pbrac{\tensorprod{\tensortwo{A}}{\tensortwo{B}}}}{\pbrac{\tensorprod{\tensortwo{C}}{\tensortwo{D}}}}&=\tensorprod{\pbrac{\doubledotprod{\tensortwo{B}}{\tensortwo{C}}}\tensortwo{A}}{\tensortwo{D}}\\
  \doubledotprod{\pbrac{\tensorprod{\tensortwo{A}}{\tensortwo{B}}}}{\pbrac{\uppertensorprod{\tensortwo{C}}{\tensortwo{D}}}}&=\tensorprod{\tensortwo{A}}{\pbrac{\transpose{\tensortwo{D}}\transpose{\tensortwo{B}}\tensortwo{C}}}\\
  \doubledotprod{\pbrac{\uppertensorprod{\tensortwo{A}}{\tensortwo{B}}}}{\pbrac{\tensorprod{\tensortwo{C}}{\tensortwo{D}}}}&=\tensorprod{\pbrac{\tensortwo{A}\transpose{\tensortwo{C}}\transpose{\tensortwo{B}}}}{\tensortwo{D}}\\
  \doubledotprod{\pbrac{\tensorprod{\tensortwo{A}}{\tensortwo{B}}}}{\pbrac{\lowertensorprod{\tensortwo{C}}{\tensortwo{D}}}}&=\tensorprod{\tensortwo{A}}{\pbrac{\transpose{\tensortwo{C}}\tensortwo{B}\tensortwo{D}}}\\
  \doubledotprod{\pbrac{\lowertensorprod{\tensortwo{A}}{\tensortwo{B}}}}{\pbrac{\tensorprod{\tensortwo{C}}{\tensortwo{D}}}}&=\tensorprod{\pbrac{\tensortwo{A}\tensortwo{C}\transpose{\tensortwo{B}}}}{\tensortwo{D}}\\
  \doubledotprod{\pbrac{\uppertensorprod{\tensortwo{A}}{\tensortwo{B}}}}{\pbrac{\uppertensorprod{\tensortwo{C}}{\tensortwo{D}}}}&=\lowertensorprod{\pbrac{\tensortwo{A}\tensortwo{D}}}{\pbrac{\tensortwo{B}\tensortwo{C}}}\\
  \doubledotprod{\pbrac{\uppertensorprod{\tensortwo{A}}{\tensortwo{B}}}}{\pbrac{\lowertensorprod{\tensortwo{C}}{\tensortwo{D}}}}&=\uppertensorprod{\pbrac{\tensortwo{A}\tensortwo{D}}}{\pbrac{\tensortwo{B}\tensortwo{C}}}\\
  \doubledotprod{\pbrac{\lowertensorprod{\tensortwo{A}}{\tensortwo{B}}}}{\pbrac{\uppertensorprod{\tensortwo{C}}{\tensortwo{D}}}}&=\uppertensorprod{\pbrac{\tensortwo{A}\tensortwo{C}}}{\pbrac{\tensortwo{B}\tensortwo{D}}}\\
  \doubledotprod{\pbrac{\lowertensorprod{\tensortwo{A}}{\tensortwo{B}}}}{\pbrac{\lowertensorprod{\tensortwo{C}}{\tensortwo{D}}}}&=\lowertensorprod{\pbrac{\tensortwo{A}\tensortwo{C}}}{\pbrac{\tensortwo{B}\tensortwo{D}}}
\end{align}
and for an upper double dot product
\begin{align}
  \upperdoubledotprod{\pbrac{\tensorprod{\tensortwo{A}}{\tensortwo{B}}}}{\pbrac{\tensorprod{\tensortwo{C}}{\tensortwo{D}}}}&=\tensorprod{\pbrac{\tensortwo{C}{\tensortwo{A}}}}{\pbrac{\tensortwo{B}{\tensortwo{D}}}}\\
  \upperdoubledotprod{\pbrac{\tensorprod{\tensortwo{A}}{\tensortwo{B}}}}{\pbrac{\uppertensorprod{\tensortwo{C}}{\tensortwo{D}}}}&=\uppertensorprod{\tensortwo{C}}{\pbrac{\transpose{\tensortwo{A}}\tensortwo{D}\transpose{\tensortwo{B}}}}\\
  \upperdoubledotprod{\pbrac{\uppertensorprod{\tensortwo{A}}{\tensortwo{B}}}}{\pbrac{\tensorprod{\tensortwo{C}}{\tensortwo{D}}}}&=\uppertensorprod{\pbrac{\tensortwo{C}\tensortwo{A}\tensortwo{D}}}{\tensortwo{B}}\\
  \upperdoubledotprod{\pbrac{\tensorprod{\tensortwo{A}}{\tensortwo{B}}}}{\pbrac{\lowertensorprod{\tensortwo{C}}{\tensortwo{D}}}}&=\lowertensorprod{\pbrac{\tensortwo{C}\transpose{\tensortwo{B}}}}{\pbrac{\transpose{\tensortwo{A}}\tensortwo{D}}}\\
  \upperdoubledotprod{\pbrac{\lowertensorprod{\tensortwo{A}}{\tensortwo{B}}}}{\pbrac{\tensorprod{\tensortwo{C}}{\tensortwo{D}}}}&=\lowertensorprod{\pbrac{\tensortwo{C}\tensortwo{A}}}{\pbrac{\tensortwo{B}\tensortwo{D}}}\\
  \upperdoubledotprod{\pbrac{\uppertensorprod{\tensortwo{A}}{\tensortwo{B}}}}{\pbrac{\uppertensorprod{\tensortwo{C}}{\tensortwo{D}}}}&=\uppertensorprod{\pbrac{\doubledotprod{\tensortwo{A}}{\tensortwo{D}}}\tensortwo{C}}{\tensortwo{B}}\\
  \upperdoubledotprod{\pbrac{\uppertensorprod{\tensortwo{A}}{\tensortwo{B}}}}{\pbrac{\lowertensorprod{\tensortwo{C}}{\tensortwo{D}}}}&=\uppertensorprod{\pbrac{\tensortwo{C}\transpose{\tensortwo{A}}\tensortwo{D}}}{\tensortwo{B}}\\
  \upperdoubledotprod{\pbrac{\lowertensorprod{\tensortwo{A}}{\tensortwo{B}}}}{\pbrac{\uppertensorprod{\tensortwo{C}}{\tensortwo{D}}}}&=\uppertensorprod{\tensortwo{C}}{\pbrac{\tensortwo{B}\transpose{\tensortwo{D}}\tensortwo{A}}}\\
  \upperdoubledotprod{\pbrac{\lowertensorprod{\tensortwo{A}}{\tensortwo{B}}}}{\pbrac{\lowertensorprod{\tensortwo{C}}{\tensortwo{D}}}}&=\tensorprod{\pbrac{\tensortwo{C}\transpose{\tensortwo{B}}}}{\pbrac{\transpose{\tensortwo{A}}\tensortwo{D}}}
\end{align}
and for an lower double dot product
\begin{align}
  \lowerdoubledotprod{\pbrac{\tensorprod{\tensortwo{A}}{\tensortwo{B}}}}{\pbrac{\tensorprod{\tensortwo{C}}{\tensortwo{D}}}}&=\tensorprod{\pbrac{\tensortwo{A}{\tensortwo{C}}}}{\pbrac{\tensortwo{D}{\tensortwo{B}}}}\\
  \lowerdoubledotprod{\pbrac{\tensorprod{\tensortwo{A}}{\tensortwo{B}}}}{\pbrac{\uppertensorprod{\tensortwo{C}}{\tensortwo{D}}}}&=\uppertensorprod{\pbrac{\tensortwo{A}\tensortwo{C}\tensortwo{B}}}{\tensortwo{D}}\\
  \lowerdoubledotprod{\pbrac{\uppertensorprod{\tensortwo{A}}{\tensortwo{B}}}}{\pbrac{\tensorprod{\tensortwo{C}}{\tensortwo{D}}}}&=\uppertensorprod{\tensortwo{A}}{\pbrac{\transpose{\tensortwo{C}}\tensortwo{B}\transpose{\tensortwo{D}}}}\\
  \lowerdoubledotprod{\pbrac{\tensorprod{\tensortwo{A}}{\tensortwo{B}}}}{\pbrac{\lowertensorprod{\tensortwo{C}}{\tensortwo{D}}}}&=\lowertensorprod{\pbrac{\tensortwo{A}\tensortwo{C}}}{\pbrac{\tensortwo{D}\tensortwo{B}}}\\
  \lowerdoubledotprod{\pbrac{\lowertensorprod{\tensortwo{A}}{\tensortwo{B}}}}{\pbrac{\tensorprod{\tensortwo{C}}{\tensortwo{D}}}}&=\lowertensorprod{\pbrac{\tensortwo{A}\transpose{\tensortwo{D}}}}{\pbrac{\transpose{\tensortwo{C}}\tensortwo{B}}}\\
  \lowerdoubledotprod{\pbrac{\uppertensorprod{\tensortwo{A}}{\tensortwo{B}}}}{\pbrac{\uppertensorprod{\tensortwo{C}}{\tensortwo{D}}}}&=\uppertensorprod{\pbrac{\doubledotprod{\tensortwo{B}}{\tensortwo{C}}}\tensortwo{A}}{\tensortwo{D}}\\
  \lowerdoubledotprod{\pbrac{\uppertensorprod{\tensortwo{A}}{\tensortwo{B}}}}{\pbrac{\lowertensorprod{\tensortwo{C}}{\tensortwo{D}}}}&=\uppertensorprod{\tensortwo{A}}{\pbrac{\tensortwo{D}\transpose{\tensortwo{B}}\tensortwo{C}}}\\
  \lowerdoubledotprod{\pbrac{\lowertensorprod{\tensortwo{A}}{\tensortwo{B}}}}{\pbrac{\uppertensorprod{\tensortwo{C}}{\tensortwo{D}}}}&=\uppertensorprod{\pbrac{\tensortwo{A}\transpose{\tensortwo{C}}\tensortwo{B}}}{\tensortwo{D}}\\
  \lowerdoubledotprod{\pbrac{\lowertensorprod{\tensortwo{A}}{\tensortwo{B}}}}{\pbrac{\lowertensorprod{\tensortwo{C}}{\tensortwo{D}}}}&=\tensorprod{\pbrac{\tensortwo{A}\transpose{\tensortwo{D}}}}{\pbrac{\transpose{\tensortwo{C}}\tensortwo{B}}}
\end{align}

Transposition of a fourth order tensor obtained from the standard tensor
product of two second order tensors gives
\begin{align}
  \fulltranspose{\pbrac{\tensorprod{\tensortwo{A}}{\tensortwo{B}}}}&=\tensorprod{\transpose{\tensortwo{B}}}{\transpose{\tensortwo{A}}}\\
  \leftrighttranspose{\pbrac{\tensorprod{\tensortwo{A}}{\tensortwo{B}}}}&=\tensorprod{\transpose{\tensortwo{A}}}{\transpose{\tensortwo{B}}}\\
  \swaptranspose{\pbrac{\tensorprod{\tensortwo{A}}{\tensortwo{B}}}}&=\tensorprod{\tensortwo{B}}{\tensortwo{A}}\\
  \innertranspose{\pbrac{\tensorprod{\tensortwo{A}}{\tensortwo{B}}}}&=\lowertensorprod{\tensortwo{A}}{\tensortwo{B}}\\
  \outertranspose{\pbrac{\tensorprod{\tensortwo{A}}{\tensortwo{B}}}}&=\lowertensorprod{\transpose{\tensortwo{B}}}{\transpose{\tensortwo{A}}}\\
  \lefttranspose{\pbrac{\tensorprod{\tensortwo{A}}{\tensortwo{B}}}}&=\tensorprod{\transpose{\tensortwo{A}}}{\tensortwo{B}}\\
  \righttranspose{\pbrac{\tensorprod{\tensortwo{A}}{\tensortwo{B}}}}&=\tensorprod{\tensortwo{A}}{\transpose{\tensortwo{B}}}
\end{align}
and for an upper tensor product
\begin{align}
  \fulltranspose{\pbrac{\uppertensorprod{\tensortwo{A}}{\tensortwo{B}}}}&=\uppertensorprod{\transpose{\tensortwo{A}}}{\transpose{\tensortwo{B}}}\\
  \leftrighttranspose{\pbrac{\uppertensorprod{\tensortwo{A}}{\tensortwo{B}}}}&=\uppertensorprod{\tensortwo{B}}{\tensortwo{A}}\\
  \swaptranspose{\pbrac{\uppertensorprod{\tensortwo{A}}{\tensortwo{B}}}}&=\uppertensorprod{\transpose{\tensortwo{B}}}{\transpose{\tensortwo{A}}}\\
  \innertranspose{\pbrac{\uppertensorprod{\tensortwo{A}}{\tensortwo{B}}}}&=\uppertensorprod{\tensortwo{A}}{\transpose{\tensortwo{B}}}\\
  \outertranspose{\pbrac{\uppertensorprod{\tensortwo{A}}{\tensortwo{B}}}}&=\uppertensorprod{\transpose{\tensortwo{A}}}{\tensortwo{B}}\\
  \lefttranspose{\pbrac{\uppertensorprod{\tensortwo{A}}{\tensortwo{B}}}}&=\lowertensorprod{\tensortwo{B}}{\tensortwo{A}}\\
  \righttranspose{\pbrac{\uppertensorprod{\tensortwo{A}}{\tensortwo{B}}}}&=\lowertensorprod{\tensortwo{A}}{\tensortwo{B}}
\end{align}
and for an lower tensor product
\begin{align}
  \fulltranspose{\pbrac{\lowertensorprod{\tensortwo{A}}{\tensortwo{B}}}}&=\lowertensorprod{\transpose{\tensortwo{B}}}{\transpose{\tensortwo{A}}}\\
  \leftrighttranspose{\pbrac{\lowertensorprod{\tensortwo{A}}{\tensortwo{B}}}}&=\lowertensorprod{\tensortwo{B}}{\tensortwo{A}}\\
  \swaptranspose{\pbrac{\lowertensorprod{\tensortwo{A}}{\tensortwo{B}}}}&=\lowertensorprod{\transpose{\tensortwo{A}}}{\transpose{\tensortwo{B}}}\\
  \innertranspose{\pbrac{\lowertensorprod{\tensortwo{A}}{\tensortwo{B}}}}&=\tensorprod{\tensortwo{A}}{\tensortwo{B}}\\
  \outertranspose{\pbrac{\lowertensorprod{\tensortwo{A}}{\tensortwo{B}}}}&=\tensorprod{\transpose{\tensortwo{B}}}{\transpose{\tensortwo{A}}}\\
  \lefttranspose{\pbrac{\lowertensorprod{\tensortwo{A}}{\tensortwo{B}}}}&=\uppertensorprod{\tensortwo{B}}{\tensortwo{A}}\\
  \righttranspose{\pbrac{\lowertensorprod{\tensortwo{A}}{\tensortwo{B}}}}&=\uppertensorprod{\tensortwo{A}}{\tensortwo{B}}
\end{align}

Transposition of the contraction of two fourth order tensors for the double
dot product gives
\begin{align}
  \fulltranspose{\pbrac{\doubledotprod{\tensorfour{A}}{\tensorfour{B}}}}&=\doubledotprod{\leftrighttranspose{\swaptranspose{\tensorfour{B}}}}{\leftrighttranspose{\swaptranspose{\tensorfour{A}}}}\\
  \leftrighttranspose{\pbrac{\doubledotprod{\tensorfour{A}}{\tensorfour{B}}}}&=\doubledotprod{\leftrighttranspose{\tensorfour{A}}}{\leftrighttranspose{\tensorfour{B}}}\\
  \swaptranspose{\pbrac{\doubledotprod{\tensorfour{A}}{\tensorfour{B}}}}&=\doubledotprod{\swaptranspose{\tensorfour{B}}}{\swaptranspose{\tensorfour{A}}}\\
  \lefttranspose{\pbrac{\doubledotprod{\tensorfour{A}}{\tensorfour{B}}}}&=\doubledotprod{\lefttranspose{\tensorfour{A}}}{\tensorfour{B}}\\
  \righttranspose{\pbrac{\doubledotprod{\tensorfour{A}}{\tensorfour{B}}}}&=\doubledotprod{\tensorfour{A}}{\righttranspose{\tensorfour{B}}}
\end{align}
and for an upper double dot product
\begin{align}
  \fulltranspose{\pbrac{\upperdoubledotprod{\tensorfour{A}}{\tensorfour{B}}}}&=\upperdoubledotprod{\fulltranspose{\tensorfour{A}}}{\fulltranspose{\tensorfour{B}}}\\
  \leftrighttranspose{\pbrac{\upperdoubledotprod{\tensorfour{A}}{\tensorfour{B}}}}&=\upperdoubledotprod{\leftrighttranspose{\tensorfour{B}}}{\leftrighttranspose{\tensorfour{A}}}\\
  \swaptranspose{\pbrac{\upperdoubledotprod{\tensorfour{A}}{\tensorfour{B}}}}&=\upperdoubledotprod{\leftrighttranspose{\outertranspose{\tensorfour{B}}}}{\leftrighttranspose{\innertranspose{\tensorfour{A}}}}\\
  \innertranspose{\pbrac{\upperdoubledotprod{\tensorfour{A}}{\tensorfour{B}}}}&=\upperdoubledotprod{\innertranspose{\tensorfour{A}}}{\tensorfour{B}}\\
  \outertranspose{\pbrac{\upperdoubledotprod{\tensorfour{A}}{\tensorfour{B}}}}&=\upperdoubledotprod{\tensorfour{A}}{\outertranspose{\tensorfour{B}}}
\end{align}
and for an lower double dot product
\begin{align}
  \fulltranspose{\pbrac{\lowerdoubledotprod{\tensorfour{A}}{\tensorfour{B}}}}&=\lowerdoubledotprod{\fulltranspose{\tensorfour{A}}}{\fulltranspose{\tensorfour{B}}}\\
  \leftrighttranspose{\pbrac{\lowerdoubledotprod{\tensorfour{A}}{\tensorfour{B}}}}&=\lowerdoubledotprod{\leftrighttranspose{\tensorfour{B}}}{\leftrighttranspose{\tensorfour{A}}}\\
  \swaptranspose{\pbrac{\lowerdoubledotprod{\tensorfour{A}}{\tensorfour{B}}}}&=\lowerdoubledotprod{\leftrighttranspose{\innertranspose{\tensorfour{B}}}}{\leftrighttranspose{\outertranspose{\tensorfour{A}}}}\\
  \innertranspose{\pbrac{\lowerdoubledotprod{\tensorfour{A}}{\tensorfour{B}}}}&=\lowerdoubledotprod{\tensorfour{A}}{\innertranspose{\tensorfour{B}}}\\
  \outertranspose{\pbrac{\lowerdoubledotprod{\tensorfour{A}}{\tensorfour{B}}}}&=\lowerdoubledotprod{\outertranspose{\tensorfour{A}}}{\tensorfour{B}}
\end{align}


\section{Tensor Calculus}
\label{sec:TensorCalculus}

\subsection{Directional derivative}
\label{subsec:DirectionalDerivativeTensorCalculus}

The directional derivative of a scalar function of a vector,
$\fnof{f}{\vectr{x}}$, with respect to $\vectr{x}$ along the vector direction
$\vectr{u}$ is defined by the limit
\begin{equation}
  \begin{split}
    \directionalderiv{\vectr{x}}{\fnof{f}{\vectr{x}}}{\vectr{u}}
    &=\genlimit{\epsilon}{0}\dfrac{\fnof{f}{\vectr{x}+\epsilon\vectr{u}}-\fnof{f}{\vectr{x}}}{\epsilon}\\
    &=\evalat{\dfrac{d}{d\epsilon}\fnof{f}{\vectr{x}+\epsilon\vectr{u}}}{\epsilon=0}\\
    &=\dotprod{\gradient{\vectr{x}}{\fnof{f}{\vectr{x}}}}{\vectr{u}} \\
    &=\dotprod{\delby{\fnof{f}{\vectr{x}}}{\vectr{x}}}{\vectr{u}}
  \end{split} 
\end{equation} 
 
Some properties of directional derivatives of a scalar function of a vector include
\begin{align}
  \directionalderiv{\vectr{x}}{\fnof{f_{1}}{\vectr{x}}+\fnof{f_{2}}{\vectr{x}}}{\vectr{u}}
  &=\directionalderiv{\vectr{x}}{\fnof{f_{1}}{\vectr{x}}}{\vectr{u}}
  +\directionalderiv{\vectr{x}}{\fnof{f_{2}}{\vectr{x}}}{\vectr{u}}
  \nonumber \\
  &=\dotprod{\pbrac{\delby{\fnof{f_{1}}{\vectr{x}}}{\vectr{x}}
      +\delby{\fnof{f_{2}}{\vectr{x}}}{\vectr{x}}}}{\vectr{u}}
  \label{eqn:AdditionRuleDirectDerivScalarValueVectorFunction} \\
  \directionalderiv{\vectr{x}}{\fnof{f_{1}}{\vectr{x}}\fnof{f_{2}}{\vectr{x}}}{\vectr{u}}
  &=\directionalderiv{\vectr{x}}{\fnof{f_{1}}{\vectr{x}}}{\vectr{u}}\fnof{f_{2}}{\vectr{x}}
  +\fnof{f_{1}}{\vectr{x}}\directionalderiv{\vectr{x}}{\fnof{f_{2}}{\vectr{x}}}{\vectr{u}}
  \nonumber \\
  &=\pbrac{\dotprod{\delby{\fnof{f_{1}}{\vectr{x}}}{\vectr{x}}}{\vectr{u}}}\fnof{f_{2}}{\vectr{x}}
  +\fnof{f_{1}}{\vectr{x}}\pbrac{\dotprod{\delby{\fnof{f_{2}}{\vectr{x}}}{\vectr{x}}}{\vectr{u}}}
  \label{eqn:ProductRuleDirectDerivScalarValueVectorFunction}\\
  \directionalderiv{\vectr{x}}{\fnof{f_{1}}{\fnof{f_{2}}{\vectr{x}}}}{\vectr{u}}
  &=\directionalderiv{f_{2}}{\fnof{f_{1}}{\fnof{f_{2}}{\vectr{x}}}}{\directionalderiv{\vectr{x}}{\fnof{f_{2}}{\vectr{x}}}{\vectr{u}}}
  \nonumber \\
  &=\delby{\fnof{f_{1}}{\fnof{f_{2}}{\vectr{x}}}}{\fnof{f_{2}}{\vectr{x}}}\pbrac{\dotprod{\delby{\fnof{f_{2}}{\vectr{x}}}{\vectr{x}}}{\vectr{u}}}
  \label{eqn:ChainRuleDirectDerivScalarValueVectorFunction}
\end{align}

Note that the normal derivative of a function is just the directional
derivative in the normal direction \ie
\begin{equation}
  \delby{\fnof{f}{\vectr{x}}}{\vectr{n}}=\dotprod{\gradient{}{\fnof{f}{\vectr{x}}}}{\vectr{n}}=\directionalderiv{\vectr{x}}{\fnof{f}{\vectr{x}}}{\vectr{n}}
\end{equation}

For a vector valued function of a vector, $\fnof{\vectr{f}}{\vectr{x}}$, the
directional derivative in the direction of a vector $\vectr{u}$ is
\begin{equation}
  \begin{split}
    \directionalderiv{\vectr{x}}{\fnof{\vectr{f}}{\vectr{x}}}{\vectr{u}}
    &=\evalat{\dfrac{d}{d\epsilon}\fnof{\vectr{f}}{\vectr{x}+\epsilon\vectr{u}}}{\epsilon=0}\\
    &=\dotprod{\gradient{\vectr{x}}{\fnof{\vectr{f}}{\vectr{x}}}}{\vectr{u}}\\
    &=\dotprod{\delby{\fnof{\vectr{f}}{\vectr{x}}}{\vectr{x}}}{\vectr{u}}\\
  \end{split}
\end{equation}

Some properties of directional derivatives of a vector function of a vector include
\begin{align}
  \directionalderiv{\vectr{x}}{\fnof{\vectr{f}_{1}}{\vectr{x}}
    +\fnof{\vectr{f}_{2}}{\vectr{x}}}{\vectr{u}}
  &=\directionalderiv{\vectr{x}}{\fnof{\vectr{f}_{1}}{\vectr{x}}}{\vectr{u}}
  +\directionalderiv{\vectr{x}}{\fnof{\vectr{f}_{2}}{\vectr{x}}}{\vectr{u}}
  \nonumber \\
  &=\dotprod{\pbrac{\delby{\fnof{\vectr{f}_{1}}{\vectr{x}}}{\vectr{x}}
      +\delby{\fnof{\vectr{f}_{2}}{\vectr{x}}}{\vectr{x}}}}{\vectr{u}}
  \label{eqn:AdditionRuleDirectDerivVectorValueVectorFunction} \\
  \directionalderiv{\vectr{x}}{\crossprod{\fnof{\vectr{f}_{1}}{\vectr{x}}}{\fnof{\vectr{f}_{2}}{\vectr{x}}}}{\vectr{u}}
  &=\crossprod{\directionalderiv{\vectr{x}}{\fnof{\vectr{f}_{1}}{\vectr{x}}}{\vectr{u}}}{\fnof{\vectr{f}_{2}}{\vectr{x}}}
  +\crossprod{\fnof{\vectr{f}_{1}}{\vectr{x}}}{\directionalderiv{\vectr{x}}{\fnof{\vectr{f}_{2}}{\vectr{x}}}{\vectr{u}}}
  \nonumber \\
  &=\crossprod{\pbrac{\dotprod{\delby{\fnof{\vectr{f}_{1}}{\vectr{x}}}{\vectr{x}}}{\vectr{u}}}}{\fnof{\vectr{f}_{2}}{\vectr{x}}}
  +\crossprod{\fnof{\vectr{f}_{1}}{\vectr{x}}}{\pbrac{\dotprod{\delby{\fnof{\vectr{f}_{2}}{\vectr{x}}}{\vectr{x}}}{\vectr{u}}}}
  \label{eqn:ProductRuleDirectDerivVectorValueVectorFunction} \\
  \directionalderiv{\vectr{x}}{\fnof{\vectr{f}_{1}}{\fnof{\vectr{f}_{2}}{\vectr{x}}}}{\vectr{u}}
  &=\directionalderiv{\vectr{f}_{2}}{\fnof{\vectr{f}_{1}}{\fnof{\vectr{f}_{2}}{\vectr{x}}}}{\directionalderiv{\vectr{x}}{\fnof{\vectr{f}_{2}}{\vectr{x}}}{\vectr{u}}}
  \nonumber \\
  &=\dotprod{\delby{\fnof{\vectr{f}_{1}}{\fnof{\vectr{f}_{2}}{\vectr{x}}}}{\fnof{\vectr{f}_{2}}{\vectr{x}}}}{\pbrac{\dotprod{\delby{\vectr{f}_{2}}{\vectr{x}}}{\vectr{u}}}}
  \label{eqn:ChainRuleDirectDerivVectorValueVectorFunction} 
\end{align}

For a scalar valued function of a second order tensor, $\fnof{f}{\tensor{A}}$,
the directional derivative in the the direction of a second order tensor
$\tensor{U}$ is
\begin{equation}
  \begin{split}
    \directionalderiv{\tensor{A}}{\fnof{f}{\tensor{A}}}{\vectr{U}}
    &=\evalat{\dfrac{d}{d\epsilon}\fnof{f}{\tensor{A}+\epsilon\tensor{U}}}{\epsilon=0}\\
    &=\doubledotprod{\gradient{\tensor{A}}{\fnof{f}{\tensor{A}}}}{\tensor{U}}\\
    &=\doubledotprod{\delby{\fnof{f}{\tensor{A}}}{\tensor{A}}}{\tensor{U}}\\
  \end{split}  
\end{equation}

Some properties of directional derivatives of a scalar function of a second
order tensor include
\begin{align}
  \directionalderiv{\tensor{A}}{\fnof{f_{1}}{\tensor{A}}+\fnof{f_{2}}{\tensor{A}}}{\tensor{U}}
  &=\directionalderiv{\tensor{A}}{\fnof{f_{1}}{\tensor{A}}}{\tensor{U}}
  +\directionalderiv{\tensor{A}}{\fnof{f_{2}}{\tensor{A}}}{\tensor{U}}
  \nonumber \\
  &=\doubledotprod{\pbrac{\delby{\fnof{f_{1}}{\tensor{A}}}{\tensor{A}}
      +\delby{\fnof{f_{2}}{\tensor{A}}}{\tensor{A}}}}{\tensor{U}}
  \label{eqn:AdditionRuleDirectDerivScalarValueTensorFunction} \\
  \directionalderiv{\tensor{A}}{\fnof{f_{1}}{\tensor{A}}\fnof{f_{2}}{\tensor{A}}}{\tensor{U}}
  &=\directionalderiv{\vectr{x}}{\fnof{f_{1}}{\vectr{x}}}{\vectr{u}}\fnof{f_{2}}{\vectr{x}}
  +\fnof{f_{1}}{\vectr{x}}\directionalderiv{\vectr{x}}{\fnof{f_{2}}{\vectr{x}}}{\vectr{u}}
  \nonumber \\
  &=\pbrac{\doubledotprod{\delby{\fnof{f_{1}}{\tensor{A}}}{\tensor{A}}}{\tensor{U}}}\fnof{f_{2}}{\tensor{A}}
  +\fnof{f_{1}}{\tensor{A}}\pbrac{\doubledotprod{\delby{\fnof{f_{2}}{\tensor{A}}}{\tensor{A}}}{\tensor{U}}}
  \label{eqn:ProductRuleDirectDerivScalarValueTensorFunction} \\
  \directionalderiv{\tensor{A}}{\fnof{f_{1}}{\fnof{f_{2}}{\tensor{A}}}}{\tensor{U}}
  &=\directionalderiv{f_{2}}{\fnof{f_{1}}{\fnof{f_{2}}{\tensor{A}}}}{\directionalderiv{\tensor{A}}{\fnof{f_{2}}{\tensor{A}}}{\tensor{U}}}
  \nonumber \\
  &=\delby{\fnof{f_{1}}{\fnof{f_{2}}{\tensor{A}}}}{\fnof{f_{2}}{\tensor{A}}}\pbrac{\doubledotprod{\delby{\fnof{f_{2}}{\tensor{A}}}{\tensor{A}}}{\tensor{U}}}
    \label{eqn:ChainRuleDirectDerivScalarValueTensorFunction}
\end{align}

For a second order tensor valued function of a second order tensor, $\fnof{F}{\tensor{A}}$,
the directional derivative in the the direction of a second order tensor
$\tensor{U}$ is
\begin{equation}
  \begin{split}
    \directionalderiv{\tensor{A}}{\fnof{\tensor{F}}{\tensor{A}}}{\vectr{U}}
    &=\evalat{\dfrac{d}{d\epsilon}\fnof{\tensor{F}}{\tensor{A}+\epsilon\tensor{U}}}{\epsilon=0}\\
    &=\doubledotprod{\gradient{\tensor{A}}{\fnof{\tensor{F}}{\tensor{A}}}}{\tensor{U}}\\
    &=\doubledotprod{\delby{\fnof{\tensor{F}}{\tensor{A}}}{\tensor{A}}}{\tensor{U}}\\
  \end{split}  
\end{equation}

Some properties of directional derivatives of a second order tensor function
of a second order tensor include
\begin{align}
  \directionalderiv{\tensor{A}}{\fnof{\tensor{F}_{1}}{\tensor{A}}
    +\fnof{\tensor{F}_{2}}{\tensor{A}}}{\tensor{U}}
  &=\directionalderiv{\tensor{A}}{\fnof{\tensor{F}_{1}}{\tensor{A}}}{\tensor{U}}
  +\directionalderiv{\tensor{A}}{\fnof{\tensor{F}_{2}}{\tensor{A}}}{\tensor{U}}
  \nonumber \\
  &=\doubledotprod{\pbrac{\delby{\fnof{\tensor{F}_{1}}{\tensor{A}}}{\tensor{A}}
      +\delby{\fnof{\tensor{F}_{2}}{\tensor{A}}}{\tensor{A}}}}{\tensor{U}}
  \label{eqn:AdditionRuleDirectDerivTensorValueTensorFunction}\\
  \directionalderiv{\tensor{A}}{\dotprod{\fnof{\tensor{F}_{1}}{\tensor{A}}}{\fnof{\tensor{F}_{2}}{\tensor{A}}}}{\tensor{A}}
  &=\dotprod{\directionalderiv{\tensor{A}}{\fnof{\tensor{F}_{1}}{\tensor{A}}}{\tensor{U}}}{\fnof{\tensor{F}_{2}}{\tensor{A}}}
  +\dotprod{\fnof{\tensor{F}_{1}}{\tensor{A}}}{\directionalderiv{\tensor{A}}{\fnof{\tensor{F}_{2}}{\tensor{A}}}{\tensor{U}}}
   \nonumber \\
  &=\dotprod{\pbrac{\doubledotprod{\delby{\fnof{\tensor{F}_{1}}{\tensor{A}}}{\tensor{A}}}{\tensor{U}}}}{\fnof{\tensor{F}_{2}}{\tensor{A}}}
  +\dotprod{\fnof{\tensor{F}_{1}}{\tensor{A}}}{\pbrac{\doubledotprod{\delby{\fnof{\tensor{F}_{2}}{\tensor{A}}}{\tensor{A}}}{\tensor{U}}}}
  \label{eqn:ProductRuleDirectDerivTensorValueTensorFunction}\\
  \directionalderiv{\tensor{A}}{\fnof{\tensor{F}_{1}}{\fnof{\tensor{F}_{2}}{\tensor{A}}}}{\tensor{U}}
  &=\directionalderiv{\tensor{F}_{2}}{\fnof{\tensor{F}_{1}}{\fnof{\tensor{F}_{2}}{\tensor{A}}}}{\directionalderiv{\tensor{A}}{\fnof{\tensor{F}_{2}}{\tensor{A}}}{\tensor{U}}}
   \nonumber \\ 
  &=\doubledotprod{\delby{\fnof{\tensor{F}_{1}}{\fnof{\tensor{F}_{2}}{\tensor{A}}}}{\fnof{\tensor{F}_{2}}{\tensor{A}}}}{\pbrac{\doubledotprod{\delby{\tensor{F}_{2}}{\tensor{A}}}{\tensor{U}}}}
  \label{eqn:ChainRuleDirectDerivTensorValueTensorFunction}
\end{align}

\subsection{Second Order Tensors}
\label{subsec:TensorCalculusSecondOrder}

\subsubsection{Derivatives of the Identity Tensor}
\label{subsubsec:IdentityDerivativeSecondOrder}

The derivative of a second order identity tensor is fourth order null tensor \ie
\begin{equation}
  \delby{\identitytensortwo}{\tensortwo{A}}=\nulltensorfour
  \label{eqn:DerivIdentityTensorSecondOrder}
\end{equation}
where $\tensortwo{A}$ is a second order tensor.

This also leads to the relationship
\begin{equation}
  \doubledotprod{\delby{\identitytensortwo}{\tensortwo{A}}}{\tensortwo{B}}=\doubledotprod{\nulltensorfour}{\tensortwo{B}}=0
\end{equation}
for a second order tensor $\tensortwo{B}$.

\subsubsection{Derivatives of a Tensor with Respect to Itself}
\label{subsubsec:SelfDerivativeSecondOrder}

If $\tensortwo{A}$ and $\tensortwo{B}$ are second order tensors then
\begin{equation}
  \doubledotprod{\delby{\tensortwo{A}}{\tensortwo{A}}}{\tensortwo{B}}=\evalat{\dby{}{\epsilon}\pbrac{\tensortwo{A}+\epsilon\tensortwo{B}}}{\epsilon=0}=\tensortwo{B}=\doubledotprod{\loweridentitytensorfour}{\tensortwo{B}}
\end{equation}
and thus
\begin{equation}
  \delby{\tensortwo{A}}{\tensortwo{A}}=\loweridentitytensorfour
  \label{eqn:DerivSelfTensorSecondOrder}
\end{equation}

Note that we also have
If $\tensortwo{A}$ and $\tensortwo{B}$ are second order tensors then
\begin{equation}
  \doubledotprod{\delby{\transpose{\tensortwo{A}}}{\tensortwo{A}}}{\tensortwo{B}}=\transpose{\tensortwo{B}}=\doubledotprod{\upperidentitytensorfour}{\tensortwo{B}}
\end{equation}
and thus
\begin{equation}
  \delby{\transpose{\tensortwo{A}}}{\tensortwo{A}}=\upperidentitytensorfour
  \label{eqn:DerivSelfTransposeTensorSecondOrder}
\end{equation}

This implies that if $\tensortwo{A}$ is a symmetric tensor \ie
$\tensortwo{A}=\transpose{\tensortwo{A}}$ then we have
\begin{equation}
  \delby{\tensortwo{A}}{\tensortwo{A}}=\dfrac{1}{2}\pbrac{\loweridentitytensorfour+\upperidentitytensorfour}=\symidentitytensorfour
  \label{eqn:DerivSelfSymmetricTensorSecondOrder}
\end{equation}


\subsubsection{Derivatives of the Determinant}
\label{subsubsec:DeterminantDerivativeSecondOrder}

Consider the directional derivative of the determinant of a tensor,
$\tensor{A}$ in the direction of another tensor, $\tensor{U}$
\begin{equation}
  \begin{split}
    \directionalderiv{}{\determinant{\tensor{A}}}{\tensor{U}}
    &=\evalat{\dby{}{\epsilon}\pbrac{\determinant{\tensor{A}+\epsilon\tensor{U}}}}{\epsilon=0}\\
    &=\evalat{\dby{}{\epsilon}\pbrac{\determinant{\tensor{A}\pbrac{\tensor{I}
          +\epsilon\inverse{\tensor{A}}\tensor{U}}}}}{\epsilon=0}\\
    &=\determinant{\tensor{A}}\evalat{\dby{}{\epsilon}\pbrac{\determinant{\pbrac{\tensor{I}
          +\epsilon\inverse{\tensor{A}}\tensor{U}}}}}{\epsilon=0}
  \end{split}
\end{equation}

Now, the characteristic equation for a tensor, $\tensor{B}$ is
\begin{equation}
  \determinant{\pbrac{\tensor{B}-\lambda\tensor{I}}}
  =\pbrac{\lambda_{1}-\lambda}\pbrac{\lambda_{2}-\lambda}\pbrac{\lambda_{3}-\lambda}
\end{equation}
where $\lambda_{i}$ is the $\nth{i}$ eigenvalue of $\tensor{B}$.

We thus have
\begin{equation}
  \directionalderiv{}{\determinant{\tensor{A}}}{\tensor{U}}
  =\determinant{\tensor{A}}\evalat{\dby{}{\epsilon}\determinant{\pbrac{\pbrac{1+\epsilon\lambda_{1}}\pbrac{1+\epsilon\lambda_{2}}\pbrac{1+\epsilon\lambda_{3}}}}}{\epsilon=0}
\end{equation}
where $\lambda_{i}$ is the $\nth{i}$ eigenvalue of
$\inverse{\tensor{A}}\tensor{U}$. Now
\begin{equation}
  \begin{split}
    \directionalderiv{}{\determinant{\tensor{A}}}{\tensor{U}}
    &=\determinant{\tensor{A}}\evalat{\dby{}{\epsilon}\determinant{\pbrac{\pbrac{1+\epsilon\lambda_{1}}\pbrac{1+\epsilon\lambda_{2}}\pbrac{1+\epsilon\lambda_{3}}}}}{\epsilon=0}\\
    &=\determinant{\tensor{A}}\pbrac{\lambda_{1}+\lambda_{2}+\lambda_{3}}\\
    &=\determinant{\tensor{A}}\trace{}{\pbrac{\inverse{\tensor{A}}\tensor{U}}}\\
    &=\determinant{\tensor{A}}\doubledotprod{\invtranspose{\tensor{A}}}{\tensor{U}}
  \end{split}
\end{equation}

Now, as $\tensortwo{U}$ is arbitrary, we have
\begin{equation}
  \delby{\pbrac{\determinant{\tensortwo{A}}}}{\tensortwo{A}}=\determinant{\tensortwo{A}}\invtranspose{\tensortwo{A}}
  \label{eqn:DerivDeterminantTensorSecondOrder}
\end{equation}


\subsubsection{Derivatives of the Trace}
\label{subsubsec:TraceDerivativeSecondOrder}

Consider the derivative of the trace of a second order tensor
$\trace{}{\tensortwo{A}}$ with respect to the tensor $\tensortwo{A}$ \ie
\begin{equation}
  \begin{split}
    \delby{\pbrac{\trace{\tensortwo{g}}{\tensortwo{A}}}}{\tensortwo{A}}
    &=\delby{\pbrac{g_{ij}A^{ij}}}{\tensortwo{A}}\\
    &=\delby{\pbrac{g_{11}A^{11}}}{A_{ij}}\tensorprod{\generalbasevector_{i}}{\generalbasevector_{j}}
    +\delby{\pbrac{g_{12}A^{12}}}{A_{ij}}\tensorprod{\generalbasevector_{i}}{\generalbasevector_{j}}
    +\delby{\pbrac{g_{13}A^{13}}}{A_{ij}}\tensorprod{\generalbasevector_{i}}{\generalbasevector_{j}} \\
    &\quad\delby{\pbrac{g_{21}A^{21}}}{A_{ij}}\tensorprod{\generalbasevector_{i}}{\generalbasevector_{j}}
    +\delby{\pbrac{g_{22}A^{22}}}{A_{ij}}\tensorprod{\generalbasevector_{i}}{\generalbasevector_{j}}
    +\delby{\pbrac{g_{23}A^{23}}}{A_{ij}}\tensorprod{\generalbasevector_{i}}{\generalbasevector_{j}} \\
    &\quad\delby{\pbrac{g_{31}A^{31}}}{A_{ij}}\tensorprod{\generalbasevector_{i}}{\generalbasevector_{j}}
    +\delby{\pbrac{g_{32}A^{32}}}{A_{ij}}\tensorprod{\generalbasevector_{i}}{\generalbasevector_{j}}
    +\delby{\pbrac{g_{33}A^{33}}}{A_{ij}}\tensorprod{\generalbasevector_{i}}{\generalbasevector_{j}} \\
    &=g_{11}\tensorprod{\generalbasevector_{1}}{\generalbasevector_{1}}
    +g_{12}\tensorprod{\generalbasevector_{1}}{\generalbasevector_{2}}
    +g_{13}\tensorprod{\generalbasevector_{1}}{\generalbasevector_{3}}\\
    &\quad g_{21}\tensorprod{\generalbasevector_{2}}{\generalbasevector_{1}}
    +g_{22}\tensorprod{\generalbasevector_{2}}{\generalbasevector_{2}}
    +g_{23}\tensorprod{\generalbasevector_{2}}{\generalbasevector_{3}}\\
    &\quad g_{31}\tensorprod{\generalbasevector_{3}}{\generalbasevector_{1}}
    +g_{32}\tensorprod{\generalbasevector_{3}}{\generalbasevector_{2}}
    +g_{33}\tensorprod{\generalbasevector_{3}}{\generalbasevector_{3}}\\
    &=\sharptensor{\tensortwo{g}}\\
    &=\sharptensor{\identitytensortwo}
  \end{split}
\end{equation}

[WHY DO WE HAVE FLAT g COMPONENTS BUT SHARP BASE VECTORS ABOVE???]
    
Thus we have
\begin{equation}
  \delby{\pbrac{\trace{}{\tensortwo{A}}}}{\tensortwo{A}}=\identitytensortwo
  \label{eqn:DerivTraceTensorSecondOrder}
\end{equation}

\subsubsection{Derivatives of the Inverse}
\label{subsubsec:InverseDerivativeSecondOrder}

Consider two second order tensors, $\tensortwo{A}=A_{ij}\tensorprod{\generalbasevector^{i}}{\generalbasevector^{j}}$ and $\tensortwo{U}$. To
find the derivative of $\inverse{\tensortwo{A}}$ with respect to
$\tensortwo{A}$ in the direction of $\tensortwo{U}$ consider
\begin{equation}
  \doubledotprod{\delby{\identitytensortwo}{\tensortwo{A}}}{\tensortwo{U}}
  =\doubledotprod{\delby{\pbrac{\dotprod{\inverse{\tensortwo{A}}}{\tensortwo{A}}}}{\tensortwo{A}}}{\tensortwo{U}}=0
\end{equation}

Now, by \eqnref{eqn:ProductRuleDirectDerivTensorValueTensorFunction} we have
\begin{equation}
  \doubledotprod{\delby{\pbrac{\dotprod{\inverse{\tensortwo{A}}}{\tensortwo{A}}}}{\tensortwo{A}}}{\tensortwo{U}}=
  \dotprod{\doubledotprod{\delby{\inverse{\tensortwo{A}}}{\tensortwo{A}}}{\tensortwo{U}}}{\tensortwo{A}}
  +\dotprod{\inverse{\tensortwo{A}}}{\doubledotprod{\delby{\tensortwo{A}}{\tensortwo{A}}}{\tensortwo{U}}}=0
\end{equation}
or, rearranging, we have
\begin{equation}
  \dotprod{\doubledotprod{\delby{\inverse{\tensortwo{A}}}{\tensortwo{A}}}{\tensortwo{U}}}{\tensortwo{A}}=-\dotprod{\inverse{\tensortwo{A}}}{\doubledotprod{\delby{\tensortwo{A}}{\tensortwo{A}}}{\tensortwo{U}}}
\end{equation}
or
\begin{equation}
  \dotprod{\doubledotprod{\delby{\inverse{\tensortwo{A}}}{\tensortwo{A}}}{\tensortwo{U}}}{\tensortwo{A}}=-\dotprod{\inverse{\tensortwo{A}}}{\tensortwo{U}}
\end{equation}

Post multipliying by $\inverse{\tensortwo{A}}$ gives
\begin{equation}
  \dotprodthree{\doubledotprod{\delby{\inverse{\tensortwo{A}}}{\tensortwo{A}}}{\tensortwo{U}}}{\tensortwo{A}}{\inverse{\tensortwo{A}}}=-\dotprodthree{\inverse{\tensortwo{A}}}{\tensortwo{U}}{\inverse{\tensortwo{A}}}
\end{equation}
or
\begin{equation}
  \doubledotprod{\delby{\inverse{\tensortwo{A}}}{\tensortwo{A}}}{\tensortwo{U}}=-\dotprodthree{\inverse{\tensortwo{A}}}{\tensortwo{U}}{\inverse{\tensortwo{A}}}
\end{equation}

We thus have
\begin{equation}
  \delby{\inverse{\tensortwo{A}}}{\tensortwo{A}}=\lowertensorprod{-\inverse{\tensortwo{A}}}{\inverse{\tensortwo{A}}}
  \label{eqn:DerivInverseTensorSecondOrder}
\end{equation}

In component form we have
\begin{equation}
  \doubledotprod{\delby{\inverse{\tensortwo{A}}}{\tensortwo{A}}}{\tensortwo{U}}=\delby{\inverse{A_{ij}}}{A_{kl}}U_{kl}\tensorprod{\generalbasevector^{i}}{\generalbasevector^{j}}=-\inverse{A_{ik}}U_{kl}\inverse{A_{lj}}\tensorprod{\generalbasevector^{i}}{\generalbasevector^{j}}
\end{equation}
or
\begin{equation}
  \delby{\inverse{\tensortwo{A}}}{\tensortwo{A}}=\delby{\inverse{A_{ij}}}{A_{kl}}\tensorprodfour{\generalbasevector^{i}}{\generalbasevector^{j}}{\generalbasevector^{k}}{\generalbasevector^{l}}=-\inverse{A_{ik}}\inverse{A_{lj}}\tensorprodfour{\generalbasevector^{i}}{\generalbasevector^{j}}{\generalbasevector^{k}}{\generalbasevector^{l}}
\end{equation}

We also have
\begin{equation}
  \doubledotprod{\delby{\invtranspose{\tensortwo{A}}}{\tensortwo{A}}}{\tensortwo{U}}=-\dotprodthree{\invtranspose{\tensortwo{A}}}{\transpose{\tensortwo{U}}}{\invtranspose{\tensortwo{A}}}
\end{equation}
that is
\begin{equation}
  \delby{\invtranspose{\tensortwo{A}}}{\tensortwo{A}}=\uppertensorprod{-\invtranspose{\tensortwo{A}}}{\invtranspose{\tensortwo{A}}}
  \label{eqn:DerivInverseTransposeTensorSecondOrder}
\end{equation}

In component form this is
\begin{equation}
  \doubledotprod{\delby{\inverse{\tensortwo{A}}}{\tensortwo{A}}}{\tensortwo{U}}=\delby{\inverse{A_{ji}}}{A_{kl}}U_{kl}\tensorprod{\generalbasevector^{i}}{\generalbasevector^{j}}=-\inverse{A_{jk}}U_{lk}\inverse{A_{li}}\tensorprod{\generalbasevector^{i}}{\generalbasevector^{j}}
\end{equation}
or
\begin{equation}
  \delby{\inverse{\tensortwo{A}}}{\tensortwo{A}}=\delby{\inverse{A_{ji}}}{A_{kl}}\tensorprodfour{\generalbasevector^{i}}{\generalbasevector^{j}}{\generalbasevector^{k}}{\generalbasevector^{l}}=-\inverse{A_{li}}\inverse{A_{jk}}\tensorprodfour{\generalbasevector^{i}}{\generalbasevector^{j}}{\generalbasevector^{k}}{\generalbasevector^{l}}
\end{equation}

This implies that if $\tensortwo{A}$ is symmetric \ie
$\tensortwo{A}=\transpose{\tensortwo{A}}$ then
\begin{equation}
  \delby{\inverse{\tensortwo{A}}}{\tensortwo{A}}=\symtensorprod{-\inverse{\tensortwo{A}}}{\inverse{\tensortwo{A}}}
  \label{eqn:DerivInverseSymmetricTensorSecondOrder}
\end{equation}
or, in component form,
\begin{equation}
  \delby{\inverse{\tensortwo{A}}}{\tensortwo{A}}=\delby{\inverse{A_{ij}}}{A_{kl}}\tensorprodfour{\generalbasevector^{i}}{\generalbasevector^{j}}{\generalbasevector^{k}}{\generalbasevector^{l}}=\dfrac{-1}{2}\pbrac{\inverse{A_{ik}}\inverse{A_{jl}}+\inverse{A_{il}}\inverse{A_{jk}}}\tensorprodfour{\generalbasevector^{i}}{\generalbasevector^{j}}{\generalbasevector^{k}}{\generalbasevector^{l}}
\end{equation}


\section{Exterior Algebra}
\label{sec:ExteriorAlgebra}

\subsection{Wedge Product}
\label{subsec:WedgeProduct}

Consider a $n$-dimensional vector space $\vectorspace{V}$ over the real
numbers $\rntopology{n}$. Recall that the \emph{inner-} or \emph{dot-product}
of two vectors in this space can be used to measure the angle between these
vectors. The \emph{exterior-} or \emph{wedge-product} will measure the area of
the parallelogram spanned by these two vectors. 

\subsection{Hodge Star}
\label{subsec:HodgeStar}

The \emph{Hodge star}\footnote{named after
\link{https://en.wikipedia.org/wiki/W._V._D._Hodge}{Sir William
  Vallance Douglas Hodge} (1903-1975), a British
mathematician.}\index{Hodge star} may be interpreted geometrically as
follows. Consider a three-dimensional space. Imagine a bivector in
this space, \ie two vectors $\vectr{a}$ and $\vectr{b}$ that start
from the same base point. These two vectors will create an oriented
parallelogram spanned by the vectors. This parallelogram may be
interpreted as the two-form $\wedgeprod{\vectr{a}}{\vectr{b}}$. The
operation of the Hodge star on a differential form produces a
complementary differential form. For the case above we can think of
the vector normal to the parallelogram as complementary to it. The
direction of the normal vector will depend on the orientation of the
two form. The normal to the bivector, however, only gives a
direction. To complete the complementary nature we have that
$\hodgestar{\pbrac{\wedgeprod{\vectr{a}}{\vectr{b}}}}$ will produce a
vector normal to the paralleogram whose length is equal to the area of
paralleogram which is given by $\crossprod{\vectr{a}}{\vectr{b}}$.

The opposite of this example also holds. If we consider a vector $\vectr{a}$
then $\hodgestar{\vectr{a}}$ will produce a bivector whose normal is in the
direction of $\vectr{a}$ and whose spanned area is equal to the length of
$\vectr{a}$.

\section{Exterior Calculus}
\label{sec:ExteriorCalculus}

\section{Differential Geometry view of a Continuum Problem}
\label{sec:DiffGeomViewContinuum}


\epstexfigure{DifferentialGeometry/svgs/meshembedding.eps_tex}{Embedding
  of a normalised mesh into a geometric manifold.}{A mesh on a
  normalised manifold $\manifold{N}$ without a metric and with
  $\vectr{\xi}$ coordinates is embedded into a geometric material
  Reimannian manifold $\manifold{M}$ with a metric $\tensortwo{G}$ and
  with $\vectr{X}$ coordinates. The same mesh may be embedded with
  differnt geometric fields/maps. Here the geometric field
  $\fnof{\vectr{f}}{\vectr{\xi}}$ embeds a small mesh and the
  geometric field $\fnof{\vectr{h}}{\vectr{\xi}}$ embeds a larger
  mesh. The process of embedding will induce a metric on the
  $\manifold{N}$ manifold.}{fig:MeshEmbedded}{0.5}
