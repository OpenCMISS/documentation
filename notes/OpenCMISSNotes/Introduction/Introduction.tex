\chapter{Introduction}
\label{cha:Introduction}

\section{Notation}
\label{sec:Notation}

 
\subsection{Vector and Tensors}
\label{subsec:VectorTensorNotation}

The following formatting will be used throughout theses notes for scalars, vectors, and tensors.
\begin{equation}
  \begin{split}
    a, b, c, \ldots & \quad\text{Scalars or $0^{th}$ order tensors} \\
    \vectr{a},\vectr{b},\vectr{c}, \ldots & \quad\text{Vectors or $1^{st}$ order
      tensors} \\
    \tensortwo{A},\tensortwo{B},\tensortwo{C}, \ldots & \quad\text{Dyadics or $2^{nd}$ order
      tensors} \\
    \tensorthree{A},\tensorthree{B},\tensorthree{C}, \ldots & \quad\text{Triadics or $3^{rd}$
      order tensors} \\
    \tensorfour{A},\tensorfour{B},\tensorfour{C}, \ldots & \quad\text{Tetradics or $4^{th}$
      order tensors}
  \end{split}
\end{equation}

\subsection{Summation Notation}
\label{subsec:SummationNotation}

The following (Einstein) summation notation\index{Summation Notation} will be
used throughout these notes. In order to eliminate summation symbols repeated
``dummy'' indices will be used \ie
\begin{equation}
  \gsum{i=1}{n}{a^{i}b_{i}}=a^{i}b_{i}
\end{equation}

To indicate an index that is not summed, parentheses will be used
\ie $a^{(i)}b_{(i)}$ is talking about the singular expression for $i$ \eg
$a^{1}b_{1}$, $a^{2}b_{2}$ \etc

In order to indicate a summation the sum must occur over indices that are
different sub/super-script \ie the sum must be over an ``upper'' and a
``lower'' index or a ``lower'' and an ``upper'' index. Note that it may be
useful to remember that if an index appears in the denominator of a fractional
expression then the index upper- or lower- ness is ``reversed''. 

For some quantities with both upper and lower indices a dot will be used to
indicate the ``second'' index \eg in the expression $A^{i}_{.j}$ then $i$ can
be considered the first index and $j$ the second index.
