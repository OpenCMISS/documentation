\clearemptydoublepage
\chapter{Port Hamiltonian}
\label{cha:PortHamiltonian}

In order to extend the \emph{Bond Graph} ideas to multi-dimensional space we
consider a physical space called the \emph{Bond space}, $\vectorspace{B}$, which
is made up from a \emph{Flow space}, $\vectorspace{F}$, and its dual space
called an \emph{Effort space}, $\vectorspace{E}=\dualspace{F}$, \ie
\begin{equation}
  \vectorspace{B}=\spaceprod{\vectorspace{F}}{\vectorspace{E}}=\spaceprod{\vectorspace{F}}{\dualspace{F}}
\end{equation}

\section{Wave Equation}
\label{sec:PHWaveEquation}

\subsection{Vector Calculus Port Hamiltonian Approach}

Consider the wave equation in a domain, $\Omega\subset\rntopology{n}$,
$n=1,2,3$ with boundary $\boundary{\Omega}$ \ie
\begin{equation}
  \deltwosqby{\fnof{z}{\vectr{x},t}}{t}=c^{2}\laplacian{}{\fnof{z}{\vectr{x},t}}
\end{equation}
where $\fnof{z}{\vectr{x},t}$ is the wave amplitude and $c$ is the wave speed.

The \emph{state} (or \emph{energy}) variables of this equation are
\begin{align}
    \fnof{u}{\vectr{x},t}&=\delby{\fnof{z}{\vectr{x},t}}{t}\in\rntopology{} \\
    \fnof{\vectr{v}}{\vectr{x},t}&=\gradient{}{\fnof{z}{\vectr{x},t}}\in\rntopology{n}
\end{align}

To formulate the wave equation in a port Hamiltonian form we need to define
three sets of equations. The first set called the \emph{structure} equations
relate the \emph{dual power variables} \ie the relationship between the
\emph{flows} and the \emph{efforts}. The second set of relationships is called
the \emph{dynamics} equations and relates state (or energy) variables and the
flow variables. The final set is called the \emph{constituative} equations and
relates the efforts (or \emph{co-energy} or \emph{co-state} variables) and the
the \emph{energy} or \emph{Hamiltonian} functional.

The dynamics equations for the flow variables are
\begin{align}
  \fnof{f^{u}}{\vectr{x},t}&=-\delby{\fnof{u}{\vectr{x},t}}{t}\in\rntopology{} \\
  \fnof{\vectr{f}^{\vectr{v}}}{\vectr{x},t}&=-\delby{\fnof{\vectr{v}}{\vectr{x},t}}{t}\in\rntopology{n}
\end{align}

For the linear wave equation the (quadratic) Hamiltonian density function,
$\mapping{\hamiltoniandensitysym}{\spaceprodfour{\rntopology{}}{\rntopology{n}}{\rntopology{n}}{\rntopology{}}}{\rntopology{}}$
is given by
\begin{equation}
  \hamiltoniandensity{u,\vectr{v},\vectr{x},t}=\dfrac{1}{2}\fnof{u^{2}}{\vectr{x},t}+\dfrac{1}{2}c^{2}\dotprod{\fnof{\vectr{v}}{\vectr{x},t}}{\fnof{\vectr{v}}{\vectr{x},t}}
\end{equation}
and thus the Hamiltonian function is given by
\begin{equation}
  \hamiltonian{u,\vectr{v},t}=\gint{\Omega}{}{\hamiltoniandensity{u,\vectr{v},\vectr{x},t}}{\Omega}
\end{equation}
where $\exteriorderiv{\Omega}$ is the volume form.

The variation of the Hamiltonian is given by
\begin{equation}
  \hamiltonian{u+\variationdir{u},\vectr{v}+\variationdir{\vectr{u}},t}=\hamiltonian{u,\vectr{v},t}+\gint{\Omega}{}{\pbrac{\variation{\hamiltonian{u,\vectr{v},t}}{u}\variationdir{u}+\dotprod{\variation{\hamiltonian{u,\vectr{v},t}}{\vectr{v}}}{\variationdir{\vectr{v}}}+\orderof{\variationdir{u},\variationdir{\vectr{v}}}}}{\Omega}
\end{equation}
where $\variation{\hamiltonian{u,\vectr{v},t}}{u}$ is the variation in the $u$
direction and $\variation{\hamiltonian{u,\vectr{v},t}}{\vectr{v}}$ is the
variation in the $\vectr{v}$ direction.

The constitutive equations for the efforts is thus given by
\begin{align}
  \fnof{e^{u}}{u,\vectr{v},t}&=\variation{\hamiltonian{u,\vectr{v},t}}{u}=\delby{\hamiltoniandensity{u,\vectr{v},\vectr{x},t}}{u}\in\rntopology{}\\
  \fnof{\vectr{e}^{\vectr{v}}}{u,\vectr{v},t}&=\variation{\hamiltonian{u,\vectr{v},t}}{\vectr{v}}=\delby{\hamiltoniandensity{u,\vectr{v},\vectr{x},t}}{\vectr{v}}\in\rntopology{n}
\end{align}

The constituative equations mean that the enerby balance for the system can be
written as
\begin{equation}
  \begin{aligned}
    \dothamiltonian{u,\vectr{v},t}&=\gint{\Omega}{}{\pbrac{\delby{\hamiltoniandensity{u,\vectr{v},\vectr{x},t}}{u}\delby{\fnof{u}{\vectr{x},t}}{t}+\dotprod{\delby{\hamiltoniandensity{u,\vectr{v},\vectr{x},t}}{\vectr{v}}}{\delby{\fnof{\vectr{v}}{\vectr{x},t}}{t}}}}{\Omega}\\
    &=-\gint{\Omega}{}{\pbrac{\fnof{e^{u}}{u,\vectr{v},t}\fnof{f^{u}}{\vectr{x},t}+\dotprod{\fnof{\vectr{e}^{\vectr{v}}}{u,\vectr{v},t}}{\fnof{\vectr{f}^{\vectr{v}}}{\vectr{x},t}}}}{\Omega}
  \end{aligned}
\end{equation}

If we now apply the divergence theorem we obtain
\begin{equation}
  \begin{aligned}
    -\gint{\Omega}{}{\pbrac{e^{u}f^{u}+\dotprod{\vectr{e}^{\vectr{v}}}{\vectr{f}^{\vectr{v}}}}}{\Omega}
    &=-\gint{\Omega}{}{\pbrac{e^{u}\divop
    \vectr{e}^{\vectr{v}}+\dotprod{\vectr{e}^{\vectr{v}}}{\gradop
      e^{u}}}}{\Omega}\\
    &=-\gint{\Omega}{}{\divop\pbrac{e^{u}\vectr{e}^{\vectr{v}}}}{\Omega}\\
    &=-\gint{\boundary{\Omega}}{}{e^{u}\dotprod{\vectr{e}^{\vectr{v}}}{\vectr{n}}}{\Gamma}
  \end{aligned}
\end{equation}
where $\vectr{n}$ is the normal to the boundary $\boundary{\Omega}=\Gamma$.

This can be written as a \emph{power balance equation}
\begin{equation}
  \innerform{e^{u}}{f^{u}}{\Omega}+\innerform{\vectr{e}^{\vectr{v}}}{\vectr{f}^{\vectr{v}}}{\Omega}+\innerform{e^{u}}{-\dotprod{\vectr{e}^{\vectr{v}}}{\vectr{n}}}{\boundary{\Omega}}=0
\end{equation}
where the functional form $\innerform{\cdot}{\cdot}{\Omega}$ and
$\innerform{\cdot}{\cdot}{\boundary{\Omega}}$ are the standard inner products
on the space. 

The wave equation can thus be written in Port-Hamiltonian form as
\begin{equation}
  \begin{aligned}
    \begin{bmatrix}
      \fnof{f^{u}}{\vectr{x},t} \\
      \fnof{\vectr{f}^{\vectr{v}}}{\vectr{x},t}
    \end{bmatrix}&=\begin{bmatrix}
    0 & \divop \\
    \gradop & \vectr{0}
    \end{bmatrix}\begin{bmatrix}
      \fnof{e^{u}}{\vectr{x},t} \\
      \fnof{\vectr{e}^{\vectr{v}}}{\vectr{x},t}
    \end{bmatrix}&\qquad&\text{Structure equations}\\
    \begin{bmatrix}
      \delby{\fnof{u}{\vectr{x},t}}{t} \\
      \delby{\fnof{\vectr{v}}{\vectr{x},t}}{t}      
    \end{bmatrix}&=\begin{bmatrix}
      -\fnof{f^{u}}{\vectr{x},{t}} \\
      -\fnof{\vectr{f}^{\vectr{v}}}{\vectr{x},{t}} \\          
    \end{bmatrix}&\qquad&\text{Dynamics equations}\\
    \begin{bmatrix}
      \fnof{e^{u}}{\vectr{x},t}\\
      \fnof{\vectr{e}^{\vectr{u}}}{\vectr{x},t}\\
    \end{bmatrix}&=\begin{bmatrix}
    \variation{\hamiltonian{\fnof{u}{\vectr{x},t},\fnof{\vectr{v}}{\vectr{x},t},t}}{\fnof{u}{\vectr{x},t}}\\
    \transpose{\pbrac{\variation{\hamiltonian{\fnof{u}{\vectr{x},t},\fnof{\vectr{v}}{\vectr{x},t},t}}{\fnof{\vectr{v}}{\vectr{x},t}}}}
    \end{bmatrix}&\qquad&\text{Constitutive equations}
  \end{aligned}
\end{equation}

\subsection{Exterior Calculus Port Hamiltonian Approach}

Now, in terms of differential forms we have the \emph{co-state} variables as
\begin{equation}
  \begin{aligned}
    \fnof{p}{\vectr{x},t}&=\hodgestar{u}\in\diffformspace{k_{p}}{\Omega} \\
    \fnof{\covectr{q}}{\vectr{x},t}&=\flattensor{\vectr{v}}\in\diffformspace{k_{q}}{\Omega}
  \end{aligned}
\end{equation}
where $k_{p}=\dimop p$ and $k_{q}=\dimop \covectr{q}$ are the dimensions of the spaces of $p$ and $\covectr{q}$ and
that $k_{p}+k_{q}=n+1$. For the wave equation $k_{p}=n$ and $k_{q}=1$.

Similarily the \emph{co-flow} and \emph{co-effort} variables are given by
\begin{equation}
  \begin{aligned}
    \fnof{f^{p}}{\vectr{x},t}&=\hodgestar{\fnof{f^{u}}{\vectr{x},t}}\in\ltwodiffformspace{k_{p}}{\Omega}\\
    \fnof{\covectr{f}^{\covectr{q}}}{\vectr{x},t}&=\flattensor{\pbrac{\vectr{f}^{\vectr{v}}}}\in\ltwodiffformspace{k_{q}}{\Omega}\\
    \fnof{e^{p}}{\vectr{x},t}&=\fnof{e^{u}}{\vectr{x},t}\in\honediffformspace{n-k_{p}}{\Omega}\\
    \fnof{\covectr{e}^{\covectr{q}}}{\vectr{x},t}&=\pbrac{-1}^{n-k_{q}}\hodgestar{\flattensor{\pbrac{\fnof{\vectr{e}^{\vectr{v}}}{\vectr{x},t}}}}\in\honediffformspace{n-k_{q}}{\Omega}
  \end{aligned} 
\end{equation}

The Hamiltonian is given by
\begin{equation}
  \hamiltonian{p,\covectr{q},t}=\dfrac{1}{2}\dualityform{p}{\hodgestar{p}}{\Omega}+\dfrac{1}{2}c^{2}\dualityform{\covectr{q}}{\hodgestar{\covectr{q}}}{\Omega}
\end{equation}
where the \emph{duality} product
$\dualityform{\covectr{\alpha}}{\covectr{\beta}}{\Omega}$ between the differential
forms $\covectr{\alpha}\in\diffformspace{k}{\Omega}$ and
$\covectr{\beta}\in\diffformspace{n-k}{\Omega}$ is given by
\begin{equation}
  \dualityform{\covectr{\alpha}}{\covectr{\beta}}{\Omega}=\gint{\Omega}{}{\wedgeprod{\covectr{\alpha}}{\covectr{\beta}}}{\Omega}
\end{equation}
and for $\dualityform{\covectr{\alpha}}{\covectr{\beta}}{\boundary{\Omega}}$
between the differential forms $\covectr{\alpha}\in\diffformspace{k}{\boundary{\Omega}}$ and
$\covectr{\beta}\in\diffformspace{n-1-k}{\boundary{\Omega}}$ is given by
\begin{equation}
  \dualityform{\covectr{\alpha}}{\covectr{\beta}}{\boundary{\Omega}}=\gint{\boundary{\Omega}}{}{\wedgeprod{\covectr{\alpha}}{\covectr{\beta}}}{\Gamma}
\end{equation}

The differential engery balance is given by
\begin{equation}
  \begin{aligned}
    \dothamiltonian{p,\covectr{q},t}&=\gint{\Omega}{}{\pbrac{\wedgeprod{\delby{\hamiltoniandensity{p,\covectr{q},\vectr{x},t}}{p}}{\delby{\fnof{p}{\vectr{x},t}}{t}}+\wedgeprod{\delby{\hamiltoniandensity{p,\covectr{q},\vectr{x},t}}{\covectr{q}}}{\delby{\fnof{\covectr{q}}{\vectr{x},t}}{t}}}}{\Omega}\\
    &=-\dualityform{\variation{\hamiltonian{p,\covectr{q},t}}{p}}{-\delby{\fnof{p}{\vectr{x},t}}{t}}{\Omega}-\dualityform{\variation{\hamiltonian{p,\covectr{q},t}}{\covectr{q}}}{-\delby{\fnof{\covectr{q}}{\vectr{x},t}}{t}}{\Omega}
  \end{aligned}
\end{equation}

We note that
\begin{equation}
  \dothamiltonian{p,\covectr{q},t}=(-1)^{n}\dualityform{\trace{}{
    \covectr{e}^{\covectr{q}}}}{\trace{}{e^{p}}}{\boundary{\Omega}}
\end{equation}
that is, the rate of change in the energy of the system equals the power
supplied through the boundary. 

The wave equation can be expressed in port hamiltonian form using differential
forms as
\begin{equation}
  \begin{aligned}
    \begin{bmatrix}
      \fnof{f^{p}}{\vectr{x},t} \\
      \fnof{\vectr{f}^{\vectr{q}}}{\vectr{x},t}
    \end{bmatrix}&=\begin{bmatrix}
    0 & (-1)^{k_{r}}\exteriorderivop \\
    \exteriorderivop & \vectr{0}
    \end{bmatrix}\begin{bmatrix}
      \fnof{e^{p}}{\vectr{x},t} \\
      \fnof{\vectr{e}^{\vectr{q}}}{\vectr{x},t}
    \end{bmatrix}&\qquad&\text{Structure equations}\\
    \begin{bmatrix}
      \delby{\fnof{p}{\vectr{x},t}}{t} \\
      \delby{\fnof{\vectr{q}}{\vectr{x},t}}{t}      
    \end{bmatrix}&=\begin{bmatrix}
      -\fnof{f^{p}}{\vectr{x},{t}} \\
      -\fnof{\vectr{f}^{\vectr{q}}}{\vectr{x},{t}} \\          
    \end{bmatrix}&\qquad&\text{Dynamics equations}\\
    \begin{bmatrix}
      \fnof{e^{p}}{\vectr{x},t}\\
      \fnof{\vectr{e}^{\vectr{q}}}{\vectr{x},t}\\
    \end{bmatrix}&=\begin{bmatrix}
    \variation{\hamiltonian{\fnof{p}{\vectr{x},t},\fnof{\vectr{q}}{\vectr{x},t},t}}{\fnof{p}{\vectr{x},t}}\\
    \variation{\hamiltonian{\fnof{p}{\vectr{x},t},\fnof{\vectr{q}}{\vectr{x},t},t}}{\fnof{\vectr{q}}{\vectr{x},t}}
    \end{bmatrix}&\qquad&\text{Constitutive equations}
  \end{aligned}
\end{equation}
where $\exteriorderivop$ is \emph{exterior derivative} and $k_{r}=k_{p}k_{q}+1$.

The equivalent power balance equation is given by
\begin{equation}
  \dualityform{e^{p}}{f^{p}}{\Omega}+\dualityform{\vectr{e}^{\vectr{q}}}{\vectr{f}^{\vectr{q}}}{\Omega}+(-1)^{n}\dualityform{\trace{}{
  \covectr{e}^{\covectr{q}}}}{\trace{}{e^{p}}}{\boundary{\Omega}}=0
\end{equation}
where $\trop$ extends the differential form to the boundary. Note that
$\trace{}{e^{p}}\in\ltwodiffformspace{n-k_{p}}{\boundary{\Omega}}$ and
$\trace{}{\covectr{e}^{\covectr{q}}}\in\ltwodiffformspace{n-k_{q}}{\boundary{\Omega}}$.

\subsection{Weak Form}

The \emph{weak form} of the equations is obtain from \emph{duality pairing} of
the co-flows and co-efforts with \emph{test forms} of an appropriate degree
that does not vanish on the boundary.

To allow for different boundary conditions with different causalities consider
partitioning the domain $\Omega$ into a number of subsets
$\Omega_{i},i=1,...,n_{\Omega}$ and
$\hat{\Omega}_{j},j=1,...,n_{\hat{\Omega}}$ such that
$\Omega=\union{\gunion{i=1}{n_{\Omega}}{\Omega_{i}}}{\gunion{j=1}{n_{\hat{\Omega}}}{\hat{\Omega}_{j}}}$. Similarily
the boundary $\boundary{\Omega}$ is partitioned into a number of subsets
$\Gamma_{i},i=1,...,n_{\Gamma}$ and
$\hat{\Gamma}_{j},j=1,...,n_{\hat{\Gamma}}$ such that
$\boundary{\Omega}=\union{\gunion{i=1}{n_{\Gamma}}{\Gamma_{i}}}{\gunion{j=1}{n_{\hat{\Gamma}}}{\hat{\Gamma}_{j}}}$
with $\intersection{\Gamma_{i}}{\hat{\Gamma}_{j}}=\emptyset$ for any $i$ and
$j$. The boundary flows and efforts can be written as
\begin{align}
  f^{\Gamma}_{i}&=\evalat{\trace{}{e^{p}}}{\Gamma_{i}}\in\ltwodiffformspace{n-k_{p}}{\boundary{\Omega}} \\
  \covectr{e}^{\Gamma}_{i}&=(-1)^{k_{p}}\evalat{\trace{}{\covectr{e}^{\covectr{q}}}}{\Gamma_{i}}\in\ltwodiffformspace{n-k_{q}}{\boundary{\Omega}}\\
  \hat{\covectr{f}}^{\Gamma}_{j}&=(-1)^{k_{p}}\evalat{\trace{}{\covectr{e}^{\covectr{q}}}}{\hat{\Gamma}_{j}}\in\ltwodiffformspace{n-k_{q}}{\boundary{\Omega}}\\
  \hat{e}^{\Gamma}_{j}&=\evalat{\trace{}{e^{p}}}{\hat{\Gamma}_{j}}\in\ltwodiffformspace{n-k_{p}}{\boundary{\Omega}}
\end{align}

With the boundary flows and efforts defined as such the power balance equation
becomes
\begin{equation}
  \dualityform{e^{p}}{f^{p}}{\Omega}+\dualityform{\covectr{e}^{\covectr{q}}}{\covectr{f}^{\covectr{q}}}{\Omega}+\gsum{i=1}{n_{\Gamma}}{\dualityform{\covectr{e}^{\Gamma}_{i}}{f^{\Gamma}_{i}}{\Gamma_{i}}}+\gsum{j=1}{n_{\hat{\Gamma}}}{\dualityform{\hat{\covectr{f}}^{\Gamma}_{j}}{\hat{e}^{\Gamma}_{j}}{\hat{\Gamma}_{j}}}=0
\end{equation}

Consider now some \emph{test forms} $v^{p}$ and $\covectr{v}^{\covectr{q}}$. We have from the
structural equations
\begin{align}
  \dualityform{v^{p}}{f^{p}}{\Omega}&=\dualityform{v^{p}}{(-1)^{k_{r}}\exteriorderiv{\covectr{e}^{\covectr{q}}}}{\Omega}&\qquad&\forall
  v^{p}\in\honediffformspace{n-k_{p}}{\Omega}\\ \dualityform{\covectr{v}^{\covectr{q}}}{\covectr{f}^{\covectr{q}}}{\Omega}&=\dualityform{\covectr{v}^{\covectr{q}}}{\exteriorderiv{e^{p}}}{\Omega}&\qquad&\forall
  \covectr{v}^{\covectr{q}}\in\honediffformspace{n-k_{q}}{\Omega}
\end{align}

The energy balance equations are then
\begin{equation}
  \dualityform{v^{p}}{f^{p}}{\Omega}=(-1)^{k_{r}+k_{q}}\dualityform{\exteriorderiv{v^{p}}}{\covectr{e}^{\covectr{q}}}{\Omega}-(-1)^{k_{r}+k_{q}}\gsum{i=1}{n_{\Gamma}}{\dualityform{\trace{}{v^{p}}}{\covectr{e}^{\Gamma}_{i}}{\Gamma_{i}}}-(-1)^{k_{r}+k_{q}}\gsum{j=1}{n_{\hat{\Gamma}}}{\dualityform{\trace{}{v^{p}}}{\hat{\covectr{f}}^{\Gamma}_{j}}{\hat{\Gamma}_{j}}}
\end{equation}
and
\begin{equation}
  \dualityform{\covectr{v}^{\covectr{q}}}{\covectr{f}^{\covectr{q}}}{\Omega}=(-1)^{k_{p}}\dualityform{\exteriorderiv{\covectr{v}^{\covectr{q}}}}{e^{p}}{\Omega}-(-1)^{k_{p}}\gsum{i=1}{n_{\Gamma}}{\dualityform{\trace{}{\covectr{v}^{\covectr{q}}}}{f^{\Gamma}_{i}}{\Gamma_{i}}}-(-1)^{k_{p}}\gsum{j=1}{n_{\hat{\Gamma}}}{\dualityform{\trace{}{\covectr{v}^{\covectr{q}}}}{\hat{e}^{\Gamma}_{j}}{\hat{\Gamma}_{j}}}  
\end{equation}

The flow differential forms and test forms will be taken from linear
combinations of the basis forms \ie
$\covectr{\psi}^{p}_{h}\subset\ltwodiffformspace{k_{p}}{\Omega}$ and
$\covectr{\psi}^{\covectr{q}}_{h}\subset\ltwodiffformspace{k_{q}}{\Omega}$. Note that
  there are $N_{p}$ bases for $\covectr{\psi}^{p}_{h}$ and $N_{q}$ for
  $\covectr{\psi}^{\covectr{q}}_{h}$ and $h$ is the discretisation parameter.
The effort differential forms and test forms will also be taken from the
linear combinations of the basis forms \ie
$\covectr{\phi}^{p}_{h}\subset\honediffformspace{n-k_{p}}{\Omega}$ and
$\covectr{\phi}^{\covectr{q}}_{h}\subset\honediffformspace{n-k_{q}}{\Omega}$. Note that
  there are $M_{p}$ bases for $\covectr{\phi}^{p}_{h}$ and $M_{q}$ for
  $\covectr{\phi}^{\covectr{q}}_{h}$.
  
We can now interpolate the flow and effort forms \ie
\begin{align}
  \fnof{f^{p}_{h}}{\vectr{x},t}&=\gsum{k=1}{N_{p}}{\fnof{f^{p}_{k}}{t}\fnof{\psi^{p}_{k}}{\vectr{x}}}=\dualityform{\fnof{\dcovectr{f}^{p}}{t}}{\fnof{\covectr{\psi}^{p}}{\vectr{x}}}{\Omega}\in\covectr{\psi}^{p}_{h}\\
  \fnof{\covectr{f}^{\covectr{q}}_{h}}{\vectr{x},t}&=\gsum{k=1}{N_{q}}{\fnof{\covectr{f}^{\covectr{q}}_{k}}{t}\fnof{\covectr{\psi}^{\covectr{q}}_{k}}{\vectr{x}}}=\dualityform{\fnof{\dcovectr{f}^{\covectr{q}}}{t}}{\fnof{\covectr{\psi}^{\covectr{q}}}{\vectr{x}}}{\Omega}\in\covectr{\psi}^{\covectr{q}}_{h}\\
  \fnof{e^{p}_{h}}{\vectr{x},t}&=\gsum{k=1}{M_{p}}{\fnof{e^{p}_{k}}{t}\fnof{\phi^{p}_{k}}{\vectr{x}}}=\dualityform{\fnof{\dcovectr{e}^{p}}{t}}{\fnof{\covectr{\phi}^{p}}{\vectr{x}}}{\Omega}\in\covectr{\phi}^{p}_{h}\\
  \fnof{\covectr{e}^{\covectr{q}}_{h}}{\vectr{x},t}&=\gsum{k=1}{M_{q}}{\fnof{\covectr{e}^{\covectr{q}}_{k}}{t}\fnof{\covectr{\phi}^{\covectr{q}}_{k}}{\vectr{x}}}=\dualityform{\fnof{\dcovectr{e}^{\covectr{q}}}{t}}{\fnof{\covectr{\phi}^{\covectr{q}}}{\vectr{x}}}{\Omega}\in\covectr{\phi}^{\covectr{q}}_{h}
\end{align}
where the discrete flow and effort vectors are
\begin{alignat}{3}
  \fnof{\dcovectr{f}^{p}}{t}=\transpose{\sqbrac{\fnof{f^{p}_{1}}{t},\ldots,\fnof{f^{p}_{N_{p}}}{t}}}&&\qquad\qquad&&\fnof{\covectr{\psi}^{p}}{\vectr{x}}=\transpose{\sqbrac{\fnof{\psi^{p}_{1}}{\vectr{x}},\ldots,\fnof{\psi^{p}_{N_{p}}}{\vectr{x}}}} \\
  \fnof{\dcovectr{f}^{\covectr{q}}}{t}=\transpose{\sqbrac{\fnof{\covectr{f}^{\covectr{q}}_{1}}{t},\ldots,\fnof{\covectr{f}^{\covectr{q}}_{N_{q}}}{t}}}&&\qquad\qquad&&\fnof{\covectr{\psi}^{\covectr{q}}}{\vectr{x}}=\transpose{\sqbrac{\fnof{\psi^{\covectr{q}}_{1}}{\vectr{x}},\ldots,\fnof{\psi^{\covectr{q}}_{N_{q}}}{\vectr{x}}}} \\
  \fnof{\dcovectr{e}^{p}}{t}=\transpose{\sqbrac{\fnof{e^{p}_{1}}{t},\ldots,\fnof{e^{p}_{M_{p}}}{t}}}&&\qquad\qquad&&\fnof{\covectr{\phi}^{p}}{\vectr{x}}=\transpose{\sqbrac{\fnof{\phi^{p}_{1}}{\vectr{x}},\ldots,\fnof{\phi^{p}_{M_{p}}}{\vectr{x}}}}  \\
  \fnof{\dcovectr{e}^{\covectr{q}}}{t}=\transpose{\sqbrac{\fnof{\covectr{e}^{\covectr{q}}_{1}}{t},\ldots,\fnof{\covectr{e}^{\covectr{q}}_{M_{q}}}{t}}}&&\qquad\qquad&&\fnof{\covectr{\phi}^{\covectr{q}}}{\vectr{x}}=\transpose{\sqbrac{\fnof{\phi^{\covectr{q}}_{1}}{\vectr{x}},\ldots,\fnof{\phi^{\covectr{q}}_{M_{q}}}{\vectr{x}}}}
\end{alignat}

The co-state variables can also be interpolated \ie
\begin{align}
  \fnof{p_{h}}{\vectr{x},t}&=\gsum{k=1}{N_{p}}{\fnof{p_{k}}{t}\fnof{\psi^{p}_{k}}{\vectr{x}}}=\dualityform{\fnof{\dcovectr{p}}{t}}{\fnof{\covectr{\psi}^{p}}{\vectr{x}}}{\Omega}\in\covectr{\psi}^{p}_{h} \\
  \fnof{\covectr{q}_{h}}{\vectr{x},t}&=\gsum{k=1}{N_{q}}{\fnof{\covectr{q}_{k}}{t}\fnof{\psi^{\covectr{q}}_{k}}{\vectr{x}}}=\dualityform{\fnof{\dcovectr{q}}{t}}{\fnof{\covectr{\psi}^{\covectr{q}}}{\vectr{x}}}{\Omega}\in\covectr{\psi}^{\covectr{q}}_{h}
\end{align}
where
\begin{align}
  \fnof{\dcovectr{p}}{t}=\transpose{\sqbrac{\fnof{p_{1}}{t},\ldots,\fnof{p_{N_{p}}}{t}}}&\qquad\fnof{\dcovectr{q}}{t}=\transpose{\sqbrac{\fnof{\covectr{q}_{1}}{t},\ldots,\fnof{\covectr{q}_{N_{q}}}{t}}}
\end{align}

The discretised balance of power equations become
\begin{multline}
  \dualityform{\dualityform{\dcovectr{v}^{p}}{\covectr{\phi}^{p}}{}}{\dualityform{\dcovectr{f}^{p}}{\covectr{\psi}^{p}}{}}{\Omega}-(-1)^{k_{r}+k_{q}}\dualityform{\dualityform{\dcovectr{v}^{p}}{\exteriorderiv{\covectr{\phi}^{p}}}{}}{\dualityform{\dcovectr{e}^{\covectr{q}}}{\covectr{\phi}^{\covectr{q}}}{}}{\Omega}\\
  +(-1)^{k_{r}+k_{q}}\gsum{\mu=1}{n_{\Gamma}}{\dualityform{\dualityform{\dcovectr{v}^{p}}{\covectr{\phi}^{p}}{}}{\dualityform{\dcovectr{e}^{\covectr{q}}}{\covectr{\phi}^{\covectr{q}}}{}}{\Gamma_{\mu}}}+(-1)^{k_{r}+k_{q}}\gsum{\nu=1}{n_{\hat{\Gamma}}}{\dualityform{\dualityform{\dcovectr{v}^{p}}{\covectr{\phi}^{p}}{}}{\dualityform{\dcovectr{e}^{\covectr{q}}}{\covectr{\phi}^{\covectr{q}}}{}}{\hat{\Gamma}_{\nu}}}=0
\end{multline}
and
\begin{multline}
  \dualityform{\dualityform{\dcovectr{v}^{\covectr{q}}}{\covectr{\phi}^{\covectr{q}}}{}}{\dualityform{\dcovectr{f}^{\covectr{q}}}{\covectr{\psi}^{\covectr{q}}}{}}{\Omega}-(-1)^{k_{p}}\dualityform{\dualityform{\dcovectr{v}^{\covectr{q}}}{\exteriorderiv{\covectr{\phi}^{\covectr{q}}}}{}}{\dualityform{\dcovectr{e}^{p}}{\covectr{\phi}^{p}}{}}{\Omega}\\
  +(-1)^{k_{p}}\gsum{\mu=1}{n_{\Gamma}}{\dualityform{\dualityform{\dcovectr{v}^{\covectr{q}}}{\covectr{\phi}^{\covectr{q}}}{}}{\dualityform{\dcovectr{e}^{p}}{\covectr{\phi}^{p}}{}}{\Gamma_{\mu}}}+(-1)^{k_{p}}\gsum{\nu=1}{n_{\hat{\Gamma}}}{\dualityform{\dualityform{\dcovectr{v}^{\covectr{q}}}{\covectr{\phi}^{\covectr{q}}}{}}{\dualityform{\dcovectr{e}^{p}}{\covectr{\phi}^{p}}{}}{\hat{\Gamma}_{\nu}}}=0
\end{multline}

Integrating the products of basis forms gives us systems of equations \ie
\begin{align}
  \dualityform{\dcovectr{v}^{p}}{\matr{M}_{p}\dcovectr{f}^{p}}{}+\dualityform{\dcovectr{v}^{p}}{\pbrac{\matr{K}_{p}+\gsum{\mu=1}{n_{\Gamma}}{\matr{L}^{\mu}_{p}}+\gsum{\nu=1}{n_{\hat{\Gamma}}}{\hat{\matr{L}}^{\nu}_{p}}}\dcovectr{e}^{\covectr{q}}}{}&=0 \\
  \dualityform{\dcovectr{v}^{\covectr{q}}}{\matr{M}_{\covectr{q}}\dcovectr{f}^{\covectr{q}}}{}+\dualityform{\dcovectr{v}^{\covectr{q}}}{\pbrac{\matr{K}_{\covectr{q}}+\gsum{\mu=1}{n_{\Gamma}}{\matr{L}^{\mu}_{\covectr{q}}}+\gsum{\nu=1}{n_{\hat{\Gamma}}}{\hat{\matr{L}}^{\nu}_{\covectr{q}}}}\dcovectr{e}^{p}}{}&=0
\end{align}
where
\begin{alignat}{3}
  \sqbrac{\matr{M}_{p}}_{ik}=\dualityform{\phi^{p}_{i}}{\psi^{p}_{k}}{\Omega}&&\qquad&&\sqbrac{\matr{M}_{\covectr{q}}}_{jl}=\dualityform{\phi^{\covectr{q}}_{j}}{\psi^{\covectr{q}}_{l}}{\Omega} \\
  \sqbrac{\matr{K}_{p}}_{ij}=-(-1)^{k_{r}+k_{q}}\dualityform{\exteriorderiv{\phi^{p}_{i}}}{\phi^{\covectr{q}}_{j}}{\Omega}&&\qquad&&\sqbrac{\matr{K}_{\covectr{q}}}_{ji}=-(-1)^{k_{p}}\dualityform{\exteriorderiv{\phi^{\covectr{q}}_{j}}}{\phi^{p}_{i}}{\Omega} \\
  \sqbrac{\matr{L}^{\mu}_{p}}_{ij}=(-1)^{k_{r}+k_{q}}\dualityform{\phi^{p}_{i}}{\phi^{\covectr{q}}_{j}}{\Gamma_{\mu}}&&\qquad&&\sqbrac{\matr{L}^{\mu}_{\covectr{q}}}_{ji}=(-1)^{k_{p}}\dualityform{\phi^{\covectr{q}}_{j}}{\phi^{p}_{i}}{\Gamma_{\mu}} \\
  \sqbrac{\hat{\matr{L}}^{\nu}_{p}}_{ij}=(-1)^{k_{r}+k_{q}}\dualityform{\phi^{p}_{i}}{\phi^{\covectr{q}}_{j}}{\hat{\Gamma}_{\nu}}&&\qquad&&\sqbrac{\hat{\matr{L}}^{\nu}_{\covectr{q}}}_{ji}=(-1)^{k_{p}}\dualityform{\phi^{\covectr{q}}_{j}}{\phi^{p}_{i}}{\hat{\Gamma}_{\nu}}
\end{alignat}
and $\matr{M}_{p}\in\rntopology{M_{p}\times N_{p}}$,
$\matr{M}_{\covectr{q}}\in\rntopology{M_{q}\times N_{q}}$,
$\matr{K}_{p},\matr{L}^{\mu}_{p},\hat{\matr{L}}^{\nu}_{p}\in\rntopology{M_{p}\times
  M_{q}}$, and
$\matr{K}_{\covectr{q}},\matr{L}^{\mu}_{\covectr{q}},\hat{\matr{L}}^{\nu}_{\covectr{q}}\in\rntopology{M_{q}\times
  M_{p}}$. We note that by the skew-symmetry of the edge product we have that
$\sqbrac{\matr{L}^{\mu}_{p}}_{ij}=\sqbrac{\matr{L}^{\mu}_{\covectr{q}}}_{ji}$
and
$\sqbrac{\hat{\matr{L}}^{\nu}_{p}}_{ij}=\sqbrac{\hat{\matr{L}}^{\nu}_{\covectr{q}}}_{ji}$
\ie $\matr{L}^{\mu}_{p}=\transpose{\pbrac{\matr{L}^{\mu}_{\covectr{q}}}}$ and
$\hat{\matr{L}}^{\nu}_{p}=\transpose{\pbrac{\hat{\matr{L}}^{\nu}_{\covectr{q}}}}$.

Now the equations above need to hold for arbitrary
$\dcovectr{v}^{p}\in\rntopology{M_{p}}$ and
$\dcovectr{v}^{\covectr{q}}\in\rntopology{M_{q}}$ which gives the following
equations
\begin{align}
  \matr{M}_{p}\dcovectr{f}^{p}+\pbrac{\matr{K}_{p}+\matr{L}_{p}}\dcovectr{e}^{\covectr{q}}&=\vectr{0}\\
  \matr{M}_{\covectr{q}}\dcovectr{f}^{\covectr{q}}+\pbrac{\matr{K}_{\covectr{q}}+\matr{L}_{\covectr{q}}}\dcovectr{e}^{p}&=\vectr{0}
\end{align}
where
\begin{align}
  \matr{L}_{p}&=\gsum{\mu=1}{n_{\Gamma}}{\matr{L}^{\mu}_{p}}+\gsum{\nu=1}{n_{\hat{\Gamma}}}{\hat{\matr{L}}}^{\nu}_{p}\in\rntopology{M_{p}\times
  M_{q}}\\
  \matr{L}_{\covectr{q}}&=\gsum{\mu=1}{n_{\Gamma}}{\matr{L}^{\mu}_{\covectr{q}}}+\gsum{\nu=1}{n_{\hat{\Gamma}}}{\hat{\matr{L}}}^{\nu}_{\covectr{q}}\in\rntopology{M_{q}\times
  M_{p}}
\end{align}
where $\matr{L}_{p}=\transpose{\matr{L}_{\covectr{q}}}$.

We now define pairs of \emph{discrete boundary port variables} that can be used for input
or output. The variables $\dcovectr{e}^{b,\mu}$, $\dcovectr{f}^{b,\mu_{0}}$
$\in\rntopology{M^{\mu}_{b}}$ associated with the boundary port
$\Gamma_{\mu}\subset\boundary{\Omega}$ and the variables
$\hat{\dcovectr{e}}^{b,\nu}$, $\hat{\dcovectr{f}}^{b,\nu}_{0}$
$\in\rntopology{\hat{M}^{\nu}_{b}}$ associated with the boundary port
$\hat{\Gamma}_{\nu}\subset\boundary{\Omega}$ satsify
\begin{align}
  \dualityform{\dcovectr{e}^{\covectr{q}}}{\matr{L}^{\mu}_{\covectr{q}}\dcovectr{e}^{p}}{}&=\dualityform{\dcovectr{e}^{b,\mu}}{\dcovectr{f}^{b,\mu}_{o}}{}\\
  \dualityform{\dcovectr{e}^{p}}{\hat{\matr{L}}^{\nu}_{p}\dcovectr{e}^{\covectr{q}}}{}&=\dualityform{\hat{\dcovectr{e}}^{b,\nu}}{\hat{\dcovectr{f}}^{b,\nu}_{o}}{}    
\end{align}

We now decompose th boundry power matrices for each boundary subdomain \ie
\begin{align}
  \matr{L}^{\mu}_{\covectr{q}}&=\transpose{\pbrac{\matr{T}^{\mu}_{\covectr{q}}}}\matr{S}^{\mu}_{p,0}\\
  \hat{\matr{L}}^{\nu}_{p}&=\transpose{\pbrac{\hat{\matr{T}}^{\nu}_{p}}}\hat{\matr{S}}^{\nu}_{\covectr{q},0}
\end{align}
where the \emph{boundary trace matrices}
$\matr{T}^{\mu}_{\covectr{q}}\in\rntopology{M^{\mu}_{b}\times M_{q}}$ and
$\hat{\matr{T}}^{\nu}_{p}\in\rntopology{M^{\nu}_{b}\times M_{p}}$ define the
boundary effort degrees-of-freedom \ie
\begin{align}
  \dcovectr{e}^{b,\mu}&=\matr{T}^{\mu}_{\covectr{q}}\dcovectr{e}^{\covectr{q}}\\
  \hat{\dcovectr{e}}^{b,\nu}&=\hat{\matr{T}}^{\nu}_{p}\dcovectr{e}^{p} 
\end{align}
that are the \emph{input variables}. The entries of $\matr{T}^{\mu}_{\covectr{q}}$
and $\hat{\matr{T}}^{\nu}_{p}$ are typically either $0$, $+1$ or $-1$
depending on the orientation of the boundary.

The matrices $\matr{S}^{\mu}_{p,0}\in\rntopology{M^{\mu}_{b}\times M_{p}}$ and
$\hat{\matr{S}}^{\nu}_{\covectr{q},0}\in\rntopology{\hat{M}^{\nu}_{b}\times
  M_{q}}$ are the \emph{collocated boundary output matrices} we define the
boundary flow variables \ie
\begin{align}
  \dcovectr{f}^{b,\mu}_{0}&=\matr{S}^{\mu}_{p,0}\dcovectr{e}^{p}\\
  \hat{\dcovectr{f}}^{b,\nu}_{0}&=\hat{\matr{S}}^{\nu}_{\covectr{q},0}\dcovectr{e}^{\covectr{q}}
\end{align}

Summation over the boundary ports $\Gamma_{\mu}$ and $\hat{\Gamma}_{\nu}$
gives us
\begin{equation}
  \matr{L}_{p}=\transpose{\matr{S}_{p,0}}\matr{T}_{\covectr{q}}+\transpose{\hat{\matr{T}}_{p}}\hat{\matr{S}}_{\covectr{q},0}=\transpose{\matr{L}_{\covectr{q}}}
\end{equation}
where
\begin{align}
  \matr{T}_{\covectr{q}}&=\transpose{\sqbrac{\matr{T}^{1}_{\covectr{q}},\ldots,\matr{T}^{n_{\Gamma}}_{\covectr{q}}}}\\
  \matr{S}_{p,0}&=\transpose{\sqbrac{\matr{S}^{1}_{p,0},\ldots,\matr{S}^{n_{\Gamma}}_{p,0}}}\\
  \hat{\matr{T}}_{p}&=\transpose{\sqbrac{\hat{\matr{T}}^{1}_{p},\ldots,\hat{\matr{T}}^{n_{\hat{\Gamma}}}_{p}}}\\
  \hat{\matr{S}}_{\covectr{q},0}&=\transpose{\sqbrac{\hat{\matr{S}}^{1}_{\covectr{q},0},\ldots,\hat{\matr{S}}^{n_{\hat{\Gamma}}}_{\covectr{q},0}}}
\end{align}
such that
\begin{align}
  \dcovectr{e}^{b}&=\matr{T}_{\covectr{q}}\dcovectr{e}^{\covectr{q}}\\
  \dcovectr{f}^{b}_{0}&=\matr{S}_{p,0}\dcovectr{e}^{p}\\
  \hat{\dcovectr{e}}^{b}&=\hat{\matr{T}}_{p}\dcovectr{e}^{p}\\
  \hat{\dcovectr{f}}^{b}_{0}&=\hat{\matr{S}}_{\covectr{q},0}\dcovectr{e}^{\covectr{q}}  
\end{align}
where
\begin{align}
  \dcovectr{e}^{b}&=\transpose{\sqbrac{\dcovectr{e}^{b,1},\ldots,\dcovectr{e}^{b,n_{\Gamma}}}}\\
  \dcovectr{f}^{b}_{0}&=\transpose{\sqbrac{\dcovectr{f}^{b,1}_{0},\ldots,\dcovectr{f}^{b,n_{\Gamma}}_{0}}}\\
  \hat{\dcovectr{e}}^{b}&=\transpose{\sqbrac{\dcovectr{e}^{b,1},\ldots,\dcovectr{e}^{b,n_{\hat{\Gamma}}}}}\\
  \hat{\dcovectr{f}}^{b}_{0}&=\transpose{\sqbrac{\hat{\dcovectr{f}}^{b,1}_{0},\ldots,\hat{\dcovectr{f}}^{b,n_{\hat{\Gamma}}}_{0}}}
\end{align}

The discrete power balance equation is now
\begin{equation}
  \dualityform{\dcovectr{e}^{p}}{\matr{M}_{p}\dcovectr{f}^{p}}{}+
  \dualityform{\dcovectr{e}^{\covectr{q}}}{\matr{M}_{\covectr{q}}\dcovectr{f}^{\covectr{q}}}{}+
  \dualityform{\dcovectr{e}^{b}}{\dcovectr{f}^{b}_{0}}{}+
  \dualityform{\hat{\dcovectr{e}}^{b}}{\hat{\dcovectr{f}}^{b}_{0}}{}=0
\end{equation}

The problem with the power balance above is that it is, in general,
\emph{degenerate} in that it can be zero for non-zero discrete flows/efforts
that lie in $\kernel{\matr{M}_{p}}$ or $\kernel{\matr{M}_{\covectr{q}}}$. In
order to avoid this problem \emph{power-preserving mappings} are introduced.

On the assumption that the factorisation
\begin{align}
  \matr{K}_{p}+\matr{L}_{p}&=-(-1)^{k_{r}}\matr{M}_{p}\dcovectr{d}_{p}\\
  \matr{K}_{\covectr{q}}+\matr{L}_{\covectr{q}}&=-\matr{M}_{\covectr{q}}\dcovectr{d}_{\covectr{q}}
\end{align}
exists then we can write
\begin{equation}
  \begin{bmatrix}
    \dcovectr{f}^{p} \\
    \dcovectr{f}^{\covectr{q}}
  \end{bmatrix} = \begin{bmatrix}
    \dcovectr{0} & (-1)^{k_{r}}\dcovectr{d}_{p} \\
    \dcovectr{d}_{\covectr{q}} & \dcovectr{0}
  \end{bmatrix} \begin{bmatrix}
    \dcovectr{e}^{p} \\
    \dcovectr{e}^{\covectr{q}}
  \end{bmatrix}
\end{equation}
where $\dcovectr{d}_{p}\in\rntopology{N_{p}\times M_{q}}$ and
$\dcovectr{d}_{\covectr{q}}\in\rntopology{N_{q}\times M_{p}}$ are the
\emph{discrete derivative matrices} that replace the exterior derivative in
the distributed parameter setting. The matrices $\dcovectr{d}_{p}$ and
$\dcovectr{d}_{\covectr{q}}$ are the transposed incidence matrices.

\subsection{Power-preserving mappings}

In order to overcome the degeneracy in the discrete power balance
\emph{power-preserving} mappings of the discrete flow and effort vectors are
used. In general the bilinear form
$\dualityform{\dcovectr{e}}{\matr{M}\dcovectr{f}}{}$ is degenerate. If
$\dcovectr{e}\in\rntopology{n_{e}}$ and $\dcovectr{f}\in\rntopology{n_{f}}$
where $n_{e}\neq n_{f}$ then $\matr{M}$ will be of reduced rank $r_{M}\leq
\min \pbrac{n_{e},n_{f}}$.

Now consider the vectors $\dcovectr{w}_{i}=\matr{M}\dcovectr{f}_{i}$ and
$\dcovectr{v}_{i}=\transpose{\matr{M}}\dcovectr{e}_{i}$ which span
$\image{\matr{M}}$ and $\image{\transpose{\matr{M}}}$ respectively where
$i=1,\ldots,r_{M}$.

Now suppose that the matrix $\matr{M}$ can be decomposed as
\begin{equation}
  \matr{M}=\transpose{\matr{P}_{e}}\matr{P}_{f}
\end{equation}
where
\begin{equation}
  \matr{P}_{e}=\begin{bmatrix}
  \transpose{\dcovectr{w}_{1}} \\
  \vdots \\
  \transpose{\dcovectr{w}_{r_{M}}}
  \end{bmatrix}\qquad\matr{P}_{f}\begin{bmatrix}
    \transpose{\dcovectr{v}_{1}} \\
    \vdots \\
    \transpose{\dcovectr{v}_{r_{M}}}
  \end{bmatrix}
\end{equation}
 
The degenerate duality form can now be replaced with a \emph{non-degenerate}
duality form $\dualityform{\tilde{\dcovectr{e}}}{\tilde{\dcovectr{f}}}{}$
where $\tilde{\dcovectr{e}}=\matr{P}_{e}\dcovectr{e}$ and
$\tilde{\dcovectr{f}}=\matr{P}_{f}\dcovectr{f}$. $\tilde{\dcovectr{e}},\tilde{\dcovectr{f}}\in\rntopology{\tilde{N}}$
are the \emph{minimal discrete power variables} with $\tilde{N}=r_{M}$.

If we have
\begin{equation}
  \tilde{\dcovectr{e}}^{p}=\matr{P}_{ep}\dcovectr{e}^{p}\qquad\tilde{\dcovectr{e}}^{\covectr{q}}=\matr{P}_{eq}\dcovectr{e}^{\covectr{q}}\qquad\tilde{\dcovectr{f}}^{p}=\matr{P}_{fp}\dcovectr{f}^{p}\qquad\tilde{\dcovectr{f}}^{\covectr{q}}=\matr{P}_{fq}\dcovectr{f}^{\covectr{q}}
\end{equation}
such that
\begin{equation}
  \tilde{N}_{p}=\dimension{\tilde{\dcovectr{e}}^{p}}=\dimension{\tilde{\dcovectr{f}}^{p}}\leq\rank{\matr{M}_{p}}\qquad
  \tilde{N}_{q}=\dimension{\tilde{\dcovectr{e}}^{\covectr{q}}}=\dimension{\tilde{\dcovectr{f}}^{\covectr{q}}}\leq\rank{\matr{M}_{\covectr{q}}}
\end{equation}

The vectors $\tilde{\dcovectr{f}}^{p},
\tilde{\covectr{e}}^{p}\in\rntopology{\tilde{N}_{p}}$ and $\tilde{\dcovectr{f}}^{\covectr{q}},
\tilde{\covectr{e}}^{\covectr{q}}\in\rntopology{\tilde{N}_{q}}$ are the
\emph{minimual discrete flows and efforts}. These discrete flows and efforts
are called \emph{power preserving} if that satisfy the discrete power balance
\begin{equation}
  \dualityform{\tilde{\dcovectr{e}}^{p}}{\tilde{\dcovectr{f}}^{p}}{}+\dualityform{\tilde{\dcovectr{e}}^{\covectr{q}}}{\tilde{\dcovectr{f}}^{\covectr{q}}}{}+\dualityform{\dcovectr{e}^{b}}{\dcovectr{f}^{b}}{}+\dualityform{\hat{\dcovectr{e}}^{b}}{\hat{\dcovectr{f}}^{b}}{}=0
\end{equation}

\subsection{Finite-Dimensional Port-Hamiltonian Model}

The unconstrained input-output representation of the Dirac structure is
\begin{align}
  \begin{bmatrix}
    -\tilde{\dcovectr{f}}^{p} \\
    -\tilde{\dcovectr{f}}^{\covectr{q}}
  \end{bmatrix} &= \begin{bmatrix}
    \dcovectr{0} & \matr{J}_{p} \\
    \matr{J}_{\covectr{q}} & \dcovectr{0}
  \end{bmatrix}\begin{bmatrix}
    \tilde{\dcovectr{e}}^{p} \\
    \tilde{\dcovectr{e}}^{\covectr{q}}
  \end{bmatrix}+\begin{bmatrix}
  \dcovectr{0} & \matr{B}_{p} \\
  \matr{B}_{\covectr{q}} & \dcovectr{0}
  \end{bmatrix}\begin{bmatrix}
    \hat{\dcovectr{e}}^{b} \\
    \dcovectr{e}^{b}
  \end{bmatrix} \\
  \begin{bmatrix}
    \hat{\dcovectr{f}}^{b} \\
    \dcovectr{f}^{b}
  \end{bmatrix} &= \begin{bmatrix}
    \dcovectr{0} & \matr{C}_{\covectr{q}} \\
    \matr{C}_{p} & \dcovectr{0}
  \end{bmatrix}\begin{bmatrix}
    \tilde{\dcovectr{e}}^{p} \\
    \tilde{\dcovectr{e}}^{\covectr{q}}
  \end{bmatrix}+\begin{bmatrix}
  \dcovectr{0} & \matr{D}_{\covectr{q}} \\
  \matr{D}_{p} & \dcovectr{0}
  \end{bmatrix}\begin{bmatrix}
    \hat{\dcovectr{e}}^{b} \\
    \dcovectr{e}^{b}
  \end{bmatrix}
\end{align}
where $\matr{J}_{p}=-\transpose{\matr{J}_{\covectr{q}}}$,
$\matr{C}_{\covectr{q}}=\transpose{\matr{B}_{\covectr{q}}}$,
$\matr{C}_{p}=\transpose{\matr{B}_{p}}$, and
$\matr{D}_{\covectr{q}}=-\transpose{\matr{D}_{p}}$.

MORE

\subsection{Whitney Finite Elements}

\subsubsection{Whitney Forms in 1D}

For a one-dimensional domain $\Omega=\pbrac{0,L}\subset\rntopology{}$ which is
divided into equidistant nodes (step size $h=\frac{L}{N}$)
$x_{i}=\pbrac{i-1}h, i=1,\ldots,N+1$ \ie $N$ intervals
$I_{k}=\pbrac{\pbrac{k-1}h,kh}, k=1,\ldots,N$ then the Whitney 0-forms
(node-forms) over the intervals are
\begin{equation}
  \fnof{w^{n_{i}}}{x} = \begin{cases}
    \frac{1}{h}\pbrac{x-x_{i-1}}, & x\in I_{i=1}, i>1 \\
    1-\frac{1}{h}\pbrac{x-x_{i}}, & x\in I_{i},i<N+1 \\
    0, & \text{otherwise}
  \end{cases}
\end{equation}

The Whitney 1-forms (edge forms) are
\begin{equation}
  \fnof{w^{e_{k}}}{x} = \begin{cases}
    \frac{1}{h}\oneform{x}, & x\in I_{k} \\
    0, & \text{otherwise}
  \end{cases}
\end{equation}

\subsubsection{Whitney Forms in 2D}

For a triangle $f_{1}=\bbrac{\pbrac{x,y}\vert x,y\ge 0, 0\leq x+y \leq h}$
with vertices $n_{1}=\pbrac{0,0}$, $n_{2}=\pbrac{h,0}$, and
$n_{3}=\pbrac{0,h}$ and orientated edges $e_{1}$, $e_{2}$, and $e_{3}$. The
Whitney 0-forms (node-forms) are
\begin{align}
  \fnof{w^{n_{1}}}{x,y} &= 1-\frac{x}{h}-\frac{y}{h} \\
  \fnof{w^{n_{2}}}{x,y} &= \frac{x}{h} \\
  \fnof{w^{n_{3}}}{x,y} &= \frac{y}{h}  
\end{align}
and the 1-forms (edge-forms) are
\begin{align}
  \fnof{w^{e_{1}}}{x,y}&=\oneform{\frac{h-y}{h^{2}}}{x}+\oneform{\frac{x}{h^{2}}}{y}\\
  \fnof{w^{e_{2}}}{x,y}&=\oneform{\frac{-y}{h^{2}}}{x}+\oneform{\frac{x}{h^{2}}}{y}\\
  \fnof{w^{e_{3}}}{x,y}&=\oneform{\frac{-y}{h^{2}}}{x}+\oneform{\frac{x-h}{h^{2}}}{y}
\end{align}
and the 2-form (face-forms) are
\begin{equation}
  \fnof{w^{f_{1}}}{x,y}=\twoform{\frac{2}{h^{2}}}{x}{y}
\end{equation}

\epstexfigure{PortHamiltonian/svgs/fourelemsimplex.eps_tex}{Four element
  simplex mesh.}{$2\times 1$ mesh to solve the 2D wave equation
  on.}{fig:FourElementSimplex}{0.75}
 
