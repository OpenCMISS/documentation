\chapter{The Finite Element Method}
\label{cha:FiniteElementMethod}

\section{One-dimensional Steady-State Heat Equation}
\label{sec:OneDSteadyStateHeatEquation}

\subsection{Integral equation}
\label{subsec:IntegralEquation}

\subsection{Integration by parts}
\label{subsec:IntegrationByParts}

\subsection{Finite element approximation}
\label{subsec:FiniteElementApproximation}

\subsection{Element integrals}
\label{subsec:ElementIntegrals}

\subsection{Assembly}
\label{subsec:Assembly}

\epstexfigure{FiniteElementMethod/svgs/fembasisweights.eps_tex}{Nodal basis weights and the stiffness matrix.}{The rows of the global stiffness matrix generated from the nodal global weight functions.}{fig:FEMBasisWeightFunctions}{0.6}

\subsection{Boundary conditions}
\label{subsec:BoundaryConditions}

\subsection{Solution}
\label{subsec:Solution}

\subsection{Fluxes}
\label{subsec:Fluxes}

\section{Two-dimensional Laplace Equation}
\label{sec:FEMTwoDLaplaceEquation}

We can now extend the finite element method by looking at a problem in two dimensions. Consider the Laplace\footnote{named after \link{https://en.wikipedia.org/wiki/Pierre-Simon_Laplace}{Pierre-Simon Laplace} (1749-1827), a French polymath.} equation\index{Laplace equation}
\begin{equation}
  \laplacian{}{\fnof{u}{\vectr{x}}}=0
  \label{eqn:FEMLaplaceEquation}
\end{equation}
where $\vectr{x}\in\Omega\subset\rntopology{n}$ and $\Gamma=\boundary{\Omega}$. For two dimensions $n=2$.

\subsection{Integral equation}
\label{subsec:FEMTwoDLaplaceIntegralEquation}

The weighted residual integral equation corresponding to \eqnref{eqn:FEMLaplaceEquation} is
\begin{equation}
  \gint{\Omega}{}{\laplacian{}{\fnof{u}{\vectr{x}}}\fnof{w}{\vectr{x}}}{\Omega}=0
  \label{eqn:FEMLaplaceWeightedResidual}
\end{equation}

\subsection{Integration by parts}
\label{subsec:FEMTwoDLaplaceIntegrationByParts}

The multi-dimensional equivalent of integration by parts is the Green-Gauss theorem TODO REF GG. Applying this theorem to \eqnref{eqn:FEMLaplaceWeightedResidual} gives
\begin{equation}
  \gint{\Omega}{}{\laplacian{}{\fnof{u}{\vectr{x}}}\fnof{w}{\vectr{x}}}{\Omega}=
    \gint{\Omega}{}{\dotprod{\gradient{}{\fnof{u}{\vectr{x}}}}{\gradient{}{\fnof{w}{\vectr{x}}}}}{\Omega}-
    \gint{\Gamma}{}{\dotprod{\gradient{}{\fnof{u}{\vectr{x}}}}{\vectr{n}}\fnof{w}{\vectr{x}}}{\Gamma}=0
\end{equation}
where $\vectr{n}$ is the normal to $\Gamma$. Rearranging gives
\begin{equation}
  \gint{\Omega}{}{\dotprod{\gradient{}{\fnof{u}{\vectr{x}}}}{\gradient{}{\fnof{w}{\vectr{x}}}}}{\Omega}=
  \gint{\Gamma}{}{\delby{\fnof{u}{\vectr{x}}}{\vectr{n}}\fnof{w}{\vectr{x}}}{\Gamma}
  \label{eqn:FEMLaplaceStartingIntegralEquation}
\end{equation}

\subsection{Finite element approximation}
\label{subsec:FEMTwoDLaplaceFiniteElementApproximation}

Let us know discretise the domain into a union of a number of finite
elements, $\Omega = \gunion{i=1}{N}{\Omega_{i}}$. As shown in
\figref{fig:FEMTwoDDomain} we can map each element, $\Omega_{i}$, onto
a normalised element in the $\xione,\xitwo$ plane. In each
normalised elements we can interpolate the geometry of the problem
using basis functions \ie
\begin{equation}
  \fnof{\vectr{x}}{\vectr{\xi}}=\lbfn{n}{\vectr{\xi}}\nodept{\vectr{x}}{n}
\end{equation}
and, also, then dependent variable, $u$ \ie
\begin{equation}
  \fnof{u}{\vectr{\xi}}=\lbfn{n}{\vectr{\xi}}\nodept{u}{n}
\end{equation}

\epstexfigure{FiniteElementMethod/svgs/FEMTwoDDomain.eps_tex}{Finite
  element mesh for a two-dimensional problem.}{An example of a finite
  element mesh for a two-dimensional problem. Each element
  $\Omega_{i}$ is mapped onto a normalised element in the
  $\xione,\xitwo$ plane.}{fig:FEMTwoDDomain}{0.6}

Now, the integrand on the left hand side of \eqnref{eqn:FEMLaplaceStartingIntegralEquation} can be now be evaluated using
\begin{equation}
  \dotprod{\gradient{}{\fnof{u}{\vectr{x}}}}{\gradient{}{\fnof{w}{\vectr{x}}}}=
  \delby{\fnof{u}{\vectr{x}}}{x^{k}}\delby{\fnof{w}{\vectr{x}}}{x^{k}}=
  \delby{\fnof{u}{\vectr{\xi}}}{\xi^{i}}\delby{\xi^{i}}{x^{k}}\delby{\fnof{w}{\vectr{\xi}}}{\xi^{j}}\delby{\xi^{j}}{x^{k}}
\end{equation}
where the geometric terms $\delby{\xi^{i}}{x^{k}}$ can be found from
the inverse matrix
\begin{equation}
  \sqbrac{\delby{\xi^{i}}{x^{k}}}=\inverse{\sqbrac{\delby{x^{k}}{\xi^{i}}}}
\end{equation}
which, for a \twodal element, is given by
\begin{equation}
  \begin{bmatrix}
    \delby{\xione}{x} & \delby{\xione}{x} \\
    \delby{\xitwo}{y} & \delby{\xitwo}{y}
  \end{bmatrix} = \inverse{ \begin{bmatrix}
    \delby{x}{\xione} & \delby{x}{\xitwo} \\
    \delby{y}{\xione} & \delby{y}{\xitwo}
  \end{bmatrix}}=\dfrac{1}{\abs{\delby{x}{\xione}\delby{y}{\xitwo}-\delby{x}{\xitwo}\delby{y}{\xione}}}
  \begin{bmatrix}
    \delby{y}{\xitwo} & -\delby{x}{\xitwo} \\
    -\delby{y}{\xione} & \delby{x}{\xione}    
  \end{bmatrix}
\end{equation}

\Eqnref{eqn:FEMTwoDLaplaceStartingIntegralEquation} now becomes
\begin{equation}
abc  
\end{equation}
or
\begin{equation}
  \matr{K}\vectr{u}=\vectr{f}
  \label{eqn:FEMTwoDLaplaceMatrixEquation}
\end{equation}

\subsection{Element integrals}
\label{subsec:FEMTwoDLaplaceElementIntegrals}

In order to evaluate the integrals to arrive at
\eqnref{eqn:FEMTwoDLaplaceMatrixEquation} we consider a specific
example as shown in \figref{fig:FEMTwoDLaplaceDomain}.

\epstexfigure{FiniteElementMethod/svgs/FEMTwoDLaplaceDomain.eps_tex}{Finite
  element mesh for a two-dimensional Laplace problem.}{Finite element
  mesh for a two-dimensional Laplace problem. The domain, $\Omega$, is
  of length $L$ and height $H$ and has $N_{x}=3$ elements in the $x$
  direction and $N_{y}=3$ elements in the $y$
  direction.}{fig:FEMTwoDLaplaceDomain}{0.6}

For the geometric coordinates we have for each element $\Omega_{i}$
\begin{equation}
  \begin{bmatrix}
    \fnof{x}{\xione,\xitwo} \\
    \fnof{y}{\xione,\xitwo}
  \end{bmatrix}= \begin{bmatrix}
    x_{i}+\dfrac{\xione L}{N_{x}} \\
    y_{i}+\dfrac{\xitwo H}{N_{y}}  
  \end{bmatrix}
\end{equation}
where $\pbrac{x_{i},y_{i}}$ is the coordinate of the bottom left of element $i$. We thus have
\begin{equation}
  \begin{bmatrix}
    \delby{x}{\xione} & \delby{x}{\xitwo} \\
    \delby{y}{\xione} & \delby{y}{\xitwo}     
  \end{bmatrix} = \begin{bmatrix}
    \dfrac{L}{N_{x}} & 0 \\
    0 & \dfrac{H}{N_{y}}
  \end{bmatrix}
\end{equation}

Inverting we have
\begin{equation}
  \begin{bmatrix}
    \fnof{\xione}{x,y} \\
    \fnof{\xitwo}{x,y}
  \end{bmatrix} = \begin{bmatrix}
    \dfrac{\pbrac{x-x_{i}}N_{x}}{L} \\
    \dfrac{\pbrac{y-y_{i}}N_{y}}{H} 
  \end{bmatrix}
\end{equation}
and
\begin{equation}
  \begin{bmatrix}
    \delby{\xione}{x} & \delby{\xione}{x} \\
    \delby{\xitwo}{x} & \delby{\xitwo}{y}         
  \end{bmatrix} = \dfrac{1}{\frac{L}{N_{x}}\frac{H}{N_{y}}-0.0} \begin{bmatrix}
    \dfrac{H}{N_{y}} & 0 \\
    0 & \dfrac{L}{N_{x}}
  \end{bmatrix} = \begin{bmatrix}
    \dfrac{N_{x}}{L} & 0 \\
    0 & \dfrac{N_{y}}{H}
  \end{bmatrix}
\end{equation}

The Jacobian between $\pbrac{x,y}$ and $\pbrac{\xione,\xitwo}$ is given by
\begin{equation}
  \abs{\fnof{J}{\vectr{\xi}}}=\dfrac{LH}{N_{x}N_{y}}
\end{equation}

Now, if we consider the LHS of \eqnref{eqn:FEMTwoDLaplaceStartingIntegralEquation} we have
\begin{equation}
  \begin{aligned}
    \gint{\Omega}{}{\dotprod{\gradient{}{\fnof{u}{\vectr{x}}}}{\gradient{}{\fnof{w}{\vectr{x}}}}}{\Omega}&=
    \giint{0}{H}{0}{L}{\gradient{}{\fnof{u}{x,y}}\transpose{\pbrac{\gradient{}{\fnof{w}{x,y}}}}}{x}{y}\\
    &=\giint{0}{H}{0}{L}{\begin{bmatrix}\delby{u}{x}&\delby{u}{y}\end{bmatrix}
      \transpose{\begin{bmatrix}\delby{w}{x}&\delby{w}{x}\end{bmatrix}}}{x}{y} \\
    &=\dintl{0}{1}\dintl{0}{1}\begin{bmatrix}\delby{u}{\xione}&\delby{u}{\xitwo}\end{bmatrix}\begin{bmatrix}
      \delby{\xione}{x} & \delby{\xione}{y} \\
      \delby{\xitwo}{x} & \delby{\xitwo}{y}
    \end{bmatrix} \\
    &\qquad\qquad\qquad\transpose{\pbrac{\begin{bmatrix}\delby{w}{\xione}&\delby{w}{\xitwo}\end{bmatrix}\begin{bmatrix}
      \delby{\xione}{x} & \delby{\xione}{y} \\
      \delby{\xitwo}{x} & \delby{\xitwo}{y}
    \end{bmatrix}}}\abs{\fnof{J}{\xione,\xitwo}}\,\exteriorderivop\xione\wedge\exteriorderivop\xitwo \\ 
    &=\dintl{0}{1}\dintl{0}{1}\begin{bmatrix}\delby{u}{\xione}&\delby{u}{\xitwo}\end{bmatrix}\begin{bmatrix}
      \dfrac{N_{x}}{L} & 0 \\
      0 & \dfrac{N_{y}}{H}
    \end{bmatrix} \\
    &\qquad\qquad\qquad\transpose{\pbrac{\begin{bmatrix}\delby{w}{\xione}&\delby{w}{\xitwo}\end{bmatrix}\begin{bmatrix}
      \dfrac{N_{x}}{L} & 0 \\
      0 & \dfrac{N_{y}}{H}
    \end{bmatrix}}}\dfrac{LH}{N_{x}N_{y}}\,\exteriorderivop\xione\wedge\exteriorderivop\xitwo \\
    &=\giint{0}{1}{0}{1}{\begin{bmatrix}\delby{u}{\xione}\dfrac{N_{x}}{L}&\delby{u}{\xitwo}\dfrac{N_{y}}{H}\end{bmatrix}
      \transpose{\begin{bmatrix}\delby{w}{\xione}\dfrac{N_{x}}{L}&\delby{w}{\xitwo}\dfrac{N_{y}}{H}\end{bmatrix}}
      \dfrac{LH}{N_{x}N_{y}}}{\xione}{\xitwo}\\
    &=\giint{0}{1}{0}{1}{\pbrac{\delby{u}{\xione}\delby{w}{\xione}\dfrac{N_{x}^{2}}{L^{2}}+
        \delby{u}{\xitwo}\delby{w}{\xitwo}\dfrac{N_{y}^{2}}{H^{2}}}\dfrac{LH}{N_{x}N_{y}}}{\xione}{\xitwo}\\
    &=\giint{0}{1}{0}{1}{\pbrac{\delby{u}{\xione}\delby{w}{\xione}\dfrac{H N_{x}}{L N_{y}}+
        \delby{u}{\xitwo}\delby{w}{\xitwo}\dfrac{L N_{y}}{H N_{x}}}}{\xione}{\xitwo}
  \end{aligned}
\end{equation}

We can now introduce basis functions to interpolate
$\fnof{u}{\xione,\xitwo}=\lbfn{n}{\xione,\xitwo}\nodept{u}{n}$ and
$\fnof{w}{\xione,\xitwo}=\lbfn{m}{\xione,\xitwo}$. The elemental
stiffness matrix for element $i$ thus becomes
\begin{equation}
  K_{mn}=\giint{0}{1}{0}{1}{\pbrac{\delby{\lbfn{m}{\xione,\xitwo}}{\xione}\delby{\lbfn{n}{\xione,\xitwo}}{\xione}
      \dfrac{H N_{x}}{L N_{y}}+\delby{\lbfn{m}{\xione,\xitwo}}{\xitwo}\delby{\lbfn{n}{\xione,\xitwo}}{\xitwo}
      \dfrac{L N_{y}}{H N_{x}}}}{\xione}{\xitwo}
\end{equation}

Consider using bilinear Lagrange basis functions \ie 
\begin{alignat*}{3}
  \lbfn{1}{\xione,\xitwo}&=\pbrac{1-\xione}\pbrac{1-\xitwo}&\quad&&\delby{\lbfn{1}{\xione,\xitwo}}{\xione}&=-\pbrac{1-\xitwo}\\
  &&&&\delby{\lbfn{1}{\xione,\xitwo}}{\xitwo}&=-\pbrac{1-\xione} \\
  \lbfn{2}{\xione,\xitwo}&=\xione\pbrac{1-\xitwo}&\qquad&&\delby{\lbfn{2}{\xione,\xitwo}}{\xione}&=\pbrac{1-\xitwo}\\
  &&&&\delby{\lbfn{2}{\xione,\xitwo}}{\xitwo}&=-\xione\\
  \lbfn{3}{\xione,\xitwo}&=\pbrac{1-\xione}\xitwo&\qquad&&\delby{\lbfn{3}{\xione,\xitwo}}{\xione}&=-\xitwo\\
  &&&&\delby{\lbfn{3}{\xione,\xitwo}}{\xitwo}&=\pbrac{1-\xione}\\
  \lbfn{4}{\xione,\xitwo}&=\xione\xitwo&\qquad&&\delby{\lbfn{4}{\xione,\xitwo}}{\xione}&=\xitwo\\
  &&&&\delby{\lbfn{4}{\xione,\xitwo}}{\xitwo}&=\xione
\end{alignat*}

The integrals for the elemental stiffness matrix are thus

\begin{equation}
  \begin{aligned}
    K_{11}&=\giint{0}{1}{0}{1}{\pbrac{\delby{\lbfn{1}{\xione,\xitwo}}{\xione}\delby{\lbfn{1}{\xione,\xitwo}}{\xione}
        \dfrac{H N_{x}}{L N_{y}}+\delby{\lbfn{1}{\xione,\xitwo}}{\xitwo}\delby{\lbfn{1}{\xione,\xitwo}}{\xitwo}
        \dfrac{L N_{y}}{H N_{x}}}}{\xione}{\xitwo}\\
    &=\giint{0}{1}{0}{1}{\pbrac{\pbrac{-\pbrac{1-\xitwo}.-\pbrac{1-\xitwo}}\dfrac{H N_{x}}{L N_{y}}+
    \pbrac{-\pbrac{1-\xione}.-\pbrac{1-\xione}}\dfrac{L N_{y}}{H N_{x}}}}{\xione}{\xitwo} \\
    &=\dfrac{H N_{x}}{L N_{y}}\giint{0}{1}{0}{1}{\pbrac{1-2\xitwo+\xitwosq}}{\xione}{\xitwo}+
    \dfrac{L N_{y}}{H N_{x}}\giint{0}{1}{0}{1}{\pbrac{1-2\xione+\xionesq}}{\xione}{\xitwo} \\
    &=\dfrac{H N_{x}}{L N_{y}}\gint{0}{1}{\sqbrac{\pbrac{\xitwo-\xitwosq+\dfrac{\xitwocube}{3}}}_{0}^{1}}{\xione}+
    \dfrac{L N_{y}}{H N_{x}}\gint{0}{1}{\sqbrac{\pbrac{\xione-\xionesq+\dfrac{\xionecube}{3}}}_{0}^{1}}{\xitwo} \\
    &=\dfrac{H N_{x}}{L N_{y}}\gint{0}{1}{\dfrac{1}{3}}{\xione}+\dfrac{L N_{y}}{H N_{x}}\gint{0}{1}{\dfrac{1}{3}}{\xitwo} \\
    &=\dfrac{H N_{x}}{3 L N_{y}}+\dfrac{L N_{y}}{3 H N_{x}} \\
    &=\dfrac{2 H^{2}N_{x}^{2}+2 L^{2}N_{y}^{2}}{6 L H N_{x} N_{y}}
  \end{aligned}
\end{equation}

where the remaining terms for the elment stiffness matrix are given in \subsecref{subsec:ElementStiffnessMatrixLaplace2D}

The element stiffness matrix is thus given by \eqnref{eqn:FEMTwoDLaplaceElementStiffnessMatrix} \ie
\begin{equation}
  K_{mn}=\begin{bmatrix}
  \dfrac{L^{2}N_{y}^{2}+H^{2}N_{x}^{2}}{3 L H N_{x} N_{y}} & \dfrac{-2 L^{2}N_{y}^{2}+H^{2}N_{x}^{2}}{6 L H N_{x} N_{y}} &
  \dfrac{L^{2}N_{y}^{2}-2 H^{2}N_{x}^{2}}{6 L H N_{x} N_{y}} & \dfrac{-L^{2}N_{y}^{2}-H^{2}N_{x}^{2}}{6 L H N_{x} N_{y}} \\
  \dfrac{-2 L^{2}N_{y}^{2}+H^{2}N_{x}^{2}}{6 L H N_{x} N_{y}} & \dfrac{L^{2}N_{y}^{2}+H^{2}N_{x}^{2}}{3 L H N_{x} N_{y}} &
  \dfrac{-L^{2}N_{y}^{2}-H^{2}N_{x}^{2}}{6 L H N_{x} N_{y}} & \dfrac{L^{2}N_{y}^{2}-2 H^{2}N_{x}^{2}}{6 L H N_{x} N_{y}} \\
  \dfrac{L^{2}N_{y}^{2}-2 H^{2}N_{x}^{2}}{6 L H N_{x} N_{y}} & \dfrac{-L^{2}N_{y}^{2}-H^{2}N_{x}^{2}}{6 L H N_{x} N_{y}} &
  \dfrac{L^{2}N_{y}^{2}+H^{2}N_{x}^{2}}{3 L H N_{x} N_{y}} & \dfrac{-2 L^{2}N_{y}^{2}+H^{2}N_{x}^{2}}{6 L H N_{x} N_{y}} \\
  \dfrac{-L^{2}N_{y}^{2}-H^{2}N_{x}^{2}}{6 L H N_{x} N_{y}} & \dfrac{L^{2}N_{y}^{2}-2 H^{2}N_{x}^{2}}{6 L H N_{x} N_{y}} &
  \dfrac{-2 L^{2}N_{y}^{2}+H^{2}N_{x}^{2}}{6 L H N_{x} N_{y}} & \dfrac{L^{2}N_{y}^{2}+H^{2}N_{x}^{2}}{3 L H N_{x} N_{y}}
  \end{bmatrix}
\end{equation}

For the case where we have square elements \ie
$\dfrac{L}{N_{x}}=\dfrac{H}{N_{y}}$, the element stiffness matrix is
given by \eqnref{eqn:FEMTwoDLaplaceSquareElementStiffnessMatrix}, that
is
\begin{equation}
  K_{mn}=\begin{bmatrix}
  +\frac{2}{3} & -\frac{1}{6} & -\frac{1}{6} & -\frac{1}{3} \\
  -\frac{1}{6} & +\frac{2}{3} & -\frac{1}{3} & -\frac{1}{6} \\
  -\frac{1}{6} & -\frac{1}{3} & +\frac{2}{3} & -\frac{1}{6} \\
  -\frac{1}{3} & -\frac{1}{6} & -\frac{1}{6} & +\frac{2}{3}
  \end{bmatrix}
\end{equation}
 
\subsection{Assembly}
\label{subsec:FEMTwoDLaplaceAssembly}

The next stage of the finite element method is to assemble the
individual elemental stiffness matrices into a global stiffness
matrix. For the example show in \figref{fig:FEMTwoDLaplaceDomain}
where $L=H$ and $N=N_{x}=N_{y}$ and so the elements are square and the
elemental stiffness matrix is given by
\eqnref{eqn:FEMTwoDLaplaceSquareElementStiffnessMatrix} and thus we
have
\begin{equation}
  \matr{K}=\tiny\begin{bmatrix}
  \fracc{Red}{2}{3} & \mathcolor{Red}-\fracc{Red}{1}{6} & 0 & 0 & \mathcolor{Red}-\fracc{Red}{1}{6} & \mathcolor{Red}-\fracc{Red}{1}{3} & 0 & 0 & 0 & 0 & 0 & 0 & 0 & 0 & 0 & 0 \\
  \mathcolor{Red}-\fracc{Red}{1}{6} & \begin{matrix}\fracc{Red}{2}{3}\\\mathcolor{Green}+\fracc{Green}{2}{3}\end{matrix} & \mathcolor{Green}-\fracc{Green}{1}{6} & 0 & \mathcolor{Red}-\fracc{Red}{1}{3} & \begin{matrix}\mathcolor{Red}-\fracc{Red}{1}{6}\\ \hfil{ }\mathcolor{Green}-\fracc{Green}{1}{6}\end{matrix} & \mathcolor{Green}-\fracc{Green}{1}{3} & 0 & 0 & 0 & 0 & 0 & 0 & 0 & 0 & 0 \\
  0 & \mathcolor{Green}-\fracc{Green}{1}{6} & \begin{matrix}\fracc{Green}{2}{3}\\\mathcolor{Blue}+\fracc{Blue}{2}{3}\end{matrix} & \mathcolor{Blue}-\fracc{Blue}{1}{6} & 0 & \mathcolor{Green}-\fracc{Green}{1}{3} & \begin{matrix}\mathcolor{Green}-\fracc{Green}{1}{6}\\ \hfil{ }\mathcolor{Blue}-\fracc{Blue}{1}{6}\end{matrix} &\mathcolor{Blue}-\fracc{Blue}{1}{3} & 0 & 0 & 0 & 0 & 0 & 0 & 0 & 0 \\
  0 & 0 & \mathcolor{Blue}-\fracc{Blue}{1}{6} & \fracc{Blue}{2}{3} & 0 & 0 & \mathcolor{Blue}-\fracc{Blue}{1}{3} & \mathcolor{Blue}-\fracc{Blue}{1}{6} & 0 & 0 & 0 & 0 & 0 & 0 & 0 & 0 \\
  \mathcolor{Red}-\fracc{Red}{1}{6} & \mathcolor{Red}-\fracc{Red}{1}{3} & 0 & 0 & \begin{matrix}\fracc{Red}{2}{3}\\ \mathcolor{Cyan}+\fracc{Cyan}{2}{3}\end{matrix} & \begin{matrix}\mathcolor{Red}-\fracc{Red}{1}{6}\\ \hfil{ }\mathcolor{Cyan}-\fracc{Cyan}{1}{6}\end{matrix} & 0 & 0 & \mathcolor{Cyan}-\fracc{Cyan}{1}{6} & \mathcolor{Cyan}-\fracc{Cyan}{1}{3} & 0 & 0 & 0 & 0 & 0 & 0 \\
  \mathcolor{Red}-\fracc{Red}{1}{3} & \begin{matrix}\mathcolor{Red}-\fracc{Red}{1}{6}\\ \hfil{ }\mathcolor{Green}-\fracc{Green}{1}{6}\end{matrix} & \mathcolor{Green}-\fracc{Green}{1}{3} & 0 & \begin{matrix}\mathcolor{Red}-\fracc{Red}{1}{6}\\ \hfil{ }\mathcolor{Cyan}-\fracc{Cyan}{1}{6}\end{matrix} & \begin{matrix}\fracc{Red}{2}{3}\mathcolor{Green}+\fracc{Green}{2}{3}\\ \mathcolor{Cyan}+\fracc{Cyan}{2}{3}\mathcolor{Magenta}+\fracc{Magenta}{2}{3}\end{matrix} & \begin{matrix}\mathcolor{Green}-\fracc{Green}{1}{6}\\ \hfil{ }\mathcolor{Magenta}-\fracc{Magenta}{1}{6}\end{matrix} & 0 & \mathcolor{Cyan}-\fracc{Cyan}{1}{3} & \begin{matrix}\mathcolor{Cyan}-\fracc{Cyan}{1}{6}\\ \hfil{ }\mathcolor{Magenta}-\fracc{Magenta}{1}{6}\end{matrix} & \mathcolor{Magenta}-\fracc{Magenta}{1}{3} & 0 & 0 & 0 & 0 & 0 \\
  0 & \mathcolor{Green}-\fracc{Green}{1}{3} & \begin{matrix}\mathcolor{Green}-\fracc{Green}{1}{6}\\ \hfil{ }\mathcolor{Blue}-\fracc{Blue}{1}{6}\end{matrix} & \mathcolor{Blue}-\fracc{Blue}{1}{3} & 0 & \begin{matrix}\mathcolor{Green}-\fracc{Green}{1}{6}\\ \hfil{ }\mathcolor{Magenta}-\fracc{Magenta}{1}{6}\end{matrix} & \begin{matrix}\fracc{Green}{2}{3}\mathcolor{Blue}+\fracc{Blue}{2}{3}\\ \mathcolor{Magenta}+\fracc{Magenta}{2}{3}\mathcolor{Gold}+\fracc{Gold}{2}{3}\end{matrix} & \begin{matrix}\mathcolor{Blue}-\fracc{Blue}{1}{6}\\ \hfil{ }\mathcolor{Gold}-\fracc{Gold}{1}{6}\end{matrix} & 0 & \mathcolor{Magenta}-\fracc{Magenta}{1}{3} & \begin{matrix}\mathcolor{Magenta}-\fracc{Magenta}{1}{6}\\ \hfil{ }\mathcolor{Gold}-\fracc{Gold}{1}{6}\end{matrix} & \mathcolor{Gold}-\fracc{Gold}{1}{3} & 0 & 0 & 0 & 0 \\
  0 & 0 & \mathcolor{Blue}-\fracc{Blue}{1}{3} & \mathcolor{Blue}-\fracc{Blue}{1}{6} & 0 & 0 & \begin{matrix}\mathcolor{Blue}-\fracc{Blue}{1}{6}\\ \hfil{ }\mathcolor{Gold}-\fracc{Gold}{1}{6}\end{matrix} & \begin{matrix}\fracc{Blue}{2}{3}\\ \mathcolor{Gold}+\fracc{Gold}{2}{3}\end{matrix} & 0 & 0 & \mathcolor{Gold}-\fracc{Gold}{1}{3} & \mathcolor{Gold}-\fracc{Gold}{1}{6} & 0 & 0 & 0 & 0 \\
  0 & 0 & 0 & 0 & \mathcolor{Cyan}-\fracc{Cyan}{1}{6} & \mathcolor{Cyan}-\fracc{Cyan}{1}{3} & 0 & 0 & \begin{matrix}\fracc{Cyan}{2}{3}\\ \mathcolor{Purple}+\fracc{Purple}{2}{3}\end{matrix} & \begin{matrix}\mathcolor{Cyan}-\fracc{Cyan}{1}{6}\\ \hfil{ }\mathcolor{Purple}-\fracc{Purple}{1}{6}\end{matrix} & 0 & 0 & \mathcolor{Purple}-\fracc{Purple}{1}{6} & \mathcolor{Purple}-\fracc{Purple}{1}{3} & 0 & 0 \\
  0 & 0 & 0 & 0 & \mathcolor{Cyan}-\fracc{Cyan}{1}{3} & \begin{matrix}\mathcolor{Cyan}-\fracc{Cyan}{1}{6}\\ \hfil{ }\mathcolor{Magenta}-\fracc{Magenta}{1}{6}\end{matrix} & \mathcolor{Magenta}-\fracc{Magenta}{1}{3} & 0 & \begin{matrix}\mathcolor{Cyan}-\fracc{Cyan}{1}{6}\\ \hfil{ }\mathcolor{Purple}-\fracc{Purple}{1}{6}\end{matrix} & \begin{matrix}\fracc{Cyan}{2}{3}\mathcolor{Magenta}+\fracc{Magenta}{2}{3}\\\mathcolor{Purple}+\fracc{Purple}{2}{3}\mathcolor{Orange}+\fracc{Orange}{2}{3}\end{matrix} & \begin{matrix}\mathcolor{Magenta}-\fracc{Magenta}{1}{6}\\ \hfil{ }\mathcolor{Orange}-\fracc{Orange}{1}{6}\end{matrix} & 0 & \mathcolor{Purple}-\fracc{Purple}{1}{3} & \begin{matrix}\mathcolor{Purple}-\fracc{Purple}{1}{6}\\ \hfil{ }\mathcolor{Orange}-\fracc{Orange}{1}{6}\end{matrix} & \mathcolor{Orange}-\fracc{Orange}{1}{3} & 0 \\
  0 & 0 & 0 & 0 & 0 & \mathcolor{Magenta}-\fracc{Magenta}{1}{3} & \begin{matrix}\mathcolor{Magenta}-\fracc{Magenta}{1}{6}\\ \hfil{ }\mathcolor{Gold}-\fracc{Gold}{1}{6}\end{matrix} & \mathcolor{Gold}-\fracc{Gold}{1}{3} & 0 & \begin{matrix}\mathcolor{Magenta}-\fracc{Magenta}{1}{6}\\ \hfil{ }\mathcolor{Orange}-\fracc{Orange}{1}{6}\end{matrix} & \begin{matrix}\fracc{Magenta}{2}{3}\mathcolor{Gold}+\fracc{Gold}{2}{3}\\ \mathcolor{Orange}+\fracc{Orange}{2}{3}\mathcolor{Lime}+\fracc{Lime}{2}{3}\end{matrix} & \begin{matrix}\mathcolor{Gold}-\fracc{Gold}{1}{6}\\ \hfil{ }\mathcolor{Lime}-\fracc{Lime}{1}{6}\end{matrix} & 0 & \mathcolor{Orange}-\fracc{Orange}{1}{3} & \begin{matrix}\mathcolor{Orange}-\fracc{Orange}{1}{6}\\ \hfil{ }\mathcolor{Lime}-\fracc{Lime}{1}{6}\end{matrix} & \mathcolor{Lime}-\fracc{Lime}{1}{3} \\
  0 & 0 & 0 & 0 & 0 & 0 & \mathcolor{Gold}-\fracc{Gold}{1}{3} & \mathcolor{Gold}-\fracc{Gold}{1}{6} & 0 & 0 & \begin{matrix}\mathcolor{Gold}-\fracc{Gold}{1}{6}\\ \hfil{ }\mathcolor{Lime}-\fracc{Lime}{1}{6}\end{matrix} & \begin{matrix}\fracc{Gold}{2}{3}\\ \mathcolor{Lime}+\fracc{Lime}{2}{3}\end{matrix} & 0 & 0 & \mathcolor{Lime}-\fracc{Lime}{1}{3} & \mathcolor{Lime}-\fracc{Lime}{1}{6} \\
  0 & 0 & 0 & 0 & 0 & 0 & 0 & 0 & \mathcolor{Purple}-\fracc{Purple}{1}{6} & \mathcolor{Purple}-\fracc{Purple}{1}{3} & 0 & 0 & \fracc{Purple}{2}{3} & \mathcolor{Purple}-\fracc{Purple}{1}{6} & 0 & 0 \\
  0 & 0 & 0 & 0 & 0 & 0 & 0 & 0 & \mathcolor{Purple}-\fracc{Purple}{1}{3} & \begin{matrix}\mathcolor{Purple}-\fracc{Purple}{1}{6}\\ \hfil{ }\mathcolor{Orange}-\fracc{Orange}{1}{6}\end{matrix} & \mathcolor{Orange}-\fracc{Orange}{1}{3} & 0 & \mathcolor{Purple}-\fracc{Purple}{1}{6} & \begin{matrix}\fracc{Purple}{2}{3}\\ \mathcolor{Orange}+\fracc{Orange}{2}{3}\end{matrix} & \mathcolor{Orange}-\fracc{Orange}{1}{6} & 0 \\
  0 & 0 & 0 & 0 & 0 & 0 & 0 & 0 & 0 & \mathcolor{Orange}-\fracc{Orange}{1}{3} & \begin{matrix}\mathcolor{Orange}-\fracc{Orange}{1}{6}\\ \hfil{ }\mathcolor{Lime}-\fracc{Lime}{1}{6}\end{matrix} & \mathcolor{Lime}-\fracc{Lime}{1}{3} & 0 & \mathcolor{Orange}-\fracc{Orange}{1}{6} & \begin{matrix}\fracc{Orange}{2}{3}\\ \mathcolor{Lime}+\fracc{Lime}{2}{3}\end{matrix} & \mathcolor{Lime}-\fracc{Lime}{1}{6} \\
  0 & 0 & 0 & 0 & 0 & 0 & 0 & 0 & 0 & 0 & \mathcolor{Lime}-\fracc{Lime}{1}{3} & \mathcolor{Lime}-\fracc{Lime}{1}{6} & 0 & 0 & \mathcolor{Lime}-\fracc{Lime}{1}{6} & \fracc{Lime}{2}{3}
  \end{bmatrix}
  \normalsize
\end{equation}

This leads to a global system of equations
\begin{equation}
  \begin{bmatrix}
  \frac{2}{3} & -\frac{1}{6} & 0 & 0 & -\frac{1}{6} & -\frac{1}{3} & 0 & 0 & 0 & 0 & 0 & 0 & 0 & 0 & 0 & 0 \\
  -\frac{1}{6} & \frac{4}{3} & -\frac{1}{6} & 0 & -\frac{1}{3} & -\frac{1}{3} & -\frac{1}{3} & 0 & 0 & 0 & 0 & 0 & 0 & 0 & 0 & 0 \\
  0 & -\frac{1}{6} & \frac{4}{3} & -\frac{1}{6} & 0 & -\frac{1}{3} & -\frac{1}{3} & -\frac{1}{3} & 0 & 0 & 0 & 0 & 0 & 0 & 0 & 0 \\
  0 & 0 & -\frac{1}{6} & \frac{2}{3} & 0 & 0 & -\frac{1}{3} & -\frac{1}{6} & 0 & 0 & 0 & 0 & 0 & 0 & 0 & 0 \\
  -\frac{1}{6} & -\frac{1}{3} & 0 & 0 & \frac{4}{3} & -\frac{1}{3} & 0 & 0 & -\frac{1}{6} & -\frac{1}{3} & 0 & 0 & 0 & 0 & 0 & 0 \\
  -\frac{1}{3} & -\frac{1}{3} & -\frac{1}{3} & 0 & -\frac{1}{3} & \frac{8}{3} & -\frac{1}{3} & 0 & -\frac{1}{3} & -\frac{1}{3} & -\frac{1}{3} & 0 & 0 & 0 & 0 & 0 \\
  0 & -\frac{1}{3} & -\frac{1}{3} & -\frac{1}{3} & 0 & -\frac{1}{3} & \frac{8}{3} & -\frac{1}{3} & 0 & -\frac{1}{3} & -\frac{1}{3} & -\frac{1}{3} & 0 & 0 & 0 & 0 \\
  0 & 0 & -\frac{1}{3} & -\frac{1}{6} & 0 & 0 & -\frac{1}{3} & \frac{4}{3} & 0 & 0 & -\frac{1}{3} & -\frac{1}{6} & 0 & 0 & 0 & 0 \\
  0 & 0 & 0 & 0 & -\frac{1}{6} & -\frac{1}{3} & 0 & 0 & \frac{4}{3} & -\frac{1}{3} & 0 & 0 & -\frac{1}{6} & -\frac{1}{3} & 0 & 0 \\
  0 & 0 & 0 & 0 & -\frac{1}{3} & -\frac{1}{3} & -\frac{1}{3} & 0 & -\frac{1}{3} & \frac{8}{3} & -\frac{1}{3} & 0 & -\frac{1}{3} & -\frac{1}{3} & -\frac{1}{3} & 0 \\
  0 & 0 & 0 & 0 & 0 & -\frac{1}{3} & -\frac{1}{3} & -\frac{1}{3} & 0 & -\frac{1}{3} & \frac{8}{3} & -\frac{1}{3} & 0 & -\frac{1}{3} & -\frac{1}{3} & -\frac{1}{3} \\
  0 & 0 & 0 & 0 & 0 & 0 & -\frac{1}{3} & -\frac{1}{6} & 0 & 0 & -\frac{1}{3} & \frac{4}{3} & 0 & 0 & -\frac{1}{3} & -\frac{1}{6} \\
  0 & 0 & 0 & 0 & 0 & 0 & 0 & 0 & -\frac{1}{6} & -\frac{1}{3} & 0 & 0 & \frac{2}{3} & -\frac{1}{6} & 0 & 0 \\
  0 & 0 & 0 & 0 & 0 & 0 & 0 & 0 & -\frac{1}{3} & -\frac{1}{3} & -\frac{1}{3} & 0 & -\frac{1}{6} & \frac{4}{3} & -\frac{1}{6} & 0 \\
  0 & 0 & 0 & 0 & 0 & 0 & 0 & 0 & 0 & -\frac{1}{3} & -\frac{1}{3} & -\frac{1}{3} & 0 & -\frac{1}{6} & \frac{4}{3} & -\frac{1}{6} \\
  0 & 0 & 0 & 0 & 0 & 0 & 0 & 0 & 0 & 0 & -\frac{1}{3} & -\frac{1}{6} & 0 & 0 & -\frac{1}{6} & \frac{2}{3}
  \end{bmatrix}\begin{bmatrix}
    \nodept{u}{1} \\ \nodept{u}{2} \\ \nodept{u}{3} \\ \nodept{u}{4} \\ \nodept{u}{5} \\ \nodept{u}{6} \\ \nodept{u}{7} \\ \nodept{u}{8} \\ \nodept{u}{9} \\ \nodept{u}{10} \\ \nodept{u}{11} \\ \nodept{u}{12} \\ \nodept{u}{13} \\ \nodept{u}{14} \\ \nodept{u}{15} \\ \nodept{u}{16}
  \end{bmatrix}=\begin{bmatrix}
    f_{1} \\ f_{2} \\ f_{3} \\ f_{4} \\ f_{5} \\ f_{6} \\ f_{7} \\ f_{8} \\ f_{9} \\ f_{10} \\ f_{11} \\ f_{12} \\ f_{13} \\ f_{14} \\ f_{15} \\ f_{16}
  \end{bmatrix}
\end{equation}

\subsection{Boundary conditions}
\label{subsec:FEMTwoDLaplaceBoundaryConditions}

We can now apply boundary conditions to the problem. For this example we will set all nodes on the boundary to the value of the analytic solution given by
\begin{equation}
  \fnof{u}{x,y}=x^{2}-y^{2}
\end{equation}

With these boundary conditions our system of equations becomes
\begin{equation}
  \begin{bmatrix}
  \frac{2}{3} & -\frac{1}{6} & 0 & 0 & -\frac{1}{6} & -\frac{1}{3} & 0 & 0 & 0 & 0 & 0 & 0 & 0 & 0 & 0 & 0 \\
  -\frac{1}{6} & \frac{4}{3} & -\frac{1}{6} & 0 & -\frac{1}{3} & -\frac{1}{3} & -\frac{1}{3} & 0 & 0 & 0 & 0 & 0 & 0 & 0 & 0 & 0 \\
  0 & -\frac{1}{6} & \frac{4}{3} & -\frac{1}{6} & 0 & -\frac{1}{3} & -\frac{1}{3} & -\frac{1}{3} & 0 & 0 & 0 & 0 & 0 & 0 & 0 & 0 \\
  0 & 0 & -\frac{1}{6} & \frac{2}{3} & 0 & 0 & -\frac{1}{3} & -\frac{1}{6} & 0 & 0 & 0 & 0 & 0 & 0 & 0 & 0 \\
  -\frac{1}{6} & -\frac{1}{3} & 0 & 0 & \frac{4}{3} & -\frac{1}{3} & 0 & 0 & -\frac{1}{6} & -\frac{1}{3} & 0 & 0 & 0 & 0 & 0 & 0 \\
  -\frac{1}{3} & -\frac{1}{3} & -\frac{1}{3} & 0 & -\frac{1}{3} & \frac{8}{3} & -\frac{1}{3} & 0 & -\frac{1}{3} & -\frac{1}{3} & -\frac{1}{3} & 0 & 0 & 0 & 0 & 0 \\
  0 & -\frac{1}{3} & -\frac{1}{3} & -\frac{1}{3} & 0 & -\frac{1}{3} & \frac{8}{3} & -\frac{1}{3} & 0 & -\frac{1}{3} & -\frac{1}{3} & -\frac{1}{3} & 0 & 0 & 0 & 0 \\
  0 & 0 & -\frac{1}{3} & -\frac{1}{6} & 0 & 0 & -\frac{1}{3} & \frac{4}{3} & 0 & 0 & -\frac{1}{3} & -\frac{1}{6} & 0 & 0 & 0 & 0 \\
  0 & 0 & 0 & 0 & -\frac{1}{6} & -\frac{1}{3} & 0 & 0 & \frac{4}{3} & -\frac{1}{3} & 0 & 0 & -\frac{1}{6} & -\frac{1}{3} & 0 & 0 \\
  0 & 0 & 0 & 0 & -\frac{1}{3} & -\frac{1}{3} & -\frac{1}{3} & 0 & -\frac{1}{3} & \frac{8}{3} & -\frac{1}{3} & 0 & -\frac{1}{3} & -\frac{1}{3} & -\frac{1}{3} & 0 \\
  0 & 0 & 0 & 0 & 0 & -\frac{1}{3} & -\frac{1}{3} & -\frac{1}{3} & 0 & -\frac{1}{3} & \frac{8}{3} & -\frac{1}{3} & 0 & -\frac{1}{3} & -\frac{1}{3} & -\frac{1}{3} \\
  0 & 0 & 0 & 0 & 0 & 0 & -\frac{1}{3} & -\frac{1}{6} & 0 & 0 & -\frac{1}{3} & \frac{4}{3} & 0 & 0 & -\frac{1}{3} & -\frac{1}{6} \\
  0 & 0 & 0 & 0 & 0 & 0 & 0 & 0 & -\frac{1}{6} & -\frac{1}{3} & 0 & 0 & \frac{2}{3} & -\frac{1}{6} & 0 & 0 \\
  0 & 0 & 0 & 0 & 0 & 0 & 0 & 0 & -\frac{1}{3} & -\frac{1}{3} & -\frac{1}{3} & 0 & -\frac{1}{6} & \frac{4}{3} & -\frac{1}{6} & 0 \\
  0 & 0 & 0 & 0 & 0 & 0 & 0 & 0 & 0 & -\frac{1}{3} & -\frac{1}{3} & -\frac{1}{3} & 0 & -\frac{1}{6} & \frac{4}{3} & -\frac{1}{6} \\
  0 & 0 & 0 & 0 & 0 & 0 & 0 & 0 & 0 & 0 & -\frac{1}{3} & -\frac{1}{6} & 0 & 0 & -\frac{1}{6} & \frac{2}{3}
  \end{bmatrix}\begin{bmatrix}
    0 \\ \frac{L^{2}}{N^{2}} \\ \frac{4L^{2}}{N^{2}} \\ \frac{9L^{2}}{N^{2}} \\ -\frac{L^{2}}{N^{2}} \\ \nodept{u}{6} \\ \nodept{u}{7} \\ \frac{8L^{2}}{N^{2}} \\ -\frac{4L^{2}}{N^{2}} \\ \nodept{u}{10} \\ \nodept{u}{11} \\ \frac{5L^{2}}{N^{2}} \\ \frac{-9L^{2}}{N^{2}} \\ \frac{-8L^{2}}{N^{2}} \\ \frac{-5L^{2}}{N^{2}} \\ 0
  \end{bmatrix}=\begin{bmatrix}
    f_{1} \\ f_{2} \\ f_{3} \\ f_{4} \\ f_{5} \\ 0 \\ 0 \\ f_{8} \\ f_{9} \\ 0 \\ 0 \\ f_{12} \\ f_{13} \\ f_{14} \\ f_{15} \\ f_{16}
  \end{bmatrix}
\end{equation}

Application of the boundary conditions allow for the global system of
equations to be reduced in size. If the value at node $i$ is known
then the \nth{i} equation row can be removed and replaced with the
value for $\nodept{u}{i}$. The \nth{i} row can be removed despite the RHS
value $f_{i}$ being unknown because it is decoupled from the other
equations. Note that the RHS value for the internal (not on the
boundary) nodes will be $0$ because the boundary, $\Gamma$ does not
pass through the internal nodes and each basis function that is not
zero at the internal node will be zero at the boundary. The integral
and RHS value will thus be zero.

The reduced system of equations is thus 
\begin{equation}
  \begin{bmatrix}
  \frac{8}{3} & -\frac{1}{3} & -\frac{1}{3} & -\frac{1}{3} \\
  -\frac{1}{3} & \frac{8}{3} & -\frac{1}{3} & -\frac{1}{3} \\
  -\frac{1}{3} & -\frac{1}{3} & \frac{8}{3} & -\frac{1}{3} \\
  -\frac{1}{3} & -\frac{1}{3} & -\frac{1}{3} & \frac{8}{3}
  \end{bmatrix}\begin{bmatrix}
    \nodept{u}{6} \\ \nodept{u}{7} \\ \nodept{u}{10} \\ \nodept{u}{11} 
  \end{bmatrix}=\begin{bmatrix}
  \frac{1}{3}.0+\frac{1}{3}\frac{L^{2}}{N^{2}}+\frac{1}{3}\frac{4L^{2}}{N^{2}}-\frac{1}{3}\frac{L^{2}}{N^{2}}-\frac{1}{3}\frac{4L^{2}}{N^{2}}\\
  \frac{1}{3}\frac{L^{2}}{N^{2}}+\frac{1}{3}\frac{4L^{2}}{N^{2}}+\frac{1}{3}\frac{9L^{2}}{N^{2}}+\frac{1}{3}\frac{8L^{2}}{N^{2}}+\frac{1}{3}\frac{5L^{2}}{N^{2}}\\
  \frac{1}{3}\frac{-L^{2}}{N^{2}}+\frac{1}{3}\frac{-4L^{2}}{N^{2}}+\frac{1}{3}\frac{-9L^{2}}{N^{2}}+\frac{1}{3}\frac{-8L^{2}}{N^{2}}+\frac{1}{3}\frac{-5L^{2}}{N^{2}}\\
  \frac{1}{3}\frac{8L^{2}}{N^{2}}+\frac{1}{3}\frac{5L^{2}}{N^{2}}+\frac{1}{3}\frac{-8L^{2}}{N^{2}}+\frac{1}{3}\frac{-5L^{2}}{N^{2}}+\frac{1}{3}.0
  \end{bmatrix}=\begin{bmatrix}
  0 \\
  \frac{9L^{2}}{N^{2}} \\
  \frac{-9L^{2}}{N^{2}} \\
  0
  \end{bmatrix}
  \label{eqn:FEMTwoDLaplaceReducedSystem}
\end{equation}

\subsection{Solution}
\label{subsec:FEMTwoDLaplaceSolution}

The unknown values of the internal nodes can be found by solving \eqnref{eqn:FEMTwoDLaplaceReducedSystem} to give
\begin{equation}
  \begin{bmatrix}
    \nodept{u}{6} \\
    \nodept{u}{7} \\
    \nodept{u}{10} \\
    \nodept{u}{11}    
  \end{bmatrix} = \begin{bmatrix}
    0 \\
    \frac{3L^{2}}{N^{2}} \\
    \frac{-3L^{2}}{N^{2}} \\
    0
  \end{bmatrix}
  \label{eqn:FEMTwoDLaplaceSolution}  
\end{equation}

\subsection{Fluxes}
\label{subsec:FEMTwoDLaplaceFluxes}

We can now find the fluxes by back-substituting the known solution \ie
\begin{equation}
  \footnotesize\begin{bmatrix}
  f_{1} \\ f_{2} \\ f_{3} \\ f_{4} \\ f_{5} \\ f_{6} \\ f_{7} \\ f_{8} \\ f_{9} \\ f_{10} \\ f_{11} \\ f_{12} \\ f_{13} \\ f_{14} \\ f_{15} \\ f_{16}
  \end{bmatrix}=\begin{bmatrix}
  \frac{2}{3} & -\frac{1}{6} & 0 & 0 & -\frac{1}{6} & -\frac{1}{3} & 0 & 0 & 0 & 0 & 0 & 0 & 0 & 0 & 0 & 0 \\
  -\frac{1}{6} & \frac{4}{3} & -\frac{1}{6} & 0 & -\frac{1}{3} & -\frac{1}{3} & -\frac{1}{3} & 0 & 0 & 0 & 0 & 0 & 0 & 0 & 0 & 0 \\
  0 & -\frac{1}{6} & \frac{4}{3} & -\frac{1}{6} & 0 & -\frac{1}{3} & -\frac{1}{3} & -\frac{1}{3} & 0 & 0 & 0 & 0 & 0 & 0 & 0 & 0 \\
  0 & 0 & -\frac{1}{6} & \frac{2}{3} & 0 & 0 & -\frac{1}{3} & -\frac{1}{6} & 0 & 0 & 0 & 0 & 0 & 0 & 0 & 0 \\
  -\frac{1}{6} & -\frac{1}{3} & 0 & 0 & \frac{4}{3} & -\frac{1}{3} & 0 & 0 & -\frac{1}{6} & -\frac{1}{3} & 0 & 0 & 0 & 0 & 0 & 0 \\
  -\frac{1}{3} & -\frac{1}{3} & -\frac{1}{3} & 0 & -\frac{1}{3} & \frac{8}{3} & -\frac{1}{3} & 0 & -\frac{1}{3} & -\frac{1}{3} & -\frac{1}{3} & 0 & 0 & 0 & 0 & 0 \\
  0 & -\frac{1}{3} & -\frac{1}{3} & -\frac{1}{3} & 0 & -\frac{1}{3} & \frac{8}{3} & -\frac{1}{3} & 0 & -\frac{1}{3} & -\frac{1}{3} & -\frac{1}{3} & 0 & 0 & 0 & 0 \\
  0 & 0 & -\frac{1}{3} & -\frac{1}{6} & 0 & 0 & -\frac{1}{3} & \frac{4}{3} & 0 & 0 & -\frac{1}{3} & -\frac{1}{6} & 0 & 0 & 0 & 0 \\
  0 & 0 & 0 & 0 & -\frac{1}{6} & -\frac{1}{3} & 0 & 0 & \frac{4}{3} & -\frac{1}{3} & 0 & 0 & -\frac{1}{6} & -\frac{1}{3} & 0 & 0 \\
  0 & 0 & 0 & 0 & -\frac{1}{3} & -\frac{1}{3} & -\frac{1}{3} & 0 & -\frac{1}{3} & \frac{8}{3} & -\frac{1}{3} & 0 & -\frac{1}{3} & -\frac{1}{3} & -\frac{1}{3} & 0 \\
  0 & 0 & 0 & 0 & 0 & -\frac{1}{3} & -\frac{1}{3} & -\frac{1}{3} & 0 & -\frac{1}{3} & \frac{8}{3} & -\frac{1}{3} & 0 & -\frac{1}{3} & -\frac{1}{3} & -\frac{1}{3} \\
  0 & 0 & 0 & 0 & 0 & 0 & -\frac{1}{3} & -\frac{1}{6} & 0 & 0 & -\frac{1}{3} & \frac{4}{3} & 0 & 0 & -\frac{1}{3} & -\frac{1}{6} \\
  0 & 0 & 0 & 0 & 0 & 0 & 0 & 0 & -\frac{1}{6} & -\frac{1}{3} & 0 & 0 & \frac{2}{3} & -\frac{1}{6} & 0 & 0 \\
  0 & 0 & 0 & 0 & 0 & 0 & 0 & 0 & -\frac{1}{3} & -\frac{1}{3} & -\frac{1}{3} & 0 & -\frac{1}{6} & \frac{4}{3} & -\frac{1}{6} & 0 \\
  0 & 0 & 0 & 0 & 0 & 0 & 0 & 0 & 0 & -\frac{1}{3} & -\frac{1}{3} & -\frac{1}{3} & 0 & -\frac{1}{6} & \frac{4}{3} & -\frac{1}{6} \\
  0 & 0 & 0 & 0 & 0 & 0 & 0 & 0 & 0 & 0 & -\frac{1}{3} & -\frac{1}{6} & 0 & 0 & -\frac{1}{6} & \frac{2}{3}
  \end{bmatrix}\begin{bmatrix}
    0 \\ \frac{L^{2}}{N^{2}} \\ \frac{4L^{2}}{N^{2}} \\ \frac{9L^{2}}{N^{2}} \\ -\frac{L^{2}}{N^{2}} \\ 0 \\ \frac{3L^{2}}{N^{2}} \\ \frac{8L^{2}}{N^{2}} \\ -\frac{4L^{2}}{N^{2}} \\ \frac{-3L^{2}}{N^{2}} \\ 0 \\ \frac{5L^{2}}{N^{2}} \\ \frac{-9L^{2}}{N^{2}} \\ \frac{-8L^{2}}{N^{2}} \\ \frac{-5L^{2}}{N^{2}} \\ 0
  \end{bmatrix} \\
  =\begin{bmatrix}
  0 \\ 0 \\ 0 \\ \frac{3L^{2}}{N^{2}} \\ 0 \\ 0 \\ 0 \\ \frac{6L^{2}}{N^{2}} \\ 0 \\ 0 \\ 0 \\ \frac{6L^{2}}{N^{2}} \\ \frac{-3L^{2}}{N^{2}} \\ \frac{-6L^{2}}{N^{2}} \\ \frac{-6L^{2}}{N^{2}} \\ 0
  \end{bmatrix}
\end{equation}

We note that the sum of the components of the flux vector is zero as expected (flux in must equal flux out).
