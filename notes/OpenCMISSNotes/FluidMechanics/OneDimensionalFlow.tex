\section{One Dimensional Navier-Stokes Equations}
\label{sec:OneDNavierStokesEquations}

\subsection{Governing equations}
\label{subsec:OneDNSGoverningEquations}

For blood flow in the vascular system we can make some simplifying
assumptions. If we adopt a cylindrical coordinate system
$\pbrac{r,\theta,x}$ where the $x$ direction is aligned with axial
direction of the vessel then the flow velocity is given by
$\vectr{v}=\pbrac{v_{r},v_{\theta},v_{x}}$. Two assumptions are that
flow in the axial direction of the vascular is large compared to blood
flow in non-axial directions and that for fully dependent flow there
is no variation of flow in the circumfrential $\theta$
direction. These assumptions simplify the Navier-Stokes momentum
equation for incompressible Newtonian fluids in
\eqnref{eqn:DiffIncompNewtonianNSEMomentum} to
\begin{equation}
  \rho\delby{v_{x}}{t}+\rho\pbrac{v_{x}\delby{v_{x}}{x}+v_{r}\delby{v_{x}}{r}}=
  -\delby{p}{x}+\dfrac{\mu}{r}\delby{}{r}\pbrac{r\delby{v_{x}}{r}}
  \label{eqn:OneDNSIncompNewtonianMomentum}
\end{equation}
and simply the conservation of mass equation in \eqnref{eqn:DiffIncompNSEMass} to
\begin{equation}
  \delby{v_{x}}{x}+\dfrac{1}{r}\delby{\pbrac{r v_{r}}}{r}
  \label{eqn:OneDNSIncompMass}
\end{equation}

We can now reformulate the \onedal Navier-Stokes equations in terms of
flow, $Q$, cross-sectional area, $A$, and pressure, $p$. Integrating
\eqnref{eqn:OneDNSIncompNewtonianMomentum} over the cross-sectional
area of the vessel gives


\epstexfigure{FluidMechanics/svgs/branchingtopology.eps_tex}{One
  dimensional branching topology.}{Branching topology for a \onedal
  flow domain. The parent domain, $\Omega_{p}$ branches into two
  daughter vessels, $\Omega_{d_{1}}$ and $\Omega_{d_{2}}$. The
  non-branching nodes $\nodenumber{1}$, $\nodenumber{3}$ and
  $\nodenumber{4}$ have one flow and one area \DoF. The branching
  node, $\nodenumber{2}$ has a number of versions of the flow and area
  \DoFs, one version for the parent, $Q_{2,v_{1}}$ and $A_{2,v_{1}}$,
  and one version for each daughter, $Q_{2,v_{2}}$, $Q_{2,v_{3}}$ and
  $A_{2,v_{2}}$, $A_{2,v_{3}}$.}{fig:OneDNSBranchingTopology}{0.75}

\subsection{Terminal impedances}
\label{subsec:OneDNSTerminalImpedances}


\begin{figure*}[t!]
  \centering
  \epstexsubfigure{FluidMechanics/svgs/oneelementwindkessel.eps_tex}{One
    element windkessel.}{One element windkessel.}{fig:oneelementwindkessel}{0.5}{0.3}
  ~ 
  \epstexsubfigure{FluidMechanics/svgs/twoelementwindkessel.eps_tex}{Two
    element windkessel.}{Two element windkessel.}{fig:twoelementwindkessel}{0.5}{0.3}
  ~ 
  \epstexsubfigure{FluidMechanics/svgs/threeelementwindkessel.eps_tex}{Three
    element windkessel.}{Three element windkessel.}{fig:threeelementwindkessel}{0.5}{0.3}
  \caption{Terminal impedance models.}
\end{figure*}
 
