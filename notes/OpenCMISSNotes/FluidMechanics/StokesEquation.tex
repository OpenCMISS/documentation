\section{Stokes Equations}
\label{sec:StokesEquations}

\subsection{Governing equations}

Stokes flow is a type of fluid flow where advective inertial forces are small
compared with viscous forces. The Reynolds number is low, \ie
$\reynoldsnum\ll 1$. For this type of flow, the flow velocities are small and
the nonlinear convective accelaration terms are assumed to be negligible. The
Navier-Stokes equations can thus be simplified to give the Stokes
equations. In the common case of an incompressible Newtonian fluid, the
\emph{Stokes equation} can be found by dropping the convective acceleration
term in \eqnref{eqn:DiffIncompNewtonianNSEMomentum} \ie
\begin{equation}
  \addtolength{\fboxsep}{5pt}
  \boxed{
    \fnof{\rho}{\vectr{x}}\delby{\fnof{\vectr{v}}{\vectr{x},t}}{t}=
    \dotprod{-\gradient{}{\fnof{p}{\vectr{x},t}}}{\sharptensor{\tensor{g}}}
    +\divergence{}{\fnof{\mu}{\vectr{x}}\pbrac{\gradient{}{\fnof{\vectr{v}}{\vectr{x},t}+
          \transpose{\gradient{}{\fnof{\vectr{v}}{\vectr{x},t}}}}}}
    +\fnof{\vectr{s}}{\vectr{x},t}
  }
  \label{eqn:StokesEquationMomentum}
\end{equation}
accompanied by the conservation of mass equation
\begin{equation}
  \addtolength{\fboxsep}{5pt}
  \boxed{
    \divergence{}{\fnof{\vectr{v}}{\vectr{x},t}}=0
  }
  \label{eqn:StokesEquationMass}
\end{equation}
where $\vectr{v}(\vectr{x},t)=\transpose{\sqbrac{v_1,v_2,v_3}}$ is the
velocity vector depending on spatial coordinates
$\vectr{x}=\transpose{\sqbrac{x_1,x_2,x_3}}$ and the time $t$,
$\fnof{p}{\vectr{x},t}$ is the scalar pressure field,
$\fnof{\rho}{\vectr{x}}$ is the material density field,
$\fnof{\mu}{\vectr{x}}$ is the material viscosity field, and
$\fnof{\vectr{s}}{\vectr{x},t}$ the applied body force sources respectively.

Note that \eqnref{eqn:StokesEquationMomentum} uses the full divergence form of the cauchy stress form. For this form the deviatoric part of the cauchy stress for a Newtonian fluid is given by
\begin{equation}
  \divergence{}{\fnof{\mu}{\vectr{x}}\pbrac{\gradient{}{\fnof{\vectr{v}}{\vectr{x},t}
        +\transpose{\gradient{}{\fnof{\vectr{v}}{\vectr{x},t}}}}}}
\end{equation}
The above expression can be simplified using \eqnref{eqn:StokesEquationMass} such that
\begin{equation}
  \divergence{}{\fnof{\mu}{\vectr{x}}\pbrac{\gradient{}{\fnof{\vectr{v}}{\vectr{x},t}
       +\transpose{\gradient{}{\fnof{\vectr{v}}{\vectr{x},t}}}}}}
      =\fnof{\mu}{\vectr{x}}\laplacian{}{\fnof{\vectr{v}}{\vectr{x},t}}
\end{equation}
which gives a momentum equation of the form
\begin{equation}
  \addtolength{\fboxsep}{5pt}
  \boxed{
    \fnof{\rho}{\vectr{x}}\delby{\fnof{\vectr{v}}{\vectr{x},t}}{t}=
    \dotprod{-\gradient{}{\fnof{p}{\vectr{x},t}}}{\sharptensor{\tensor{g}}}
    +\fnof{\mu}{\vectr{x}}\laplacian{}{\fnof{\vectr{v}}{\vectr{x},t}}
    +\fnof{\vectr{s}}{\vectr{x},t}
  }
  \label{eqn:StokesEquationLaplaceMomentum}
\end{equation}
which is known as the \emph{Laplace Stokes equation}.

Whereas \eqnref{eqn:StokesEquationMomentum} has been formulated in Eulerian
form, moving domain approaches often require the ALE modification taking an
additional term into account, depending on the fluid density and a correction
velocity $\fnof{\vectr{v}^*}{\vectr{x},t}$ which yields to the \emph{ALE Stokes
  equation} \ie
\begin{equation}
  \addtolength{\fboxsep}{5pt}
  \boxed{
    \begin{gathered}
      \fnof{\rho}{\vectr{x}}\delby{\fnof{\vectr{v}}{\vectr{x},t}}{t}
      -\fnof{\rho}{\vectr{x}}\dotprod{\fnof{\vectr{v}^{*}}{\vectr{x},t}}{
        \gradient{}{\fnof{\vectr{v}}{\vectr{x},t}}}\\
      =\dotprod{-\gradient{}{\fnof{p}{\vectr{x},t}}}{\sharptensor{\tensor{g}}}
      +\divergence{}{\fnof{\mu}{\vectr{x}}\pbrac{\gradient{}{\fnof{\vectr{v}}{\vectr{x},t}+
            \transpose{\gradient{}{\fnof{\vectr{v}}{\vectr{x},t}}}}}}
      +\fnof{\vectr{s}}{\vectr{x},t}
    \end{gathered}
    }
  \label{eqn:StokesEquationALEMomentum}
\end{equation}

In the steady state a Stokes flow has no dependence on time (other than
through changing boundary conditions). We can thus modify
\eqnref{eqn:StokesEquationMomentum} to drop the time derivative on the left
hand side to give the \emph{static Stokes equation} \ie
\begin{equation}
  \addtolength{\fboxsep}{5pt}
  \boxed{
    \vectr{0}=\dotprod{-\gradient{}{\fnof{p}{\vectr{x},t}}}{\sharptensor{\tensor{g}}}
    +\divergence{}{\fnof{\mu}{\vectr{x}}\pbrac{\gradient{}{\fnof{\vectr{v}}{\vectr{x},t}+
          \transpose{\gradient{}{\fnof{\vectr{v}}{\vectr{x},t}}}}}}
    +\fnof{\vectr{s}}{\vectr{x},t}
  }
  \label{eqn:StokesEquationStaticMomentum}
\end{equation}

The most general form of
\eqnthrurefs{eqn:StokesEquationMomentum}{eqn:StokesEquationStaticMomentum}, and
the form that will be considered for the rest of this section, is
\begin{multline}
  \fnof{\rho}{\vectr{x}}\delby{\fnof{\vectr{v}}{\vectr{x},t}}{t}
  -\fnof{\rho}{\vectr{x}}\dotprod{\fnof{\vectr{v}^{*}}{\vectr{x},t}}{
    \gradient{}{\fnof{\vectr{v}}{\vectr{x},t}}}
  +\dotprod{\gradient{}{\fnof{p}{\vectr{x},t}}}{\sharptensor{\tensor{g}}}\\
  -\divergence{}{\fnof{\mu}{\vectr{x}}\pbrac{\gradient{}{\fnof{\vectr{v}}{\vectr{x},t}+
        \transpose{\gradient{}{\fnof{\vectr{v}}{\vectr{x},t}}}}}}
  -\fnof{\vectr{s}}{\vectr{x},t}=\vectr{0}
  \label{eqn:StokesEquationGenMomentum}
\end{multline}
and
\begin{equation}
  \divergence{}{\fnof{\vectr{v}}{\vectr{x},t}}=0
  \label{eqn:StokesEquationGenMass}
\end{equation}

\subsection{Weak formulation}

The corresponding weak form of the equation system consisting of
\eqnref{eqn:StokesEquationGenMomentum} and \eqnref{eqn:StokesEquationGenMass}
can be found by applying standard Galerkin weight functions $\vectr{w}$ for
the velocity variables \ie
\begin{equation}
  \gint{\Omega}{}{\pbrac{\rho\delby{\vectr{v}}{t}-\rho\dotprod{\vectr{v}^{*}}{\gradient{}{\vectr{v}}}
      +\dotprod{\gradient{}{p}}{\sharptensor{\tensor{g}}}
      -\divergence{}{\mu\pbrac{\gradient{}{\vectr{v}+\transpose{\gradient{}{\vectr{v}}}}}}
      -\vectr{s}}{\vectr{w}}}{\Omega} = \vectr{0}
  \label{eqn:WeakFormStokesGenMomentum}
\end{equation}
and 
\begin{equation}
  \gint{\Omega}{}{\divergence{}{\vectr{v}}w^{p}}{\Omega} = 0
 \label{eqn:WeakFormStokesGenMass}
\end{equation}
where $w^{p}$ is the Galerkin weight function for the pressure variable.

Applying the divergence theorem to \eqnref{eqn:WeakFormStokesGenMomentum}
gives us the weak form of the momentum equations with the associated natural
boundary conditions at the boundary $\Gamma$ \ie
\begin{multline}
  \gint{\Omega}{}{\rho\delby{\vectr{v}}{t}{\vectr{w}}}{\Omega}
  -\gint{\Omega}{}{\dotprod{\rho\vectr{v}^{*}}{\gradient{}{\vectr{u}}}\vectr{w}}{\Omega}
  -\gint{\Omega}{}{\dotprod{p\sharptensor{\tensor{g}}}{\gradient{}{\vectr{w}}}}{\Omega}
  +\gint{\Omega}{}{\doubledotprod{\pbrac{\mu\pbrac{\gradient{}{\vectr{v}}
          +\gradient{}{\transpose{\vectr{v}}}}}}{\gradient{}{\vectr{w}}}}{\Omega} \\
  -\gint{\Omega}{}{\vectr{s}\vectr{w}}{\Omega}
  =\gint{\Gamma}{}{\dotprod{\pbrac{\mu\pbrac{\gradient{}{\vectr{v}}+\transpose{\gradient{}{\vectr{v}}}}-
        p\sharptensor{\vectr{g}}}}{\vectr{n}}\vectr{w}}{\Gamma}
  \label{eqn:GalerkinStokesGenMomentum}
\end{multline}

Dirichlet boundary conditions on a boundary
$\Gamma_D$ for velocity will take the form:
\begin{equation}
  \fnof{\vectr{v}}{\vectr{x},t} = \fnof{\vectr{v}_{D}}{\vectr{x},t} \quad \vectr{x}\in\Gamma_{D}\\
  \label{eqn:StokesVelocityDirichletBC} 
\end{equation}
and for pressure they will take the form
\begin{equation}
  \fnof{p}{\vectr{x},t} = \fnof{p_{D}}{\vectr{x},t} \quad \vectr{x}\in\Gamma_{D}\\
  \label{eqn:StokesPressureDirichletBC} 
\end{equation}

Neumann boundary conditions will consist of a pressure term and
viscous stress acting normal to a given boundary.
\begin{equation}
  \fnof{\vectr{q}}{\vectr{x},t} =
  \dotprod{\pbrac{\mu\pbrac{\gradient{}{\fnof{\vectr{v}}{\vectr{x},t}}+\transpose{\gradient{}{\fnof{\vectr{v}}{\vectr{x},t}}}}-
      \fnof{p}{\vectr{x},t}\sharptensor{\vectr{g}}}}{\fnof{\vectr{n}}{\vectr{x},t}}  \quad\quad \vectr{x}\in\Gamma_{N}
  \label{eqn:StokesNeumannBC}  
\end{equation}

\subsection{Tensor notation}

\Eqnref{eqn:GalerkinStokesGenMomentum} in tensor notation becomes
\begin{multline}
  \gint{\Omega}{}{\rho\dot{v}^{i}w^{i}}{\Omega}
 -\gint{\Omega}{}{\rho v^{*j}\covarderiv{v^{i}}{j}w^{i}}{\Omega}
 -\gint{\Omega}{}{pg^{ik}\covarderiv{w^{i}}{k}}{\Omega}\\
 +\gint{\Omega}{}{{\mu}g^{jk}\covarderiv{v^{i}}{k}\covarderiv{w^{i}}{j}}{\Omega}
 +\gint{\Omega}{}{{\mu}g^{ik}\covarderiv{v^{j}}{k}\covarderiv{w^{i}}{j}}{\Omega}\\
 -\gint{\Omega}{}{s^{i}w^{i}}{\Omega}
 =\gint{\Gamma_N}{}{\pbrac{\mu\pbrac{g^{jk}\covarderiv{v^{i}}{k}+g^{ik}\covarderiv{v^{j}}{k}}-pg^{ij}}n_{j}w^{i}}{\Gamma}
 \label{eqn:TensorStokesEquation}
\end{multline}
or
\begin{multline}
  \gint{\Omega}{}{\rho\dot{v}^{i}w^{i}}{\Omega}
 -\gint{\Omega}{}{\rho v^{*j}\pbrac{\partialderiv{v^{i}}{j}+\christoffel{i}{j}{h}v^{h}}w^{i}}{\Omega}
 -\gint{\Omega}{}{pg^{ik}\partialderiv{w^{i}}{k}}{\Omega}\\
 +\gint{\Omega}{}{{\mu}g^{jk}\pbrac{\partialderiv{v^{i}}{k}+\christoffel{i}{k}{h}v^{h}}\partialderiv{w^{i}}{j}}{\Omega}
 +\gint{\Omega}{}{{\mu}g^{ik}\pbrac{\partialderiv{v^{j}}{k}+\christoffel{j}{k}{h}v^{h}}\partialderiv{w^{i}}{j}}{\Omega}\\
 -\gint{\Omega}{}{s^{i}w^{i}}{\Omega}
 =\gint{\Gamma_N}{}{\pbrac{\mu\pbrac{g^{jk}\pbrac{\partialderiv{v^{i}}{k}+\christoffel{i}{k}{h}v^{h}}+g^{ik}\pbrac{\partialderiv{v^{j}}{k}+\christoffel{j}{k}{h}v^{h}}}-pg^{ij}}n_{j}w^{i}}{\Gamma}
 \label{eqn:Tensor2StokesEquation}
\end{multline}
or
\begin{multline}
  \gint{\Omega}{}{\rho\dot{v}^{i}w^{i}}{\Omega}
 -\gint{\Omega}{}{\rho v^{*j}\partialderiv{v^{i}}{j}w^{i}}{\Omega}
 -\gint{\Omega}{}{pg^{ik}\partialderiv{w^{i}}{k}}{\Omega}\\
 +\gint{\Omega}{}{\mu\pbrac{g^{jk}\partialderiv{v^{i}}{k}+g^{ik}\partialderiv{v^{j}}{k}}\partialderiv{w^{i}}{j}}{\Omega}
 -\gint{\Omega}{}{s^{i}w^{i}}{\Omega} \\
 -\gint{\Omega}{}{\rho v^{*j}\christoffel{i}{j}{h}v^{h}w^{i}}{\Omega}
 +\gint{\Omega}{}{\mu\pbrac{g^{jk}\christoffel{i}{k}{h}v^{h}+
     g^{ik}\christoffel{j}{k}{h}v^{h}}\partialderiv{w^{i}}{j}}{\Omega}\\
 =\gint{\Gamma_N}{}{\pbrac{\mu\pbrac{g^{jk}\partialderiv{v^{i}}{k}+g^{ik}\partialderiv{v^{j}}{k}}-pg^{ij}}n_{j}w^{i}}{\Gamma}
 +\gint{\Gamma_N}{}{\mu\pbrac{g^{jk}\christoffel{i}{k}{h}v^{h}+g^{ik}\christoffel{j}{k}{h}v^{h}}n_{j}w^{i}}{\Gamma}
 \label{eqn:Tensor3StokesEquation}
\end{multline}
where $g^{jk}$ is the contravariant metric tensor and $\christoffel{i}{j}{k}$
is the Christoffel symbol of the second kind for the spatial coordinates.

The conservation of mass equation can be written as
\begin{equation}
  \gint{\Omega}{}{\covarderiv{v^{i}}{i}w^{p}}{\Omega}=
  \gint{}{}{\pbrac{\partialderiv{v^{i}}{i}+\christoffel{i}{i}{j}v^{j}}w^{p}}{\Omega}=
  \gint{\Omega}{}{\partialderiv{v^{i}}{i}w^{p}}{\Omega}+\gint{\Omega}{}{\christoffel{i}{i}{j}v^{j}w^{p}}{\Omega}=0
  \label{eqn:TensorStokesMass}
\end{equation}

\subsection{Finite Element Formulation}

We can now discretise the domain into finite elements \ie
$\Omega=\displaystyle{\bigcup_{e=1}^{E}}\Omega_{e}$ with
$\Gamma=\displaystyle{\bigcup_{f=1}^{F}}\Gamma_{f}$. \Eqnref{eqn:Tensor3StokesEquation} now
becomes:
\begin{multline}
  \gsum{e=1}{E}{\gint{\Omega_{e}}{}{\rho\dot{v}^{i}w^{i}}{\Omega}}
 -\gsum{e=1}{E}{\gint{\Omega_{e}}{}{\rho v^{*j}\partialderiv{v^{i}}{j}{w^{i}}}{\Omega}}
 -\gsum{e=1}{E}{\gint{\Omega_{e}}{}{pg^{ik}\partialderiv{w^{i}}{k}}{\Omega}}\\
 +\gsum{e=1}{E}{\gint{\Omega_{e}}{}{\mu\pbrac{g^{jk}\partialderiv{v^{i}}{k}+
       g^{ik}\partialderiv{v^{j}}{k}}\partialderiv{w^{i}}{j}}{\Omega}}
 -\gsum{e=1}{E}{\gint{\Omega_{e}}{}{s^{i}w^{i}}{\Omega}}\\
 -\gsum{e=1}{E}{\gint{\Omega_{e}}{}{\rho
     v^{*j}\christoffel{i}{j}{h}v^{h}w^{i}}{\Omega}}
 +\gsum{e=1}{E}{\gint{\Omega_{e}}{}{\mu\pbrac{g^{jk}\christoffel{i}{k}{h}v^{h}+
     g^{ik}\christoffel{j}{k}{h}v^{h}}\partialderiv{w^{i}}{j}}{\Omega}}\\
 =\gsum{f=1}{F}{\gint{\Gamma_{N_{f}}}{}{\pbrac{\mu\pbrac{g^{jk}\partialderiv{v^{i}}{k}+
         g^{ik}\partialderiv{v^{j}}{k}}-pg^{ij}}n_{j}w^{i}}{\Gamma}}\\
 +\gsum{f=1}{F}{\gint{\Gamma_{N_{f}}}{}{\mu\pbrac{g^{jk}\christoffel{i}{k}{h}v^{h}+
       g^{ik}\christoffel{j}{k}{h}v^{h}}n_{j}w^{i}}{\Gamma}}
 \label{eqn:FEMStokesEquationMomentum}
\end{multline}

\Eqnref{eqn:TensorConvOfMass} now becomes
\begin{equation}
  \gsum{e=1}{E}{\gint{\Omega_{e}}{}{\partialderiv{v^{i}}{i}w^{p}}{\Omega}}
  +\gsum{e=1}{E}{\gint{\Omega_{e}}{}{\christoffel{i}{i}{j}v^{j}w^{p}}{\Omega}}=0
  \label{eqn:FEMStokesEquationMass}
\end{equation}

We will assume that the dependent variables $\vectr{v}$ and $p$ can be
interpolated separately in space and time \ie
\begin{gather}
  \fnof{\vectr{v}}{\vectr{x},t}=\gbfn{n}{}{\vectr{x}}\fnof{\nodept{\vectr{v}}{n}}{t}\\
  \fnof{p}{\vectr{x},t}=\altgbfn{n}{}{\vectr{x}}\fnof{\nodept{p}{n}}{t}
\end{gather}

In standard interpolation notation within an element, we transform from
$\vectr{x}$ to $\vectr{\xi}$ \ie
\begin{gather}
  \fnof{v^{i}}{\vectr{\xi},t}=\idxgbfn{i}{n}{\beta}{\vectr{\xi}}
  \fnof{\idxnodedof{v}{i}{n}{\beta}}{t}\idxgsf{i}{n}{\beta}\\
  \fnof{p}{\vectr{\xi},t}=\altgbfn{n}{\beta}{\vectr{\xi}}
  \fnof{\nodedof{p}{n}{\beta}}{t}\gsf{n}{\beta}
\end{gather}

For the natural boundary, we can separate $q^{i}$ into its component velocity
and pressure terms as noted in \ref{eqn:StokesNeumannBC} and shown in
\ref{eqn:FEMStokesEquationMomentum}. For our current treatment, we will also assume the values
of viscosity $\mu$ and density $\rho$ are constant. These can be interpolated:
\begin{equation}
  \begin{split}
    \fnof{q^{i}}{\vectr{\xi},t} &= \idxgbfn{i}{o}{\gamma}{\vectr{\xi}}
    \fnof{\idxnodedof{q}{i}{o}{\gamma}}{t}\idxgsf{i}{o}{\gamma} \\
    \fnof{v^{*i}}{\vectr{\xi},t} &=\idxgbfn{i}{d}{\delta}{\vectr{\xi}}
    \fnof{\idxnodedof{v}{*i}{d}{\delta}}{t}\idxgsf{i}{d}{\delta} \\
    \fnof{s^{i}}{\vectr{\xi},t} &=\idxgbfn{i}{e}{\epsilon}{\vectr{\xi}}
    \fnof{\idxnodedof{s}{i}{e}{\epsilon}}{t}\idxgsf{i}{e}{\epsilon} \\
    \fnof{\mu}{\vectr{\xi}} &=\gbfn{f}{\zeta}{\vectr{\xi}}
    \nodedof{\mu}{f}{\zeta}\gsf{f}{\zeta} \\
    \fnof{\rho}{\vectr{\xi}} &=\gbfn{g}{\eta}{\vectr{\xi}}\nodedof{\rho}{g}{\eta}
    \gsf{g}{\eta}
  \end{split}
\end{equation}
where $\fnof{\idxnodedof{q}{i}{o}{\gamma}}{t}$,
$\fnof{\idxnodedof{v}{*i}{d}{\delta}}{t}$, 
$\fnof{\idxnodedof{s}{i}{e}{\epsilon}}{t}$, $\nodedof{\mu}{f}{\zeta}$
and $\nodedof{\rho}{g}{\eta}$ are the nodal degrees-of-freedom for the
variables for component $i$ and time $t$.

The Galerkin weight functions are given by
\begin{equation}
  \fnof{w^{i}}{\vectr{\xi}}=\idxgbfn{i}{m}{\alpha}{\vectr{\xi}}\idxgsf{i}{m}{\alpha}
\end{equation}
and
\begin{equation}
  \fnof{w^{p}}{\vectr{\xi}}=\altgbfn{m}{\alpha}{\vectr{\xi}}\gsf{m}{\alpha}
\end{equation}

\subsection{Spatial Integration}

Using standard integration notation, we can rewrite our Galerkin FEM
formulation from \ref{eqn:FEM3StokesEquation}:
 \begin{multline}
   %time dependence
   \gsum{e=1}{E}{\gint{\vectr{0}}{\vectr{1}}{\fnof{\rho}{\vectr{\xi}}\delby{\idxgbfn{i}{n}{\beta}{\vectr{\xi}}
         \fnof{\idxnodedof{\dot{v}}{i}{n}{\beta}}{t}\idxgsf{i}{n}{\beta}}{t}\idxgbfn{i}{m}{\alpha}{\vectr{\xi}}\idxgsf{i}{m}{\alpha}\abs{\fnof{\matr{J}}{\vectr{\xi}}}}{\vectr{\xi}}}\\
   %ALE term
   -\gsum{e=1}{E}{\gint{\vectr{0}}{\vectr{1}}{\fnof{\rho}{\vectr{\xi}}
       \fnof{v^{*j}}{\vectr{\xi},t}\delby{\idxgbfn{i}{n}{\beta}{\vectr{\xi}}
         \fnof{\idxnodedof{v}{i}{n}{\beta}}{t}\idxgsf{i}{n}{\beta}}{x^{j}}\idxgbfn{i}{m}{\alpha}{\vectr{\xi}}\idxgsf{i}{m}{\alpha}\abs{\fnof{\matr{J}}{\vectr{\xi}}}}{\vectr{\xi}}}\\
   %pressure
   -\gsum{e=1}{E}{\gint{\vectr{0}}{\vectr{1}}{\fnof{g^{ik}}{\vectr{\xi}}\altgbfn{n}{\beta}{\vectr{\xi}}
       \fnof{\nodedof{p}{n}{\beta}}{t}\gsf{n}{\beta}\delby{\idxgbfn{i}{m}{\alpha}{\vectr{\xi}}\idxgsf{i}{m}{\alpha}}{x^{k}}\abs{\fnof{\matr{J}}{\vectr{\xi}}}}{\vectr{\xi}}}\\
   %viscous stress
   +\dsuml{e=1}{E}\dintl{\vectr{0}}{\vectr{1}}\fnof{\mu}{\vectr{\xi}}\pbrac{\fnof{g^{jk}}{\vectr{\xi}}\delby{\idxgbfn{i}{n}{\beta}{\vectr{\xi}}
     \fnof{\idxnodedof{v}{i}{n}{\beta}}{t}\idxgsf{i}{n}{\beta}}{x^{k}}
     +\fnof{g^{ik}}{\vectr{\xi}}\delby{\idxgbfn{j}{n}{\beta}{\vectr{\xi}}
     \fnof{\idxnodedof{v}{j}{n}{\beta}}{t}\idxgsf{j}{n}{\beta}}{x^{k}}} \\ \delby{\idxgbfn{i}{m}{\alpha}{\vectr{\xi}}\idxgsf{i}{m}{\alpha}}{x^{j}}\abs{\fnof{\matr{J}}{\vectr{\xi}}}\exteriorderiv{\vectr{\xi}} \\
   %body force
   -\gsum{e=1}{E}{\gint{\vectr{0}}{\vectr{1}}{\fnof{s^{i}}{\vectr{\xi},t}\idxgbfn{i}{m}{\alpha}{\vectr{\xi}}\idxgsf{i}{m}{\alpha}\abs{\fnof{\matr{J}}{\vectr{\xi}}}}{\vectr{\xi}}}\\
   %boundary terms
   -\gsum{f=1}{F}{\gint{\vectr{0}}{\vectr{1}}{\idxgbfn{i}{o}{\gamma}{\vectr{\xi}}
       \fnof{\idxnodedof{\bar{q}}{i}{o}{\gamma}}{t}\idxgsf{i}{o}{\gamma}\idxgbfn{i}{m}{\alpha}{\vectr{\xi}}\idxgsf{i}{m}{\alpha}\abs{\fnof{\matr{J}}{\vectr{\xi}}}}{\vectr{\xi}}}=0
   %ALE Christoffel
   -\gsum{e=1}{E}{
     \gint{\vectr{0}}{\vectr{1}}{
       \fnof{\rho}{\vectr{\xi}}\fnof{v^{*j}}{\vectr{\xi},t}\fnof{\christoffel{i}{j}{h}}{\vectr{\xi}}
       \idxgbfn{h}{n}{\beta}{\vectr{\xi}}\fnof{\idxnodedof{v}{h}{n}{\beta}}{t}\idxgsf{h}{n}{\beta}
       \idxgbfn{i}{m}{\alpha}{\vectr{\xi}}\idxgsf{i}{m}{\alpha}\abs{\fnof{\matr{J}}{\vectr{\xi}}}}{\vectr{\xi}}}\\
   %Viscous Christoffel
   +\dsuml{e=1}{E} \dintl{\vectr{0}}{\vectr{1}}
   \fnof{\mu}{\vectr{\xi}}\pbrac{\fnof{g^{jk}}{\vectr{\xi}}\fnof{\christoffel{i}{k}{h}}{\vectr{\xi}}
   \idxgbfn{h}{n}{\beta}{\vectr{\xi}}\fnof{\idxnodedof{v}{h}{n}{\beta}}{t}\idxgsf{h}{n}{\beta}
   +\fnof{g^{ik}}{\vectr{\xi}}\fnof{\christoffel{j}{k}{h}}{\vectr{\xi}}\idxgbfn{h}{n}{\beta}{\vectr{\xi}}
   \fnof{\idxnodedof{v}{h}{n}{\beta}}{t}\idxgsf{h}{n}{\beta}}\\
   \delby{\idxgbfn{i}{m}{\alpha}{\vectr{\xi}}\idxgsf{i}{m}{\alpha}}{x^{j}}
   \abs{\fnof{\matr{J}}{\vectr{\xi}}}\exteriorderiv{\vectr{\xi}}\\
   %boundary Christoffel
   +\gsum{f=1}{F}{\gint{\vectr{0}}{\vectr{1}}{\fnof{\mu}{\vectr{\xi}}
       \fnof{g^{jk}}{\vectr{\xi}}\fnof{\christoffel{i}{k}{h}}{\vectr{\xi}}\idxgbfn{h}{n}{\beta}{\vectr{\xi}}
       \fnof{\idxnodedof{v}{h}{n}{\beta}}{t}\idxgsf{h}{n}{\beta}\fnof{n_{j}}{\vectr{\xi}}
       \idxgbfn{i}{m}{\alpha}{\vectr{\xi}}\idxgsf{i}{m}{\alpha}\abs{\fnof{\matr{J}}{\vectr{\xi}}}}{\vectr{\xi}}}==0
   \label{eqn:FEM4StokesEquation}
\end{multline}
where the appropriate face bases are used for the surface integrals. Note that
$\fnof{\matr{J}}{\vectr{\xi}}$ represents the jacobian matrix from $\vectr{x}$
to $\vectr{\xi}$ coordinates.

Taking constant terms outside the integral signs and converting derivatives
with respect to $\vectr{x}$ to derivatives with respect to $\vectr{\xi}$ gives
\begin{multline}
  %time dependence
  \gsum{e=1}{E}{\idxgsf{i}{m}{\alpha}\idxgsf{i}{n}{\beta}\fnof{\idxnodedof{\dot{v}}{i}{n}{\beta}}{t}
    \gint{\vectr{0}}{\vectr{1}}{\fnof{\rho}{\vectr{\xi}}
      \idxgbfn{i}{m}{\alpha}{\vectr{\xi}}\idxgbfn{i}{n}{\beta}{\vectr{\xi}}
      \abs{\fnof{\matr{J}}{\vectr{\xi}}}}{\vectr{\xi}}}\\
  %ALE term
  -\gsum{e=1}{E}{\idxgsf{i}{m}{\alpha}\idxgsf{i}{n}{\beta}\fnof{\idxnodedof{v}{i}{n}{\beta}}{t}
    \gint{\vectr{0}}{\vectr{1}}{\fnof{\rho}{\vectr{\xi}}\fnof{v^{*j}}{\vectr{\xi},t}
      \idxgbfn{i}{m}{\alpha}{\vectr{\xi}}\delby{\idxgbfn{i}{n}{\beta}{\vectr{\xi}}}{\xi^{r}}
      \delby{\xi^{r}}{x^{j}}\abs{\fnof{\matr{J}}{\vectr{\xi}}}}{\vectr{\xi}}}\\
  %pressure
  -\gsum{e=1}{E}{\idxgsf{i}{m}{\alpha}\gsf{n}{\beta}\fnof{\nodedof{p}{n}{\beta}}{t}\gint{\vectr{0}}{\vectr{1}}{
      \fnof{g^{ik}}{\vectr{\xi}}\delby{\idxgbfn{i}{m}{\alpha}{\vectr{\xi}}}{\xi^{r}}\delby{\xi^{r}}{x^{k}}
      \altgbfn{n}{\beta}{\vectr{\xi}}\abs{\fnof{\matr{J}}{\vectr{\xi}}}}{\vectr{\xi}}}\\
  %viscous stress
  +\gsum{e=1}{E}{\idxgsf{i}{m}{\alpha}\idxgsf{i}{n}{\beta}\fnof{\idxnodedof{v}{i}{n}{\beta}}{t}\gint{\vectr{0}}{\vectr{1}}{
      \fnof{\mu}{\vectr{\xi}}\fnof{g^{jk}}{\vectr{\xi}}\delby{\idxgbfn{i}{m}{\alpha}{\vectr{\xi}}}{\xi^{r}}
      \delby{\xi^{r}}{x^{j}}\delby{\idxgbfn{i}{n}{\beta}{\vectr{\xi}}}{\xi^{s}}\delby{\xi^{s}}{x^{k}}
      \abs{\fnof{\matr{J}}{\vectr{\xi}}}}{\vectr{\xi}}}\\
  +\gsum{e=1}{E}{\idxgsf{i}{m}{\alpha}\idxgsf{j}{n}{\beta}\fnof{\idxnodedof{v}{j}{n}{\beta}}{t}
    \gint{\vectr{0}}{\vectr{1}}{\fnof{\mu}{\vectr{\xi}}\fnof{g^{ik}}{\vectr{\xi}}\delby{\idxgbfn{i}{m}{\alpha}{\vectr{\xi}}}{\xi^{r}}
      \delby{\xi^{r}}{x^{j}}\delby{\idxgbfn{j}{n}{\beta}{\vectr{\xi}}}{\xi^{s}}\delby{\xi^{s}}{x^{k}}
      \abs{\fnof{\matr{J}}{\vectr{\xi}}}}{\vectr{\xi}}} \\
  %body force
  -\gsum{e=1}{E}{\idxgsf{i}{m}{\alpha}\gint{\vectr{0}}{\vectr{1}}{\fnof{s^{i}}{\vectr{\xi},t}
      \idxgbfn{i}{m}{\alpha}{\vectr{\xi}}\abs{\fnof{\matr{J}}{\vectr{\xi}}}}{\vectr{\xi}}}\\
  %boundary terms
  -\gsum{f=1}{F}{\idxgsf{i}{m}{\alpha}\idxgsf{i}{o}{\gamma}\fnof{\idxnodedof{\bar{q}}{i}{o}{\gamma}}{t}
    \gint{\vectr{0}}{\vectr{1}}{\idxgbfn{i}{m}{\alpha}{\vectr{\xi}}\idxgbfn{i}{o}{\gamma}{\vectr{\xi}}
      \abs{\fnof{\matr{J}}{\vectr{\xi}}}}{\vectr{\xi}}}\\
  %ALE Christoffel
  -\gsum{e=1}{E}{\idxgsf{i}{m}{\alpha}\idxgsf{j}{n}{\beta}\fnof{\idxnodedof{v}{h}{n}{\beta}}{t}
    \gint{\vectr{0}}{\vectr{1}}{\fnof{\rho}{\vectr{\xi}}\fnof{\idxnodedof{v}{*j}{n}{\beta}}{t}
      \fnof{\christoffel{i}{j}{h}}{\vectr{\xi}}\idxgbfn{i}{m}{\alpha}{\vectr{\xi}}
      \idxgbfn{j}{n}{\beta}{\vectr{\xi}}\abs{\fnof{\matr{J}}{\vectr{\xi}}}}{\vectr{\xi}}}\\
  %Viscous Christoffel
  -\gsum{e=1}{E}{\idxgsf{i}{m}{\alpha}\idxgsf{h}{n}{\beta}\fnof{\idxnodedof{v}{h}{n}{\beta}}{t}
    \gint{\vectr{0}}{\vectr{1}}{\fnof{\mu}{\vectr{\xi}}\fnof{g^{jk}}{\vectr{\xi}}\fnof{\christoffel{i}{k}{h}}{\vectr{\xi}}
      \delby{\idxgbfn{i}{m}{\alpha}{\vectr{\xi}}}{\xi^{s}}\delby{\xi^{s}}{x^{j}}
      \idxgbfn{h}{n}{\beta}{\vectr{\xi}}\abs{\fnof{\matr{J}}{\vectr{\xi}}}}{\vectr{\xi}}}\\
  -\gsum{e=1}{E}{\idxgsf{i}{m}{\alpha}\idxgsf{h}{n}{\beta}\fnof{\idxnodedof{v}{h}{n}{\beta}}{t}
    \gint{}{}{\fnof{\mu}{\vectr{\xi}}\fnof{g^{ik}}{\vectr{\xi}}\fnof{\christoffel{j}{k}{h}}{\vectr{\xi}}
      \delby{\idxgbfn{i}{m}{\alpha}{\vectr{\xi}}}{\xi^{s}}\delby{\xi^{s}}{x^{j}}
      \idxgbfn{h}{n}{\beta}{\vectr{\xi}}\abs{\fnof{\matr{J}}{\vectr{\xi}}}}{\vectr{\xi}}}\\
  %Boundary Christoffel
  -\gsum{f=1}{F}{\idxgsf{i}{m}{\alpha}\idxgsf{h}{n}{\beta}\fnof{\idxnodedof{v}{h}{n}{\beta}}{t}
    \gint{\vectr{0}}{\vectr{1}}{\fnof{\mu}{\vectr{\xi}}\fnof{g^{jk}}{\vectr{\xi}}\idxgbfn{i}{m}{\alpha}{\vectr{\xi}}
      \fnof{\christoffel{i}{k}{h}}{\vectr{\xi}}\idxgbfn{h}{n}{\beta}{\vectr{\xi}}\fnof{n_{j}}{\vectr{\xi}}
      \abs{\fnof{\matr{J}}{\vectr{\xi}}}}{\vectr{\xi}}}=0
  \label{eqn:FEM4StokesEquation}
\end{multline}

If we assume a rectangular cartesian coordinate system, this simplifies
significantly, as the metric tensor $g^{jk}$ will become $\delta^{jk}$ and the
Christoffel symbols $\christoffel{i}{j}{k}$ will all be zero. This gives:
\begin{multline}
  %time dependence
  \gsum{e=1}{E}{\idxgsf{i}{m}{\alpha}\idxgsf{i}{n}{\beta}\fnof{\idxnodedof{\dot{v}}{i}{n}{\beta}}{t}
    \gint{\vectr{0}}{\vectr{1}}{\fnof{\rho}{\vectr{\xi}}
      \idxgbfn{i}{m}{\alpha}{\vectr{\xi}}\idxgbfn{i}{n}{\beta}{\vectr{\xi}}
      \abs{\fnof{\matr{J}}{\vectr{\xi}}}}{\vectr{\xi}}}\\
  %ALE term
  -\gsum{e=1}{E}{\idxgsf{i}{m}{\alpha}\idxgsf{i}{n}{\beta}\fnof{\idxnodedof{v}{i}{n}{\beta}}{t}
    \gint{\vectr{0}}{\vectr{1}}{\fnof{\rho}{\vectr{\xi}}\fnof{v^{*j}}{\vectr{\xi},t}
      \idxgbfn{i}{m}{\alpha}{\vectr{\xi}}\delby{\idxgbfn{i}{n}{\beta}{\vectr{\xi}}}{\xi^{r}}
      \delby{\xi^{r}}{x^{j}}\abs{\fnof{\matr{J}}{\vectr{\xi}}}}{\vectr{\xi}}}\\
  %pressure
  -\gsum{e=1}{E}{\idxgsf{i}{m}{\alpha}\gsf{n}{\beta}\fnof{\nodedof{p}{n}{\beta}}{t}\gint{\vectr{0}}{\vectr{1}}{
      \fnof{\delta^{ik}}{\vectr{\xi}}\delby{\idxgbfn{i}{m}{\alpha}{\vectr{\xi}}}{\xi^{r}}\delby{\xi^{r}}{x^{k}}
      \altgbfn{n}{\beta}{\vectr{\xi}}\abs{\fnof{\matr{J}}{\vectr{\xi}}}}{\vectr{\xi}}}\\
  %viscous stress
  +\gsum{e=1}{E}{\idxgsf{i}{m}{\alpha}\idxgsf{i}{n}{\beta}\fnof{\idxnodedof{v}{i}{n}{\beta}}{t}\gint{\vectr{0}}{\vectr{1}}{
      \fnof{\mu}{\vectr{\xi}}\fnof{\delta^{jk}}{\vectr{\xi}}\delby{\idxgbfn{i}{m}{\alpha}{\vectr{\xi}}}{\xi^{r}}
      \delby{\xi^{r}}{x^{j}}\delby{\idxgbfn{i}{n}{\beta}{\vectr{\xi}}}{\xi^{s}}\delby{\xi^{s}}{x^{k}}
      \abs{\fnof{\matr{J}}{\vectr{\xi}}}}{\vectr{\xi}}}\\
  +\gsum{e=1}{E}{\idxgsf{i}{m}{\alpha}\idxgsf{j}{n}{\beta}\fnof{\idxnodedof{v}{j}{n}{\beta}}{t}
    \gint{\vectr{0}}{\vectr{1}}{\fnof{\mu}{\vectr{\xi}}\fnof{\delta^{ik}}{\vectr{\xi}}\delby{\idxgbfn{i}{m}{\alpha}{\vectr{\xi}}}{\xi^{r}}
      \delby{\xi^{r}}{x^{j}}\delby{\idxgbfn{j}{n}{\beta}{\vectr{\xi}}}{\xi^{s}}\delby{\xi^{s}}{x^{k}}
      \abs{\fnof{\matr{J}}{\vectr{\xi}}}}{\vectr{\xi}}} \\
  %body force
  -\gsum{e=1}{E}{\idxgsf{i}{m}{\alpha}\gint{\vectr{0}}{\vectr{1}}{\fnof{s^{i}}{\vectr{\xi},t}
      \idxgbfn{i}{m}{\alpha}{\vectr{\xi}}\abs{\fnof{\matr{J}}{\vectr{\xi}}}}{\vectr{\xi}}}\\
  %boundary terms
  -\gsum{f=1}{F}{\idxgsf{i}{m}{\alpha}\idxgsf{i}{o}{\gamma}\fnof{\idxnodedof{\bar{q}}{i}{o}{\gamma}}{t}
    \gint{\vectr{0}}{\vectr{1}}{\idxgbfn{i}{m}{\alpha}{\vectr{\xi}}\idxgbfn{i}{o}{\gamma}{\vectr{\xi}}
      \abs{\fnof{\matr{J}}{\vectr{\xi}}}}{\vectr{\xi}}}=0\\
  \label{eqn:FEM4StokesEquation}
\end{multline}

\subsection{General form}

We now seek to assemble this into the corresponding general form for
dynamic equations, as outlined in \secref{sec:dynamicequations}:

\begin{equation}
  \matr{M}\fnof{\ddot{\vectr{d}}}{t}+\matr{C}\fnof{\dot{\vectr{d}}}{t}+\matr{K}\fnof{\vectr{d}}{t}+
  \fnof{\vectr{s}}{t}+\fnof{\vectr{f}}{t}=\vectr{0}
  \label{eqn:SEGeneralDynamic}
\end{equation}
where $\dot{\vectr{d}}(t)$ and $\ddot{\vectr{d}}(t)$ represent the first and
second derivatives (respectively) of the degrees of freedom vector
$\vectr{d}(t)$, which consists of the dependent variables
$\vectr{v}(\vectr{x},t)$ and $p(\vectr{x},t)$. $\matr{M}$ is the mass matrix, which
provides the shape function based weights and $\matr{C}$ is the transient
damping matrix. $\matr{K}$ represents the stiffness matrix, and $\fnof{\vectr{f}}{t}$ is the forcing vector.

We will assume cartesian coordinates $\vectr{x}=\transpose{\sqbrac{x^{1},x^{2},x^{3}}}$ and denote the
corresponding velocity components $\vectr{v}=\transpose{\sqbrac{v^{1},v^{2},v^{3}}}$, with $N$
representing the number of velocity DOFs and $M$ the number of pressure
DOFs. The vector $\fnof{\vectr{d}}{t}$ then becomes:
\begin{equation}
  \fnof{\vectr{d}}{t}=\begin{bmatrix}
  \fnof{\vectr{v}^{1}}{t} \\
  \fnof{\vectr{v}^{2}}{t} \\
  \fnof{\vectr{v}^{3}}{t} \\
  \fnof{\vectr{p}}{t}
  \end{bmatrix}
\end{equation}
and the vector $\fnof{\dot{\vectr{d}}}{t}$ then becomes:
\begin{equation}
  \fnof{\dot{\vectr{d}}}{t}=\begin{bmatrix}
  \fnof{\dot{\vectr{v}}^{1}}{t} \\
  \fnof{\dot{\vectr{v}}^{2}}{t} \\
  \fnof{\dot{\vectr{v}}^{3}}{t} \\
  \fnof{\dot{\vectr{p}}}{t}
  \end{bmatrix}
\end{equation}

All the components of \ref{eqn:SEGeneralDynamic} will depend on the number of dimensions, $n_{dim}$.

Returning to the general case, the Navier-Stokes equations do not have a mass
matrix $\matr{M}$ and so 
\begin{equation}
  \matr{M}=\matr{0}
\end{equation}

The elemental damping matrix $\matr{C}$ is given by
\begin{equation}
  %dynamic term
  C^{i\alpha\beta}_{mn}=\matr{C}^{i}=\idxgsf{i}{m}{\alpha}\idxgsf{i}{n}{\beta}
  \gint{\vectr{0}}{\vectr{1}}{\fnof{\rho}{\vectr{\xi}}
    \idxgbfn{i}{m}{\alpha}{\vectr{\xi}}\idxgbfn{i}{n}{\beta}{\vectr{\xi}}
    \abs{\fnof{\matr{J}}{\vectr{\xi}}}}{\vectr{\xi}}
\end{equation}
for $i=1,\ldots,n_{dim}$ and $\matr{C}^{n_{dim}+1}\equiv\matr{0}$.

For the stiffness matrix $\matr{K}$ we can combine the momentum a mass equations to give a matrix of the form
\begin{equation}
  \matr{K}=
   \begin{bmatrix}
     \matr{A}^{1}+\matr{D}^{11} & \matr{D}^{12} & \matr{D}^{13} & \matr{B}^{1}\\
     \matr{D}^{21} & \matr{A}^{2}+\matr{D}^{22} & \matr{D}^{23} & \matr{B}^{2}\\
     \matr{D}^{31} & \matr{D}^{32} & \matr{A}^{3}+\matr{D}^{13} & \matr{B}^{3}\\
     -\transpose{\matr{B}^{1}} & -\transpose{\matr{B}^{2}} & -\transpose{\matr{B}^{3}} & \matr{0}
   \end{bmatrix}
\end{equation}
where
\begin{multline}
  %ALE term
  A^{i\alpha\beta}_{mn}=\matr{A}^{i}= 
  -\gsum{e=1}{E}{\idxgsf{i}{m}{\alpha}\idxgsf{i}{n}{\beta}
    \gint{\vectr{0}}{\vectr{1}}{\fnof{\rho}{\vectr{\xi}}\fnof{v^{*j}}{\vectr{\xi},t}
      \idxgbfn{i}{m}{\alpha}{\vectr{\xi}}\delby{\idxgbfn{i}{n}{\beta}{\vectr{\xi}}}{\xi^{r}}
      \delby{\xi^{r}}{x^{j}}\abs{\fnof{\matr{J}}{\vectr{\xi}}}}{\vectr{\xi}}} \\
  %Viscous stress  
  +\idxgsf{i}{m}{\alpha}\idxgsf{i}{n}{\beta}\gint{\vectr{0}}{\vectr{1}}{
      \fnof{\mu}{\vectr{\xi}}\fnof{g^{jk}}{\vectr{\xi}}\delby{\idxgbfn{i}{m}{\alpha}{\vectr{\xi}}}{\xi^{s}}
      \delby{\xi^{s}}{x^{j}}\delby{\idxgbfn{i}{n}{\beta}{\vectr{\xi}}}{\xi^{r}}\delby{\xi^{r}}{x^{k}}
      \abs{\fnof{\matr{J}}{\vectr{\xi}}}}{\vectr{\xi}}\\
\end{multline}
where $i=1,\dots,n_{dim}$ and $\matr{A}^{n_{dim}+1}\equiv\matr{0}$.

Also,
\begin{equation}
  %Pressure term  
  B^{i\alpha\beta}_{mn}=\matr{B}^{i}=\idxgsf{i}{m}{\alpha}\gsf{n}{\beta}\gint{\vectr{0}}{\vectr{1}}{
      \fnof{g^{ik}}{\vectr{\xi}}\delby{\idxgbfn{i}{m}{\alpha}{\vectr{\xi}}}{\xi^{r}}\delby{\xi^{r}}{x^{k}}
      \altgbfn{n}{\beta}{\vectr{\xi}}\abs{\fnof{\matr{J}}{\vectr{\xi}}}}{\vectr{\xi}}
\end{equation}
where $i=1,\dots,n_{dim}$ and $\matr{B}^{n_{dim}+1}\equiv\matr{0}$.

Now
\begin{equation}
  D^{ij\alpha\beta}_{mn}=\matr{D}^{ij}=\bar{D}^{ij\alpha\beta}_{mn}+\tilde{D}^{ij\alpha\beta}_{mn}=\bar{\matr{D}}^{ij}+\tilde{\matr{D}}^{ij}
\end{equation}
where
\begin{equation}
  \bar{D}^{ij\alpha\beta}_{mn}=\bar{\matr{D}}^{ij}=\idxgsf{i}{m}{\alpha}\idxgsf{j}{n}{\beta}
  \gint{\vectr{0}}{\vectr{1}}{\fnof{\mu}{\vectr{\xi}}\fnof{g^{ik}}{\vectr{\xi}}\delby{\idxgbfn{i}{m}{\alpha}{\vectr{\xi}}}{\xi^{s}}
    \delby{\xi^{s}}{x^{j}}\delby{\idxgbfn{j}{n}{\beta}{\vectr{\xi}}}{\xi^{r}}\delby{\xi^{r}}{x^{k}}
    \abs{\fnof{\matr{J}}{\vectr{\xi}}}}{\vectr{\xi}}\\
\end{equation}
and
\begin{multline}
  %ALE term
  \tilde{D}^{ij\alpha\beta}_{mn}=\tilde{\matr{D}}^{ij}=
  -\gsum{e=1}{E}{\idxgsf{i}{m}{\alpha}\idxgsf{j}{n}{\beta}\fnof{\idxnodedof{v}{h}{n}{\beta}}{t}
    \gint{\vectr{0}}{\vectr{1}}{\fnof{\rho}{\vectr{\xi}}\fnof{\idxnodedof{v}{*j}{n}{\beta}}{t}
      \fnof{\christoffel{i}{j}{h}}{\vectr{\xi}}\idxgbfn{i}{m}{\alpha}{\vectr{\xi}}
      \idxgbfn{j}{n}{\beta}{\vectr{\xi}}\abs{\fnof{\matr{J}}{\vectr{\xi}}}}{\vectr{\xi}}}\\
  %Viscous terms  
  +\idxgsf{i}{m}{\alpha}\idxgsf{j}{n}{\beta}
  \dintl{\vectr{0}}{\vectr{1}}\fnof{\mu}{\vectr{\xi}}\pbrac{\fnof{g^{hk}}{\vectr{\xi}}
  \delby{\idxgbfn{i}{m}{\alpha}{\vectr{\xi}}}{\xi^{s}}\delby{\xi^{s}}{x^{h}}
  \fnof{\christoffel{i}{k}{j}}{\vectr{\xi}}\idxgbfn{j}{n}{\beta}{\vectr{\xi}}
  +\fnof{g^{ik}}{\vectr{\xi}}
  \delby{\idxgbfn{i}{m}{\alpha}{\vectr{\xi}}}{\xi^{s}}\delby{\xi^{s}}{x^{h}}
  \fnof{\christoffel{h}{k}{j}}{\vectr{\xi}}\idxgbfn{j}{n}{\beta}{\vectr{\xi}}} 
  \abs{\fnof{\matr{J}}{\vectr{\xi}}}\exteriorderiv{\vectr{\xi}}
\end{multline}
where $i,j=1,\dots,n_{dim}$ and $\matr{D}^{ij}\equiv\matr{0}\quad\forall\quad i,j=n_{dim}+1$.

The source vector is given by
\begin{equation}
  s^{i\alpha}_{m}=\vectr{s}^{i}=-\idxgsf{i}{m}{\alpha}\gint{\vectr{0}}{\vectr{1}}{\fnof{s^{i}}{\vectr{\xi},t}
      \idxgbfn{i}{m}{\alpha}{\vectr{\xi}}\abs{\fnof{\matr{J}}{\vectr{\xi}}}}{\vectr{\xi}}
\end{equation}
where $i=1,\dots,n_{dim}$ and $\vectr{s}^{n_{dim}+1}\equiv\vectr{0}$.

The boundary traction vector is given by
\begin{equation}
  f^{i\alpha}_{m}=\bar{f}^{i\alpha}_{m}+\tilde{f}^{i\alpha}_{m}
\end{equation}
where
\begin{equation}
  \bar{f}^{i\alpha}_{m}=\bar{\vectr{f}}^{i}=-\idxgsf{i}{m}{\alpha}
  \gint{\vectr{0}}{\vectr{1}}{\idxgbfn{i}{m}{\alpha}{\vectr{\xi}}\idxgbfn{i}{o}{\gamma}{\vectr{\xi}}\fnof{\idxnodedof{q}{i}{o}{\gamma}}{t}\idxgsf{i}{o}{\gamma}
    \abs{\fnof{\matr{J}}{\vectr{\xi}}}}{\vectr{\xi}}
\end{equation}
and
\begin{equation}
  \tilde{f}^{i\alpha}_{m}=\tilde{\vectr{f}}^{i}=-\gint{\vectr{0}}{\vectr{1}}{\fnof{\mu}{\vectr{\xi}}
       \fnof{g^{jk}}{\vectr{\xi}}\fnof{\christoffel{i}{k}{h}}{\vectr{\xi}}\idxgbfn{h}{n}{\beta}{\vectr{\xi}}
       \fnof{\idxnodedof{v}{h}{n}{\beta}}{t}\idxgsf{h}{n}{\beta}\fnof{n_{j}}{\vectr{\xi}}
       \idxgbfn{i}{m}{\alpha}{\vectr{\xi}}\idxgsf{i}{m}{\alpha}\abs{\fnof{\matr{J}}{\vectr{\xi}}}}{\vectr{\xi}}
\end{equation}
where $i=1,\dots,n_{dim}$ and $\vectr{f}^{n_{dim}+1}\equiv\vectr{0}$.

The final system is given by
\begin{multline}
  \begin{bmatrix}
    \matr{C}^{1} & \matr{0} & \matr{0} & \matr{0} \\
    \matr{0} & \matr{C}^{2} & \matr{0} & \matr{0} \\
    \matr{0} & \matr{0} & \matr{C}^{3} & \matr{0} \\
    \matr{0} & \matr{0} & \matr{0} & \matr{0}
  \end{bmatrix}\begin{bmatrix}
      \fnof{\dot{\vectr{v}}^{1}}{t} \\
      \fnof{\dot{\vectr{v}}^{2}}{t} \\
      \fnof{\dot{\vectr{v}}^{3}}{t} \\
      \vectr{0}
  \end{bmatrix}\\
  +\begin{bmatrix}
     \matr{A}^{1}+\matr{D}^{11} & \matr{D}^{12} & \matr{D}^{13} & \matr{B}^{1}\\
     \matr{D}^{21} & \matr{A}^{2}+\matr{D}^{22} & \matr{D}^{23} & \matr{B}^{2}\\
     \matr{D}^{31} & \matr{D}^{32} & \matr{A}^{3}+\matr{D}^{13} & \matr{B}^{3}\\
     -\transpose{\matr{B}^{1}} & -\transpose{\matr{B}^{2}} & -\transpose{\matr{B}^{3}} & \matr{0}
  \end{bmatrix}\begin{bmatrix}
    \fnof{\vectr{v}^{1}}{t} \\
    \fnof{\vectr{v}^{2}}{t} \\
    \fnof{\vectr{v}^{3}}{t} \\
    \fnof{\vectr{p}}{t}
  \end{bmatrix}\\
  +\begin{bmatrix}
    \fnof{\vectr{s}^{1}}{t} \\
    \fnof{\vectr{s}^{2}}{t} \\
    \fnof{\vectr{s}^{3}}{t} \\
    \vectr{0}
  \end{bmatrix}+\begin{bmatrix}
    \fnof{\vectr{f}^{1}}{t} \\
    \fnof{\vectr{f}^{2}}{t} \\
    \fnof{\vectr{f}^{3}}{t} \\
    \vectr{0}
  \end{bmatrix}=\begin{bmatrix}
  \vectr{0} \\
  \vectr{0} \\
  \vectr{0} \\
  \vectr{0}
  \end{bmatrix}
\end{multline}
