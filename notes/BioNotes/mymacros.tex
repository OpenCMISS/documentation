%
% mymacros.tex
%


\newcommand{\vertnum}[1]{\makebox(0,0)[t]{\footnotesize{$#1$}}}



%%
%% Shortcuts
%%

%\renewcommand{\exp}[1]{\ensuremath{\negmedspace\times\negmedspace\tento{#1}}\xspace} 

\newcommand{\ital}[1]{\text{\emph{#1}}}
\newcommand{\fe}{finite element\xspace}
\newcommand{\FE}{Finite element\xspace }

\newcommand{\rd}[1]{\ensuremath{#1^{\text{rd}}}} %^rd eg\nrd{n} => n^rd
\newcommand{\nd}[1]{\ensuremath{#1^{\text{nd}}}} %^nd eg\nnd{n} => n^nd
\newcommand{\st}[1]{\ensuremath{#1^{\text{st}}}} %^st eg\nst{n} => n^st

\newcommand{\dof}{degrees of freedom\xspace}

%% max. absolute potential diff
%% min. absolute potential diff.
%% max. absolute percentage potential difference   
%% min. absolut  percentage potential difference   
\newcommand{\maxpotdiff}{Max. $\abs{\triangle \Phi}$}
\newcommand{\minpotdiff}{Min. $\abs{\triangle \Phi}$}
\newcommand{\maxpercpotdiff}{Max. $\triangle \Phi \%$} 
\newcommand{\minpercpotdiff}{Min. $\triangle \Phi \%$}


%%
%% Note: the $ symbols is used to turn OFF the math font
%%
\newcommand{\scft}{$\sc\footnotesize$}
\newcommand{\mvup}{\vspace{-2.5mm}}
\newcommand{\mvupb}{\vspace{-0.3mm}}
\newcommand{\mvdwn}{\xspace\smallskip}
\newcommand{\mvdwnb}{\xspace\medskip}
\newcommand{\mynegsp}{-2.5mm}
\newcommand{\spacetables}{\vspace{10mm}}
\newcommand{\spaceheading}{\vspace{3mm}}
\newcommand{\mvleft}{\hspace{-3.2mm}}



\newcommand{\txtscft}[1]{\vspace{-2mm}$\textsc{\footnotesize\begin{spacing}{0.5}{#1}\end{spacing}}$}
\newcommand{\txtscftN}[1]{\vspace{-2mm}\textsc{\footnotesize{#1}}}
\newcommand{\txttable}[1]{\footnotesize{#1}}

\newcommand{\tableheading}[1]{\subsubsection{#1}\spaceheading}



%%
%% Maths Macros
%%

\newcommand{\evdelby}[3]{\ensuremath{
    \brac{.}{\dfrac{\del{#1}}{\del{#2}}}{|}_{#3}
    }} %e.g. \evdelby{a}{b}{c}  \delby{a}{b}|_{c}

\newcommand{\evdeltwoby}[4]{\ensuremath{ 
    \brac{.}{\dfrac{\del^{2} #1}{\del #2 \del #3}}{|}_{#4}
    }} % e.g. \delnby{a}{b}{c}{d} => del^2 a / del b del c |d

\newcommand{\delntwonby}[6]{\ensuremath{ %
    \dfrac{ \del^{#1} #2}
    {\del^#3 {#4} \medspace\del^#5 {#6}} %
    }} % e.g. \delntwoby{a}{u}{x}{b}{y}{c} => del^a u / del^x b del^y c


%%
%% Boolean Macros
%%


\newcommand{\printquality}[2]{
  \ifthenelse{\boolean{@printdraft}}
  {#1}{#2}
  }
% Logical switch to determine print quality 
%  (should be generalised for diff variables?)


\newcommand{\iftwoside}[1]{
  \ifthenelse{\boolean{@twoside}}
  {#1}{}
  }
% Logical. Only does #1 if printing two sided. 
%   ie. gaps between chapters



%%
%% Floats Macros
%%

\newcommand{\mypstexfigure}[5]{ %
  \begin{figure}[{#2}] \centering %
    \input{#1} %
    \ifthenelse{\equal{#2}{}}{ %
      \caption{#4}}{ %
      \caption[#3]{#4} %
      } %
    \label{#5} %
  \end{figure} %
  } 
%  \pstexfigure{figure}
%              {htbp}
%              {short caption}
%              {long caption}
%              {fig:blah}
%


%
% boxed equation - with no numbering or numbering inside the box
% Lookup ...
%       Subject: Re: How to frame a formula defined in multline environment ?  
%       Author: Michael John Downes <epsmjd@epsilon.ams.org>
%       Date: 2000/01/31
%       Forum: comp.text.tex
% for more info.
% \newcommand{\boxedeqn}[1]{%
%   \[\fbox{%
%       \addtolength{\linewidth}{-2\fboxsep}%
%       \addtolength{\linewidth}{-2\fboxrule}%
%       \begin{minipage}{\linewidth}%
%       \begin{equation}#1\end{equation}%
%       \end{minipage}%
%     }\]%
%
%
% Usage: 
%\boxedeqn{0.6\linewidth}
%               {\begin{equation}
%                       x=y
%               \end{equation}}


\newcommand{\boxedeqn}[2]{
        \begin{center}
                \fbox
                {
                        \addtolength{\linewidth}{-2\fboxsep}
                        \addtolength{\linewidth}{-2\fboxrule}
                        \begin{minipage}{#1}
                        #2
                        \end{minipage}
                }
        \end{center}
}







\newcommand{\twopicts}[7]{\begin{figure}[ht] \centering
  \mbox{\subfigure[{#3}]
    {\input{#1}}\quad
    \subfigure[{#4}]
    {\input{#2}}}\caption[#5]{#6}\label{#7}
  \end{figure}}
% eg \twopicts{pic1}{pic2}{subcap1}{subcap2}{shortcap}{longcap}{fig:blah}


\newcommand{\fourpicts}[9]{\begin{figure}[ht] \centering
  \mbox{\subfigure[]
      {\epsfig{file=#1,height=#5,width=#6}}\qquad\qquad
      \subfigure[]
      {\epsfig{file=#2,height=#5,width=#6}}
    } \vspace{3pt} 
    \mbox{
      \subfigure[]
      {\epsfig{file=#3,height=#5,width=#6}}\qquad\qquad
      \subfigure[]
      {\epsfig{file=#4,height=#5,width=#6}}
    }
    \caption[#7]{#8}
    \label{#9}
  \end{figure}}
%eg \fourpicts{pic1}{pic2}{pic3}{pic4}{height}{width}{shortcap}{longcap}{fig:}



%%% Local Variables: 
%%% mode: latex
%%% TeX-master: t
%%% End: 
