%
% opencmiss_symbols.tex
%
% Global OpenCMISS symbol definitions i.e., assigning the symbol used to a latexname for a physical quantities
% (i.e., non-mathematical symbols).
%

%
% General
%

\newcommand{\gravitysymbol}{\ensuremath{ %
    a %
}} % gravity symbol e.g., \gravitysymbol => a
\newcommand{\gravityvector}{\ensuremath{ %
    \vectr{\gravitysymbol} %
}} % gravity vector e.g., \gravityvector => a
\newcommand{\voltagesymbol}{\ensuremath{ %
    V %
}} % voltage symbol e.g., \voltagesymbol => V
\newcommand{\currentsymbol}{\ensuremath{ %
    I %
}} % current symbol e.g., \currentsymbol => I
\newcommand{\currentdensitysymbol}{\ensuremath{ %
    J %
}} % current density symbol e.g., \currentdensitysymbol => J
\newcommand{\currentdensityvector}{\ensuremath{ %
    \vectr{J} %
}} % current density vector e.g., \currentdensityvector => J
\newcommand{\potentialsymbol}{\ensuremath{ %
    \phi %
}} % potential symbol e.g., \potentialsymbol => V
\newcommand{\conductivitytensorsymbol}{\ensuremath{ %
    \sigma %
}} % conductivity tensor symbol e.g., \conductivitytensorsymbol => \sigma
\newcommand{\conductivitytensor}{\ensuremath{ %
    \tensortwo{\conductivitytensorsymbol} %
}} % conductivity tensor e.g., \conductivitytensor => \sigma

%
% Interpolation
%

\newcommand{\xione}{\ensuremath{\xi^{1}}\xspace} % xi 1
\newcommand{\xionesq}{\ensuremath{{\xi^{1}}^{2}}\xspace} % (xi 1)^2
\newcommand{\xionecube}{\ensuremath{{\xi^{1}}^{3}}\xspace} % (xi 1)^3
\newcommand{\xitwo}{\ensuremath{\xi^{2}}\xspace} % xi 2
\newcommand{\xitwosq}{\ensuremath{{\xi^{2}}^{2}}\xspace} % (xi 2)^2
\newcommand{\xitwocube}{\ensuremath{{\xi^{2}}^{3}}\xspace} % (xi 2)^3
\newcommand{\xithree}{\ensuremath{\xi^{3}}\xspace} % xi 3
\newcommand{\xithreesq}{\ensuremath{{\xi^{3}}^{2}}\xspace} % (xi 3)^2
\newcommand{\xithreecube}{\ensuremath{{\xi^{3}}^{3}}\xspace} % (xi 3)^3

%
% Solid Mechanics
%

% Constants

\newcommand{\poissonsratiosymbol}{\ensuremath{ %
    \nu %
}} % Poisson's ratio symbol e.g., \poissonsratiosymbol => \nu
\newcommand{\bulkstrainmodulussymbol}{\ensuremath{ %
    K %
}} % Bulk strain modulus symbol e.g., \bulkstrainmodulussymbol => K
\newcommand{\soliddensitysymbol}{\ensuremath{ %
    \densitysymbol_{s} %
}} % Solid density symbol e.g., \soliddensitysymbol => \rho_s


% Deformation and strain

\newcommand{\deformationgradienttensorsymbol}{\ensuremath{ %
    F %
}} % Deformation gradient tensor symbol e.g., \deformationgradienttensorsymbol => F
\newcommand{\deformationgradienttensor}{\ensuremath{ %
    \tensortwo{\deformationgradienttensor} %
}} % Deformation gradient tensor e.g., \deformationgradienttensor => F
\newcommand{\rightstretchtensorsymbol}{\ensuremath{ %
    U %
}} % Right stretch tensor symbol e.g., \rightstretchtensorsymbol => U
\newcommand{\rightstretchtensor}{\ensuremath{ %
    \tensortwo{\rightstretchtensorsymbol} %
}} % Right stretch tensor e.g., \rightstretchtensor => U
\newcommand{\leftstretchtensorsymbol}{\ensuremath{ %
    V %
}} % Left stretch tensor symbol e.g., \leftstretchtensorsymbol => V
\newcommand{\leftstretchtensor}{\ensuremath{ %
    \tensortwo{\leftstretchtensorsymbol} %
}} % Left stretch tensor e.g., \leftstretchtensor => V
\newcommand{\rightcauchygreentensorsymbol}{\ensuremath{ %
    C %
}} % Right Cauchy-Green (or Green) deformation tensor symbol e.g., \rightcauchygreentensorsymbol => C
\newcommand{\rightcauchygreentensor}{\ensuremath{ %
    \tensortwo{\rightcauchygreentensorsymbol} %
}} % Right Cauchy-Green (or Green) deformation tensor e.g., \rightcauchygreentensor => C
\newcommand{\pioladeformationtensorsymbol}{\ensuremath{ %
    B %
}} % Piola deformation tensor symbol e.g., \pioladefomationtensorsymbol => B
\newcommand{\pioladeformationtensor}{\ensuremath{ %
    \tensortwo{\pioladeformationtensorsymbol} %
}} % Piola deformation tensor e.g., \pioladefomationtensor => B
\newcommand{\greenlagrangestraintensorsymbol}{\ensuremath{ %
    E %
}} % Green-Lagrange strain tensor symbol e.g., \greenlagrangestraintensorsymbol => E
\newcommand{\greenlagrangestraintensor}{\ensuremath{ %
    \tensortwo{\greenlagrangetensorsymbol} %
}} % Green-Lagrange strain tensor e.g., \greenlagrangestraintensor => B
\newcommand{\leftcauchygreentensorsymbol}{\ensuremath{ %
    b %
}} % Left Cauchy-Green (or Finger) deformation tensor symbol e.g., \leftcauchygreentensorsymbol => b
\newcommand{\leftcauchygreentensor}{\ensuremath{ %
    \tensortwo{\leftcauchygreentensorsymbol} %
}} % Left Cauchy-Green (or Finger) deformation tensor e.g., \leftcauchygreentensor => b
\newcommand{\cauchydeformationtensorsymbol}{\ensuremath{ %
    c %
}} % Cauchy deformation tensor symbol e.g., \cauchydeformationtensorsymbol => c
\newcommand{\cauchydeformationtensor}{\ensuremath{ %
    \tensortwo{\cauchydeformationtensorsymbol} %
}} % Cauchy deformation tensor e.g., \cauchydeformationtensor => c
\newcommand{\euleralmansistraintensorsymbol}{\ensuremath{ %
    e %
}} % Euler-Almansi strain tensor symbol e.g., \euleralmansistraintensorsymbol => e
\newcommand{\euleralmansistraintensor}{\ensuremath{ %
    \tensortwo{\euleralmansistraintensorsymbol} %
}} % Euler-Almansi strain tensor e.g., \euleralmansistraintensor => e

% Stress

\newcommand{\firstpiolakirchofftensorsymbol}{\ensuremath{ %
    P %
}} % First Piola-Kirchoff stress tensor symbol e.g., \firstpiolakirchoffstresstensorsymbol => P
\newcommand{\firstpiolakirchoffstresstensor}{\ensuremath{ %
    \tensortwo{\firstpiolakirchoffstresstensorsymbol} %
}} % First Piola-Kirchoss stress tensor e.g., \firstpiolakirchoffstresstensor => P
\newcommand{\secondpiolakirchofftensorsymbol}{\ensuremath{ %
    S %
}} % Second Piola-Kirchoff stress tensor symbol e.g., \secondpiolakirchoffstresstensorsymbol => S
\newcommand{\secondpiolakirchoffstresstensor}{\ensuremath{ %
    \tensortwo{\secondpiolakirchoffstresstensorsymbol} %
}} % Second Piola-Kirchoss stress tensor e.g., \secondpiolakirchoffstresstensor => S
\newcommand{\cauchystresstensorsymbol}{\ensuremath{ %
    \sigma %
}} % Cauchy stress tensor symbol e.g., \cauchystresstensorsymbol => \sigma
\newcommand{\cauchystresstensor}{\ensuremath{ %
    \tensortwo{\cauchystresstensorsymbol} %
}} % Cauchy stress tensor e.g., \cauchystresstensor => \sigma
\newcommand{\kirchoffstresstensorsymbol}{\ensuremath{ %
    \tau %
}} % Kirchoff stress tensor symbol e.g., \kirchoffstresstensorsymbol => \tau
\newcommand{\kirchoffstresstensor}{\ensuremath{ %
    \tensortwo{\kirchoffstresstensorsymbol} %
}} % Kirchoff stress tensor e.g., \kirchoffstresstensor => \tau

% Elasticity

\newcommand{\materialfirstelasticitytensorsymbol}{\ensuremath{ %
    A %
}} % Material first elasticity tensor symbol e.g., \materialfirstelasticitytensorsymbol => A
\newcommand{\materialfirstelasticitytensor}{\ensuremath{ %
    \tensorfour{\materialfirstelasticitytensorsymbol} %
}} % Material first elasticity tensor e.g., \materialfirstelasticitytensor => A
\newcommand{\materialsecondelasticitytensorsymbol}{\ensuremath{ %
    C %
}} % Material second elasticity tensor symbol e.g., \materialsecondelasticitytensorsymbol => C
\newcommand{\materialsecondelasticitytensor}{\ensuremath{ %
    \tensorfour{\materialsecondelasticitytensorsymbol} %
}} % Material second elasticity tensor e.g., \materialsecondelasticitytensor => C
\newcommand{\spatialfirstelasticitytensorsymbol}{\ensuremath{ %
    a %
}} % Spatial first elasticity tensor symbol e.g., \spatialfirstelasticitytensorsymbol => a
\newcommand{\spatialfirstelasticitytensor}{\ensuremath{ %
    \tensorfour{\spatialfirstelasticitytensorsymbol} %
}} % Spatial first elasticity tensor e.g., \spatialfirstelasticitytensor => a
\newcommand{\spatialsecondelasticitytensorsymbol}{\ensuremath{ %
    c %
}} % Spatial second elasticity tensor symbol e.g., \spatialsecondelasticitytensorsymbol => c
\newcommand{\spatialsecondelasticitytensor}{\ensuremath{ %
    \tensorfour{\spatialsecondelasticitytensorsymbol} %
}} % Spatial second elasticity tensor e.g., \spatialsecondelasticitytensor => c

%
% Bioelectrics
%

\newcommand{\membraneareavolumeratio}{\ensuremath{ %
    A_{m} %
}} % membrane area to volume ratio e.g., \membraneareavolumeration => A_m
\newcommand{\membranecapacitance}{\ensuremath{ %
    C_{m} %
}} % membrane capacitance e.g., \membranecapacitance => C_m
\newcommand{\transmembranevoltage}{\ensuremath{ %
    {\voltagesymbol}_{m} %
}} % transmembrane voltage e.g., \transmembranevoltage => V_m
\newcommand{\transmembranecurrent}{\ensuremath{ %
    {\currentsymbol}_{m} %
}} % transmembrane current e.g., \transmembranecurrent => I_m
\newcommand{\ioniccurrent}{\ensuremath{ %
    {\currentsymbol}_{\text{ion}} %
}} % Ionic current e.g., \ioniccurrent => I_ion
\newcommand{\stimuluscurrent}{\ensuremath{ %
    {\currentsymbol}_{s} %
}} % stimulus current e.g., \stimulus current => I_s
\newcommand{\intracellularpotential}{\ensuremath{ %
    {\potentialsymbol}_{i} %
}} % intra-cellular potential e.g., \intracellularpotential => \phi_i
\newcommand{\intracellularstimuluscurrent}{\ensuremath{ %
    {\stimuluscurrent}_{i} %
}} % intra-cellular stimulus current e.g., \intracellularstimuluscurrent => {I_s}_i
\newcommand{\intracellularcurrentdensityvector}{\ensuremath{ %
    {\currentdensityvector}_{i} %
}} % intra-cellular current density vector e.g., \intracellularcurrentdensityvector => J_i
\newcommand{\intracellularconductivitytensor}{\ensuremath{ %
    {\conductivitytensor}_{i} %
}} % intra-cellular conductivity tensor e.g., \intracellularconductivitytensor => \sigma_i
\newcommand{\extracellularpotential}{\ensuremath{ %
    {\potentialsymbol}_{e} %
}} % extra-cellular potential e.g., \extracellularpotential => \phi_e
\newcommand{\extracellularstimuluscurrent}{\ensuremath{ %
    {\stimuluscurrent}_{e} %
}} % extra-cellular stimulus current e.g., \extracellularstimuluscurrent => {I_s}_e
\newcommand{\extracellularcurrentdensityvector}{\ensuremath{ %
    {\currentdensityvector}_{e} %
}} % extra-cellular current density vector e.g., \extracellularcurrentdensityvector => J_e
\newcommand{\extracellularconductivitytensor}{\ensuremath{ %
    {\conductivitytensor}_{e} %
}} % extra-cellular conductivity tensor e.g., \extracellularconductivitytensor => \sigma_e
\newcommand{\monodomainconductivitytensor}{\ensuremath{ %
    {\conductivitytensor}_{m} %
}} % monodomain conductivity tensor e.g., \monodomainconductivitytensor => \sigma_m
